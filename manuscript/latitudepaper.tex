\documentclass[12pt]{article}  
\usepackage{amsmath}
\usepackage{url}
\usepackage[dvips]{graphicx}
\usepackage{multirow}
\usepackage{geometry}
\usepackage{pdflscape}
%\usepackage{rotating}
% make Figure 1 etc bold
\usepackage[labelfont=bf]{caption}
\usepackage{setspace}

\usepackage[running]{lineno}

% let's get some nature formatted citations
%\usepackage{overcite}
\usepackage[round]{natbib}

% some cheats to reduce the need to type complicated bits and pieces
\newcommand{\expect}[1]{\left\langle #1 \right\rangle}
\newcommand{\etal}{\textit{et al.\ }}

\newcommand{\beginsupplement}{%
        \setcounter{table}{0}
        \renewcommand{\thetable}{S\arabic{table}}%
        \setcounter{figure}{0}
        \renewcommand{\thefigure}{S\arabic{figure}}%
     }

% the abstract formatting
\newenvironment{sciabstract}{%
\begin{quote} \bf}
{\end{quote}}
\renewcommand\refname{References}

% margin sizes`
\topmargin 0.0cm
\oddsidemargin 0.2cm

\textwidth 16cm 
\textheight 21cm
\footskip 1.0cm


\title{Generality in food webs scales with species richness, not latitude}
\author{Alyssa Cirtwill, Tamara Romanuk?, Daniel Stouffer?}
\date{$^1$School of Biological Sciences\\University of Canterbury\\
Private Bag 4800\\Christchurch 8140, New Zealand}

\begin{document}
\maketitle
\baselineskip=8.5mm
 
\vspace{0.4 in}

%%%%%%%%%%%%%%%%%%%%%%%%%%%%%%%%%%%%%%%%%%%%%%%%%%%%%%%%%%%%%%%%%%%%%%%%%%%%%%%%%%%%%%%%%%%%%%%%%%%%%%%%%%%%%%%%%%
%%%%%%%%%%%%%%%%%%%%%%%%%%%%%%%%%%%%%%%%%%%%%%%%%%%%%%%%%%%%%%%%%%%%%%%%%%%%%%%%%%%%%%%%%%%%%%%%%%%%%%%%%%%%%%%%%%
%%
%%    Look into non-linear regressions for all properties (nls?)
%%%%%%%%%%%%%%%%%%%%%%%%%%%% 

\section*{Introduction}

%Food webs are a thing. (Mention gen, vul, LS.) %Food webs have strong scaling structure.
Food webs --networks of feeding links between species-- have been used for over a century to summarize the structure of 
ecological communities \citep{Williams2000,earlierwork} and to understand how that structure relates to environmental variables 
such as land use \citep{Townsend1998,Townsend2004c,Digel2014}, primary productivity \citep{}, and temperature. The latter
variables in turn have strong gradients over latitude, with productivity and temperature both being higher in the tropics \citep{}, 
suggesting that latitude may indirectly affect food-web structure.
Indeed, these latitudinal environmental gradients have been put forward as potential driverd for the latitudinal gradient in species 
richness, one of the most general and robust patterns in ecology \citep{Schemske2009a,,}.


This effect of latitude on species richness, however, presents a problem for the analysis of other aspects of food-web structure.
Although it was initially thought that food-web structure was small-world and scale invariant \citep{}, further 
analyses with higher-quality data have revealed that most food-web structural properties scale strongly with species 
richness, connectance (the proportion of possible links that are realized), or both \citep{Riede2010} 
(see \citet{Dunne2006} for a review of the history of scaling laws in food web theory). This means that any relationship
between latitude and food-web structure must be considered in light of the latitudinal gradient in species richness.
This is especially true for food-web properties that both scale with species richness and are related to candidate
explanations for the latitudinal diversity gradient, such as measures of specialization [[DBS requests citation, ARC thinks this is the motivation for the paper and isn't aware of a source]].


[[Smoother transition needed here]]
% Explain niche space hypothesis, Schemske, Vazquez. Then make clear that niche space suggests opposite gradient to ~S^B.
%Then give hypotheses.
A latitudinal gradient in specialization lies behind the ``latitude-niche breadth hypothesis'' \citep{Vazquez2004}.
This hypothesis predicts that decreased seasonality in the tropics should lead to more stable populations, which in
turn should evolve smaller niches \citep{Vazquez2004}. These narrow niches should therefore allow more species to 
coexist in the tropics than can at higher latitudes. Alternatively, higher productivity of the tropics \citep{Brown2004}
may result in a broader niche space \citep{Davies2007} which could also allow greater diversification even if niche 
sizes are globally similar. Although the assumptions of the latitude-niche breadth hypothesis are only equivocably 
supported \citep{Vazquez2004}, it remains a compelling potential mechanism for the latitudinal gradient in species 
richness \citep{Lappalainen2006,Krasnov2008,Slove2010}. 


Importantly, the expectation of the latitude-niche breadth hypothesis that higher species richness is associated with 
reduced competition is contary to the expected increase in the mean number of interactions per species (one potential 
measure of specialization) as species richness increases \citep{Dunne2006,Riede2010}. If, therefore, there is a
distinct effect of latitude on specialization then this effect should be visible in changing scaling relationships.
Here, we tested for changes in the scaling of mean links per species, mean generality, and mean vulnerability with species
richness over latitude. We focused on these three food-web properties --only some of the wide array of metrics that have been
described (see e.g. \citep{})-- because together they provide a measure a species' biotic specialization, which underlies the 
``latitude-niche breadth hypothesis''.


% In addition to species richness, there are general and robust latitudinal gradients in metabolic rates \citep{Stegen2012} and 
% rates and intensities of species interactions \citep{Marquis2005,Schemske2009}. These gradients, together with the strength and 
% generality of associations between biological processes and environmental variables, suggest that variation in network structure 
% may also be explained by ecological gradients \citep{Baiser2012}. Importantly, some of these gradients may act directly 
% on food-web properties such that they change over latitude beyond the expected changes due to changing species richness. For 
% example, if niche sizes vary with latitude then mean numbers of predators, prey, and links per species may also vary with 
% latitude. Some of this variation may be explained by variation in species richness, but as niche sizes are believed to be 
% related directly to latitude it is reasonable to expect that there may also be latitudinal effects unrelated to those of 
% changing species richness.



\section*{Methods}

\subsection*{Data Set} 

We compiled a list of 263 empirical food webs from
multiple sources (see Supplemental Information for web origins and selection
criteria). We recorded study site latitude from the original source where
possible or, where study sites were described but exact coordinates were not
given, obtained estimated coordinates using Google Earth \citep{GoogleEarth}.
Webs were also classified according to ecosystem type (terrestrial, lake,
stream, estuary, or marine), year of publication (to control for potential change
in standards of data collection over time \citep{Dunne2006}), and whether or not
humans were included.


% [[list in supplemental (347 GlobalWeb webs \citep{GlobalWeb}), supplemented 
% with 31 food webs used to study the relationship between consumer and resource body sizes (SizeWebs \citep{Brose2006})
% and seven food webs used to study the impact of parasites on network structure (ParWebs \citep{Dunne2013}).
% Of the GlobalWeb webs, 122 were rejected because their original source could not be found or was unpublished (53), they were source or sink
% webs rather than descriptions of an entire community (13), were focused on plant-pollinator, host-parasitoid, or 
% competitive interactions (8, 34, and 2, respectively), described inferred interactions in an extinct community (2), were ``generalized 
% schemes'' rather than being based on empirical observation (8), or because it was not clear which of a variety of 
% described sites the published food web represented (2). (See Table S1 for a list of rejected webs and reasons for their
% exclusion.) For three GlobalWeb webs which included parasites as a minor component of the community, as well as the 
% seven ParWebs webs, we included a modified version of each web which included only free-living species and the 
% interactions between them \citep{Dunne2013}. 


We then converted each web to a trophic-species web by aggregating species with identical predator and prey sets.
Using trophic species webs helps to reduce variation in resolution across different studies and across taxa within
a study \citep{Martinez1991,Vermaat2009,Dunne2004,Dunne2013}. The number of species ($S$) and links ($L$) in each 
trophic species
web were counted and used to calculate mean links per species ($Z$), generality ($G$), and vulnerability ($V$).

\subsection*{Relationships with Latitude}

We first examined simple linear relationships between latitude and $S$, $Z$, $G$, and $V$. We fit models of the form

\begin{eqnarray}
S_{i} = \alpha_{0} + \alpha_{1} L_{i} + \epsilon_{i} 
\end{eqnarray}

where $S_{i}$ is the species richness of web $i$, $\Lambda_{i}$ its absolute latitude (degrees north or south 
regardless of direction), and $\epsilon_{i}$ a residual error term.
These simple linear models, fit using the function lm in R \citep{R} package stats, were used to confirm that there are indeed latitudinal gradients in food-web structure in this data set.


\subsection*{Scaling Relationships with S}

We were then able to examine the form of the scaling relationship between each property ($Z$, $G$, and $V$) and $S$.
The scaling relationship between $Z$ and $S$ has been shown to be a power law \citep{Riede2010} of the form 

\begin{equation}
\label{Power}
Z_{i}=\alpha S_{i}^{\beta} + \epsilon_{i} 
\end{equation}

which is often re-expreseed in logarithmic form 

\begin{equation}
\label{Loglog}
log(Z_{i}) = log(\alpha) + \beta log(S_{i}) + \epsilon_{i}  
\end{equation}

Although scaling relationships between $S$, $G$, and $V$ have not been explicitly examined (but see scaling 
relationships for the standard devaitions of each property in \citet{Riede2010}), we expect that they will follow
power laws similar to that of the relationship between $S$ and $Z$. This is because the links taken into account in
calculating $G$ and $V$ are subsets of the total links included when calculating $Z$. As with $Z$, we explicitly 
compared the error distributions of models for $G$ and $V$ using both the power-law and logarithmic formulations. 



%%% TO SI

% Note that the two forms of the power-law relationship imply different error structures: normally-distributed, additive error in 
% \ref{Power} and lognormal, multiplicative error in \ref{Loglog}. As there is no intuitive reason to believe that our
% data have one error distribution over another, we follow the recommendations in \citet{Xiao2011} and compared the two
% model formulations explicitly. The model with the error distribution most resembling that observed in the empirical
% data was then used to test for potential effects of latitude.




\subsection*{Effect of Latitude on Scaling}

% Logarithm of a sum is awful:   log(a+c) = log(a) + log(1+b^[log(c)-log(a)]) But can probably feed that into R...


We then assessed
the impact of latitude on this relationship; specifically, the effect of latitude on the scaling exponent $\beta$. 
In addition to latitude, we included a factorial variable for ecosystem type (stream, lake, terrestrial, marine, or estuary) to account for recognized variation in food web structure across ecosystem types \citep{}.
% We also considered other trends in the data that could affect relationships between latitude and species richness.
% To account for changing standards in data collection over time \citep{Dunne2006} we included an effect of year of
% publication. As 11 of our XX webs included humans, and therefore might have different properties than other networks,
% we included a factorial effect of the presence of humans. 
Finally, some of our webs describe the same area in different
seasons or different years. To account for likely correlation among these webs, we included a random effect of site ID.


We therefore began by considering models of the form:

\begin{equation}
\label{PowerLat}
Z_{i}=\alpha S_{i}^{\beta_{0}+\beta_{1}L_{i}+\beta_{3}E_{i}} + P_{i} + \epsilon_{i} 
\end{equation}

where $L_{i}$ is the absolute latitude of food web $i$ (degrees from the equator, regardless of direction),
$E_{i}$ indicates the ecosystem type of food web $i$, and $P_{i}$ is a random effect of 
site ID.


The logarithmic formulation of this model is:

\begin{equation}
\label{LogLat}
log(Z_{i})= log(\alpha)+\beta_{0}log(S_{i}) + \beta_{1}L*log(S_{i}) +\beta_{3}E*log(S_{i})+ P_{i} +\epsilon_{i}
\end{equation}


We fit equation \ref{LogLat}, as well as similar equations for $G$ and $V$ using R \citep{R} function lmer from package lme4
\citep{lmerTest}.
We calculated the AIC of the chosen maximal model, as well as the AIC's of a suite of candidate simplified models.
Simplified models were obtained by systematically removing all possible combinations of terms from the full model.
The term for species richness was not removed from a model if any of the terms affecting the exponent $\beta$ were 
included in the model. The model with the lowest AIC was selected as the best-fitting model. 
This simplification was performed using the
R \citep{R} function dredge from package MuMIn \citep{MuMIn}. We then used the R \citep{R} function lmer
from the package lmerTest \citep{lmerTest} to estimate the standardized effects ($\beta$s ) for each fixed effect in the 
best-fitting models, as well as their corresponding $p$-values. 



% of  the University of Canberra's GlobalWeb database (www.globalwebdb.com) 
% \citep{}. Of these, 302 webs had accessible original sources. 
% 18 source and sink webs, which describe links to or from a focal taxon, were excluded as they are likely 
% to have different food web properties than webs attempting to
% describe the entire community of a site \citep{Williams2002}, as were 8 plant-pollinator, 34 host-parasitoid, and 2 
% competition-focused networks
% \citep{Riede2010}. Of the remaining 230 food webs, we excluded two paleowebs (comprised of probable interactions 
% between extinct or prehistoric species) \citep{Simenstad1978} and five ``generalized schemes'' 
% \citep{Nybakken1982,Percival1929,Swan1961,Landry1977,Petipa1979,Harrison1963}, as 
% we wished to measure
% directly observed rather than inferred interactions. We further excluded a food web aggregated from several sites in 
% the Pacific Ocean which were widely spread in both latitude and longitude and at markedly different successional stages
% \citep{Vinogradov1978}, such that it is not clear which site the food web represents. 
% This left 223 food webs with widely varying levels of taxonomic resolution, 11 of which contained humans. 




\section*{Results}

\subsection*{Relationships with Latitude}

$S$ and $G$ both increased with increasing latitude, contrary to expected latitudinal gradients (Table \ref{Latlms}). 
$G$ and $V$ were not significantly correlated with latitude. 

\begin{table}[!h]
\caption{Intercepts and slopes for simple linear regressions of food web properties against absolute latitude.}
\label{Latlms}
\begin{tabular}{l | l l | l l}

\multirow{2}{*}{Property} & \multicolumn{2}{|c}{Intercept} & \multicolumn{2}{|c}{Slope} \\
                      & $\alpha_{0}$ & $p$-value & $\alpha_{1}$ & $p$-value \\
\hline
$S$ & 23.10 & \textless0.001 & 0.284 & 0.040 \\
$Z$ & 2.566 & \textless0.001 & 0.020 & 0.180 \\
$G$ & 3.269 & \textless0.001 & 0.049 & 0.014 \\
$V$ & 2.566 & \textless0.001 & 0.020 & 0.180 \\
\end{tabular}
\end{table}


\subsection*{Scaling Relationships with S}

% For models of $Z$, $G$, and $V$, the assumption of multiplicative log-normal error was better supported (Table \ref{Errorfits}). We therefore use only models with logarithmic form for the remainder of the paper.

% \begin{table}[!t]

% \caption{AIC's for model \ref{Power}, assuming normal error, and model \ref{Loglog}, assuming lognormal error.}
% \label{Errorfits}
% \begin{tabular}{l | l l | c}
% Property & Normal error & Lognormal error & Model selected \\
% \hline
% $Z$ & 1151 &  767 & \ref{Loglog} \\
% $G$ & 1266 & 1037 & \ref{Loglog} \\
% $V$ & 1151 & 767 & \ref{Loglog} \\
% \end{tabular}
% \end{table}


In this data set, none of the best fit models for species richness, connectance, or links per species included an 
effect of latitude on either the scaling exponent or residual (Table \ref{Bestfits}). 
$G$ increased slightly in more recent webs ($\rho_{2}$=0.005, $p$\textless0.001), but the best-fit models for $V$ and $Z$ did not include a term for year of publication. 
In contrast, $Z$ and $V$ were higher in lake food webs ($\beta_{4}$=0.10, $p$\textless0.001 and $\beta_{4}$=0.10, $p$\textless0.001, respectively) while the best-fit model for $G$ did not include an effect for any ecosystem type.

\begin{table}[!h]

\caption{Standard effects and their $p$-values for the best-fitting models of $Z$, $G$, and $V$.}
\label{Bestfits}
\begin{tabular}{l | l l | l l | l l | l l }
\multirow{2}{*}{Property} & \multicolumn{2}{c}{Intercept} & \multicolumn{2}{c}{log(Species)} & \multicolumn{2}{c}{log(Species)$\Lambda$} & \multicolumn{2}{c}{$\Upsilon$} \\
& $\beta$ & $p$-value &  $\beta$ & $p$-value &  $\beta$ & $p$-value &  $\beta$ & $p$-value  \\
\hline
$Z$ & -0.516 & \textless0.001 & 0.662 & \textless0.001 & 0.099 & \textless0.001 & \multicolumn{2}{c}{NA} \\
$G$ & -9.999 & \textless0.001 & 0.547 & \textless0.001 & \multicolumn{2}{c|}{NA} & 0.005 & \textless0.001 \\
$V$ & -0.516 & \textless0.001 & 0.662 & \textless0.001 & 0.099 & \textless0.001 & \multicolumn{2}{c}{NA} \\
\end{tabular}
\end{table}


\section*{Discussion}

The tendency of food-web structure to exhibit scaling relationships with both
species richness and connectance has been  well-established \citep{}. As
species richness in particular is also known to vary systematically over
latitude \citep{}, intuitively one might suspect that any relationship
between food-web properties such as generality might be due to the latitudinal
gradient in species richness. Here, we found that while there were latitudinal
gradients in species richness and generality, latitude was not related to
either the scaling relationship between generality and species richness or the
residuals of this scaling relationship.


The latitudinal gradient in species richness in this data set ran in the opposite 
direction to the general trends for (list types of taxa). This may be due to the 
fact that latitudinal gradient in species richness are usually studied within a
taxonomic or trophic group (e.g., birds, vascular plants), whereas food webs 
measure many taxa concurrently. It is possible that while many taxa generally increase
towards the tropics, the number of species across all taxa that co-occur within an 
study site decreases. Alternatively, it is possible that by using trophic-species versions
of our food webs, we have artificially reversed the species-richness gradient. [[re-run analyses
with species-species webs]]. This would imply that, while there may be more species in the tropics,
many of those species tend to have identical predator and prey sets to co-occuring species. In either
case, it does not seem likely that the increase in
generality with latitude that we observed is due to an underlying tendency for
species to pursue a wider variety of prey at higher latitudes. Rather, the
trend in generality can be more parsimoniously explained by the trend in
species richness.


Our findings here add further doubt to the niche-breadth hypothesis, as
narrower niches in the tropics were associated with lower species richness in
this data set. It appears instead that, as more species are available, each
species tends to consume more prey, be consumed by more predators, and have
more links in general (check with figs). This makes intuitive sense as a more
species-rich community is likely to have more species with similar behaviour
and morphology that thereby have overlapping predator or prey sets.


Parallel increases in generality and species richness across latitude have
also been noted in studies of plant-pollinator networks (although in that case
species richness was highest in the tropics) \citep{Schleuning2012}. Although
the authors of that study do not explicitly examine the relationship between
specialization and network size, they do cite increasing plant species
richness as a potential driver for increasing generalization
\citep{Schleuning2012}. Although the scaling of mutualist network structure
does not appear to have been explored as systematically as that of unipartite
food webs, some poperties such as nestedness \citep{Vazquez2004} and XX
\citep{} have been shown to scale with species richness. As such scaling
appears to be common to both antagonistic and mutualistic food webs, and
trends in species richness can explain counterintuitive trends in network
structure \citep{Schleuning2012}, it seems likely that many latitudinal trends
in network structure are at least partly due to trends in species richness.


%Nestedness increases with A+P, increases number of extreme specialists AND number of generalists. Not clear how degree should scale with S according to scaling with nestedness.


\section*{Conclusion}

It was not our initial intention to question the validity of the wide range of
studies exploring variation in network structure across changing environments \citep{}.
However, if those studies have not taken into account changes in species
richness over environmental gradients, it is possible that observed changes in
food-web structure are due only indirectly to the environmental variables of
interest. Effects of environmental variables on network structure should
therefore be interpreted with caution unless effects of changing species
richness are made explicit.


%\end{spacing}
\newpage
\bibliographystyle{jae}%Compile with jae.bst style file
\bibliography{noISN.bib}% your .bib file(s)

\newpage

\section*{Tables}

\begin{center}
\begin{table}[!h]
\caption{Symbols for and descriptions of food-web properties.}
\label{symboltable}
\begin{tabular}{c l l}
\hline
Symbol      & Name & Description \\
\hline
S & Species richness & Number of trophic species \\
LS & Links per species & Mean number of links per taxon \\
C & Connectance & Number of observed links divided by the number of possible links ($L/S^{2}$) \\
\%Top & \% Top predators & Percentage of taxa without consumers \\
\%Int & \% Intermediate consumers & Percentage of taxa with both consumers and resources \\
\%Basal & \% Basal Resources & Percentage of taxa without resources \\
\%Herb	& \% Herbivores	& Percentage of herbivores and detrivores (taxa that feed on basal taxa) \\
\%Omni & \%Omnivores & Percentage of omnivores (taxa that feed on ≥2 taxa with different trophic levels) \\
Gen & Generality & Mean number of resources per taxon \\
GenSD & SD of generality & Standard deviation of resources per taxon \\
Vul & Vulnerability & Mean number of consumers per taxon \\
VulSD & SD of vulnerability & Standard deviation of consumers per taxon \\
LinkSD & SD of links per species & Standard deviation of links (consumers and resources) per taxon \\
SWTL & Mean short-weighted trophic level & Short-weighted trophic level averaged across taxa \\
Path & Path length & Characteristic path length, the mean shortest food chain between species pairs \\
\hline
\end{tabular}
\end{table}
\end{center}
% NODF & Average Nestedness & Degree to which interactions among specialists are subsets or interactions among generalists \\
% Chain & Chain length & Mean food chain length \\
% Cluster & Clustering coefficient & Mean clustering coefficient (probability that two taxa linked to the same taxon are also linked) \\

\begin{center}
\begin{table}[!h]
\caption{Full models for S, LS, and C.}
\label{symbols}
\begin{tabular}{c c c c c c c}
\hline
 & \multicolumn{2}{c}{S} & \multicolumn{2}{c}{C} & \multicolumn{2}{c}{LS} \\
\hline
Effect & Estimate & $p$-value & Estimate & $p$-value & Estimate & $p$-value \\
\hline
Latitude 			& 0.002  & 0.873  & -0.006 & 0.480  & -0.004 & 0.613 \\ 
Estuary 			& 1.078  & 0.022  & -0.809 & 0.012  & 0.269 & 0.362 \\
Freshwater 			& 1.347  & \textless0.001  & -1.229 & \textless0.001 & 0.118 & 0.394 \\
Marine 				& 1.127  & \textless0.001 & -0.761 & \textless0.001 & 0.366 & 0.012 \\
Terrestrial 		& 1.097  & \textless0.001 & -0.919 & \textless0.001 & 0.178 & 0.354\\
Latitude:Freshwater & 0.002  & 0.902  & 0.010  & 0.299  & 0.012 & 0.194\\
Latitude:Marine 	& 0.003  & 0.847  & 0.001  & 0.902  & 0.004 & 0.662 \\
Latitude:Terrestrial & 0.001 & 0.923  & 0.003  & 0.724  & 0.005 & 0.597 \\
\end{tabular}
\end{table}
\end{center}

\newpage

\section*{Figures}

\begin{figure}[!h]
\label{gen_all}
\includegraphics[width=.9\textwidth]{Figures/gen_all.eps}
\caption{Relationships between mean generality and latitude for food webs with moderate connectance (0.1, mean of empirical webs) at high (S=109, solid line), moderate (S=36, dashed line), and low (S=5, dotted line) species richness.}
\end{figure}


\begin{figure}[!h]
\label{TL_all}
\includegraphics[width=.9\textwidth]{Figures/TL_all.eps}
\caption{Relationships between mean trophic level and latitude for food webs with moderate connectance (0.1, mean of empirical webs) at high (S=109, solid line), moderate (S=36, dashed line), and low (S=5, dotted line) species richness.}
\end{figure}


\end{document}


