\documentclass[12pt]{article}  
\usepackage{amsmath}
\usepackage{gensymb}
\usepackage{url}
\usepackage[dvips]{graphicx}
\usepackage{multirow}
\usepackage{geometry}
\usepackage{pdflscape}
%\usepackage{rotating}
% make Figure 1 etc bold
\usepackage[labelfont=bf]{caption}
\usepackage{setspace}

\usepackage[running]{lineno}

% let's get some nature formatted citations
%\usepackage{overcite}
\usepackage[round]{natbib}

% some cheats to reduce the need to type complicated bits and pieces
\newcommand{\expect}[1]{\left\langle #1 \right\rangle}
\newcommand{\etal}{\textit{et al.\ }}

\newcommand{\beginsupplement}{%
        \setcounter{table}{0}
        \renewcommand{\thetable}{S\arabic{table}}%
        \setcounter{figure}{0}
        \renewcommand{\thefigure}{S\arabic{figure}}%
     }

% the abstract formatting
\newenvironment{sciabstract}{%
\begin{quote} \bf}
{\end{quote}}
\renewcommand\refname{References}

% margin sizes`
\topmargin 0.0cm
\oddsidemargin 0.2cm

\textwidth 16cm 
\textheight 21cm
\footskip 1.0cm
\begin{document}


\title{Generality in food webs scales with species richness not not with latitude}
\author{Alyssa R. Cirtwill$^{1,2}$, Daniel B. Stouffer$^{1}$, Tamara N. Romanuk$^{2}$}
\date{$^1$School of Biological Sciences\\University of Canterbury\\
Private Bag 4800\\Christchurch 8140, New Zealand \\
\medskip$^2$Department of Biology\\
Life Science Centre, Dalhousie University\\1355 Oxford St., P0 BOX 15000\\
Halifax NS, B3H 4R2\\ Canada\\}

\maketitle
\baselineskip=8.5mm
 
\vspace{0.3 in}

\begin{spacing}{1.0}
\section*{Abstract}

Several properties of food webs---the networks of feeding links between
species---vary systematically with the species richness of the underlying
community.  Under the ``latitude-niche breadth hypothesis'', which predicts
that species in the tropics will tend to evolve narrower niches, one might
expect that these scaling relationships could also be affected by latitude. To
test this hypothesis, we analysed the scaling relationships between species
richness and average generality, vulnerability, and links per species across a
set of 163 empirical food webs.  We also investigated scaling relationships
between the three food-web properties and the proportions of the web made up 
by basal resources, intermediate consumers, and top predators. While we
observed no effect of latitude on scaling relationships in the estuarine,
marine, and terrestrial food webs, there were strong effects of latitude on
scaling relationships in the freshwater food webs. In these communities,
the latitude-niche breadth hypothesis appears to hold true while in other habitat
types niches appear to be broader in the tropics. These contrasting findings
indicate that it is important to account for habitat type when exploring gradients
in food-web structure.
\end{spacing}


\section*{Introduction} 

  %Food webs are a thing. (Mention gen, vul, LS.) %Food webs have strong scaling structure.
  Food webs --networks of feeding links between species-- have been used for several decades to summarize the structure of 
  ecological communities~\citep{Petchey2008,Williams2000,Paine1966} and to understand how that structure relates to environmental variables 
  such as habitat type~\citep{Shurin2006,Briand1983}, primary productivity~\citep{Vermaat2009,Thompson2005a,Townsend1998}, and climate~\citep{Baiser2012,Petchey2010}. The latter
  variables in turn have strong gradients over latitude, with productivity and temperature both being higher in the tropics while 
  climate is more variable at high latitudes~\citep{Field1998}. These variables affect both the resources available and species' 
  metabolisms~\citep{Hechinger2011,White2011,OConnor2009,White2007}, and have been proposed as determinants of the strength of 
  interspecific interactions~\citep{Lang2012,Schleuning2012,Schemske2009}. 
  By modulating interactions between species, latitudinal gradients may also shape food-web structure.
  Indeed, these latitudinal environmental gradients have been put forward as potential drivers for the 
  latitudinal gradient in species richness, one of the most general and robust patterns in ecology~\citep{Schemske2009a,Macpherson2002,Kaufman1995}.

  % These gradients, together with the strength and generality of associations between biological processes and environmental variables, suggest that variation in network structure 
  % may also be explained by ecological gradients \citep{Baiser2012}. 


  One proposed link between species richness and latitude is the ``latitude-niche breadth hypothesis''~\citep{Vazquez2004}. 
  This hypothesis predicts that decreased seasonality in the tropics should lead to more stable populations, which in
  turn should evolve smaller niches~\citep{Vazquez2004}. These narrow niches should therefore allow more species to 
  coexist in the tropics than at higher latitudes. Alternatively, the higher productivity of the tropics~\citep{Brown2004}
  may result in a broader niche space~\citep{Davies2007} which could also sustain greater biodiversity even if niche 
  sizes are globally similar. Although the assumptions of the latitude-niche breadth hypothesis are only equivocally 
  supported~\citep{Vazquez2004}, it remains a compelling potential mechanism for the latitudinal gradient in species 
  richness~\citep{Lappalainen2006,Krasnov2008,Slove2010}. 


  If the latitude-niche breadth hypothesis is correct, there should also be direct relationships between latitude and the
  degree of specialization (i.e., niche breadth) of species within food webs.
  Attempts to unravel these effects, however,
  are complicated by known relationships between species richness and many other network properties~\citep{Riede2010}. 
  For example,
  narrower niches imply fewer links per species (i.e., greater specialization in the tropics~\citep{Dyer2007,Marra1997} but see~\citep{Schleuning2012}). However,
  average numbers of links per species tend to increase in larger food webs~\citep{Dunne2006,Riede2010}. This means
  that a latitudinal effect on specialization may be obscured by a latitudinal gradient in species richness. If this is
  the case, it may still be possible to uncover effects of latitude on specialization by examining the shape of the scaling 
  relationship between specialization and species richness over changing latitude. Here, we use three measures of specialization:
  mean links per species, mean generality (number of prey), and mean vulnerability (number of predators). By testing whether
  latitude affects the scaling of each property with species richness, we test effects of latitude on specialization implied by
  the ``latitude-niche breadth hypothesis''.



\section*{Methods}

\subsection*{Data Set} 

  We compiled a list of 163 empirical food webs from
  multiple sources (see Supplemental Information for web origins and selection
  criteria). We recorded study site latitude from the original source where
  possible or, where study sites were described but exact coordinates were not
  given, obtained estimated coordinates using Google Earth~\citep{GoogleEarth}.
  We then divided the species in each web into basal resources (those species with
  consumers but no prey), top predators (those species with prey but no predators),
  and intermediate consumers (species with both predators and prey, including cannibalistic
  species). 

  % [[list in supplemental (347 GlobalWeb webs \citep{GlobalWeb}), supplemented 
  % with 31 food webs used to study the relationship between consumer and resource body sizes (SizeWebs \citep{Brose2006})
  % and seven food webs used to study the impact of parasites on network structure (ParWebs \citep{Dunne2013}).
  % Of the GlobalWeb webs, 122 were rejected because their original source could not be found or was unpublished (53), they were source or sink
  % webs rather than descriptions of an entire community (13), were focused on plant-pollinator, host-parasitoid, or 
  % competitive interactions (8, 34, and 2, respectively), described inferred interactions in an extinct community (2), were ``generalized 
  % schemes'' rather than being based on empirical observation (8), or because it was not clear which of a variety of 
  % described sites the published food web represented (2). (See Table S1 for a list of rejected webs and reasons for their
  % exclusion.) For three GlobalWeb webs which included parasites as a minor component of the community, as well as the 
  % seven ParWebs webs, we included a modified version of each web which included only free-living species and the 
  % interactions between them \citep{Dunne2013}. 
  % As we were considering the log of proportions of basal resources, top predators, and intermediate consumers, each web used in our analysis 
  % had to have at least one species in each trophic group.

  % of  the University of Canberra's GlobalWeb database (www.globalwebdb.com) 
  % \citep{}. Of these, 302 webs had accessible original sources. 
  % 18 source and sink webs, which describe links to or from a focal taxon, were excluded as they are likely 
  % to have different food web properties than webs attempting to
  % describe the entire community of a site \citep{Williams2002}, as were 8 plant-pollinator, 34 host-parasitoid, and 2 
  % competition-focused networks
  % \citep{Riede2010}. Of the remaining 230 food webs, we excluded two paleowebs (comprised of probable interactions 
  % between extinct or prehistoric species) \citep{Simenstad1978} and five ``generalized schemes'' 
  % \citep{Nybakken1982,Percival1929,Swan1961,Landry1977,Petipa1979,Harrison1963}, as 
  % we wished to measure
  % directly observed rather than inferred interactions. We further excluded a food web aggregated from several sites in 
  % the Pacific Ocean which were widely spread in both latitude and longitude and at markedly different successional stages
  % \citep{Vinogradov1978}, such that it is not clear which site the food web represents. 
  % This left 223 food webs with widely varying levels of taxonomic resolution, 11 of which contained humans. 


  As the food webs in this dataset are derived from a variety of sources and were compiled over many decades, it
  is likely that they vary in their resolution and in the amount of sampling effort invested in their assembly.
  Many analyses of food-web structure attempt to reduce this variation by aggregating species with identical predator and prey
  sets to form ``trophic species'' webs (e.g.~\citep{Martinez1991,Vermaat2009,Dunne2004,Dunne2013}). As this study
  is concerned directly with the number of species at a particular latitude, however, we did not wish to ignore 
  species with redundant sets of interactions. We therefore analysed both original and trophic-species versions
  of the dataset; in each case using the number of species ($S$) and 
  links ($L$) in each web to calculate the mean number of links per species ($Z$), mean generality 
  ($G$), and mean vulnerability ($V$) of the web. 
  The version of the dataset used did not qualitatively change the results, suggesting that
  the scaling relationships between species richness, other food-web properties, and latitude are very 
  similar whether or not redundant species are included. For simplicity, we present only
  the results for the original webs.


\subsection*{Relationships with Latitude}

  To determine whether there were latitudinal gradients in food-web structure,
  we first examined simple linear relationships between latitude and each of 
  species richness, links per species, generality, vulnerability, and proportions
  of basal resources, intermediate consumers, and top predators. We fit models of the form

  \begin{equation}
  \label{Latfull}
  S_{i} = \alpha_{0} + \alpha_{1} L_{i} + \alpha_{2} E_{i} + \alpha_{3} L_{i} E_{i} + \epsilon_{i} ,
  \end{equation}

  \noindent where $S_{i}$ is the species richness of web $i$, $L_{i}$ its absolute
  latitude (degrees north or south  regardless of direction), $E_{i}$ is a categorical
  variable indicating the ecosystem type of network $i$ (comprising terms for stream, 
  marine, lake, and terrestrial networks with estuarine
  networks providing the intercept) and $\epsilon_{i}$ a residual error term. 
  We next calculated the AIC
  of the maximal model as well as the AIC's of a suite of candidate simplified models identified
  using the R~\citep{R} function dredge from package MuMIn~\citep{MuMIn}. 
  Simplified models were obtained by
  systematically removing all possible combinations of terms from the full model.
  The best-fitting model was then determined to be the model with the fewest terms 
  where $\Delta$AIC\textless2. If several models shared the fewest number of terms 
  and had $\Delta$AIC\textless2, the model with the lowest AIC in that set was chosen as the best-fit
  model.


\subsection*{Scaling Relationships with S}

  Next, we examined the form of the scaling relationship between each 
  property (links per species, generality, and vulnerability) and 
  species richness. The scaling relationship between links per species ($Z$) and 
  species richness ($S$) has been shown to be a power law~\citep{Riede2010} of the form 

  \begin{equation}
  \label{Power}
  Z_{i} \sim \alpha S_{i}^{\beta}  ,
  \end{equation}

  \noindent which is often re-expressed in logarithmic form 

  \begin{equation}
  \label{Loglog}
  \log{Z_{i}} \sim \log{\alpha} + \beta\log{S_{i}}  .
  \end{equation}


  \noindent Although these relationships are very similar, they imply different error distributions~\citep{Xiao2011}.
  Specifically, equation~(\ref{Power}) implies a normally-distributed, additive error and equation~(\ref{Loglog}) a lognormal,
  multiplicative error. As we have no \emph{a priori} reason to believe that our dataset has one error distribution
  over another, we follow the recommendations in~\citet{Xiao2011} and compared the two
  model formulations explicitly. The model with the error distribution most resembling that observed in the empirical
  data was then used to test for potential effects of latitude.


  Although scaling relationships between species richness and generality or
  species richness and vulnerability have not been explicitly examined (but see scaling 
  relationships for the standard devaitions of each property in~\citet{Riede2010}), we expect that they will follow
  power laws similar to that of the relationship between species richness and links per species.
  This is because the links taken into account in calculating generality and vulnerability are subsets 
  of the total links included when calculating links per species. As with links per species, we explicitly 
  compared the error distributions of models for generality and vulnerability using
  both the power-law and logarithmic formulations. 
  In each case, we used the best-fitting equation as a template when assessing the effect of latitude on scaling with
  species richness.


\subsection*{Effect of Latitude on Scaling}

  % Logarithm of a sum is awful:   log(a+c) = log(a) + log(1+b^[log(c)-log(a)]) But can probably feed that into R...


  We then assessed the impact of latitude on the scaling relationships between species richness and 
  link density, generality, vulnerability. In the context of the scaling relationships above, note 
  that this implies that we aim to determine the effect of latitude on
  the scaling exponent $\beta$. As when examining the relationships between latitude and each food 
  web property directly, we included a categorical variable for ecosystem type (stream, lake, 
  terrestrial, marine, or estuary), as well as interactions between food web type and latitude.


  We therefore began by considering models of the form

  \begin{equation}
  \label{PowerLat}
  Z_{i}=\alpha S_{i}^{\beta_{0}+\beta_{1}L_{i}+\beta_{2}E_{i}+\beta_{3}LE_{i}} + \epsilon_{i} ,
  \end{equation}

  \noindent where $S_{i}$, $L_{i}$, and $E_{i}$ are as defined previously. The logarithmic formulation of this model is

  \begin{equation}
  \label{LogLat}
  \log{Z_{i}} = \log{\alpha}+\beta_{0}\log{S_{i}} + \beta_{1}L\log{S_{i}} +\beta_{2}E\log{S_{i}} +\beta_{3}LE\log{S_{i}} +\epsilon_{i} .
  \end{equation}

  For each property in link density, generality, and vulnerability, we used the form of the equation that was best supported
  when describing the scaling of the property with species richness alone~\citep{Xiao2011}.
  We then simplified each model following the same procedure as for the relationships between
  latitude and food-web properties, except that species richness was retained in all reduced models. 


\section*{Scaling by Trophic Levels}

  We were also interested in the ways that scaling relationships with species richness might
  be affected by changes to the distribution of species among trophic levels. To
  that end, we repeated all of the above analyses replacing species richness by
  proportion of basal resources, proportion of intermediate consumers,
  or proportion of top predators. All model fitting and model
  simplification procedures were identical to those described for species
  richness.



\section*{Results}

\subsection*{Relationships with Latitude}

  Contrary to the expected latitudinal gradient, the best-fit version of
  equation~(\ref{Latfull}) for species richness did not include a signficant effect of latitude for any ecosystem type except streams
  ($\beta_{Latitude}$=0.095, $p$=0.626; $\beta_{Latitude:Stream}$=-1.69, $p$=0.007).  This
  relationship was robust to the removal of three outliers (based on Cook's Distance). 
  There were also relationships between latitude and link density and latitude and vulnerability
  in stream food webs, but these trends were non-signficant after the removal of three outliers.
  Link density and vulnerability did not vary with latitude in any other ecosystem type.
  The best-fit version of equation~(\ref{Latfull}) for generality did not include any effect of
  latitude in any ecosystem type. 


  The relationships between latitude and the proportions of species at each trophic level were broadly similar
  to the relationship between latitude and species richness. 
  The proportion of basal resources increased towards the poles in stream webs ($\beta_{Latitude:Stream}$=0.012, $p$\textless0.001) 
  but did not vary with latitude in estuarine, marine, lake, or terrestrial 
  food webs ($\beta_{Latitude}$=0.001, $p$=0.139).
  The proportion of intermediate consumers did not vary with latitude in any ecosystem type.
  The proportion of top predators did not vary with latitude in 
  estuarine, marine, or terrestrial food webs ($\beta_{Latitude}$=0.001, $p$=0.136) but decreased towards the poles in stream and lake food 
  webs ($\beta_{Latitude:Stream}$=-0.006, $p$=0.030; 
  $\beta_{Latitude:Lake}$=-0.006, $p$=0.001). 
  All of these trends were robust to the removal of outliers.


\subsection*{Form of scaling relationships}

  When considering the relationships between species richness or the proportions 
  of basal resources, intermediate consumers, or top predators and all response variables 
  (link density, generality, vulnerability), equation~(\ref{Loglog}) had a
  lower AIC than did equation~(\ref{Power}). This indicates that the
  data support an assumption of multiplicative lognormal error better than an
  assumption of additive normal error. That is, models where $\epsilon$ is
  modelled as an additive term on
  the logarithmic scale provide a better description of the data than models
  where $\epsilon$ is modelled as an additive term on the arithmetic scale.  
  We therefore used logarithmic-form models when assessing the
  effect of latitude on scaling relationships  with species richness.



\subsection*{Effect of Latitude on Scaling with S, B, I, and T}

  Link density, generality, and vulnerability each increased with increasing
  species richness (Fig.~\ref{S}; $\beta_0$=0.666, $p$\textless0.001; $\beta_0$=0.623,
  $p$\textless0.001; and $\beta_0$=0.666, $p$\textless0.001 respectively).  For estuarine,
  marine, and terrestrial food webs this increase did not vary with latitude
  ($\beta_{Latitude}$=-0.001, $p$=0.267 for link density; $\beta_{Latitude}$=-0.001, $p$=0.363 for generality;
  and $\beta_{Latitude}$=-0.001, $p$=0.267 for vulnerability). In lake food webs, link density,
  generality, and vulnerability all increased more quickly towards the poles
  ($\beta_{Latitude:Lake}$=0.005, $p$=0.008; $\beta_{Latitude:Lake}$=0.005,
  $p$=0.002; and $\beta_{Latitude:Lake}$=0.005, $p$=0.008, respectively). In
  stream food webs generality increased more rapidly towards the poles 
  ($\beta_{Latitude}$=0.007, $p$=0.002) while link density and vulnerability did not vary with latitude.


  In general, link density, generality, and vulnerability decreased as the
  proportion of basal resources in a web increased (Fig.~\ref{B}). For
  estuarine, lake, and stream food webs these decreases were more gradual towards the
  poles ($\beta_{Latitude}$=-0.007, $p$=0.006 for link density;
  $\beta_{Latitude}$=-0.009, $p$=0.002 for generality; and
  $\beta_{Latitude}$=-0.007, $p$=0.006 for vulnerability).
  In marine food webs there was very little effect of latitude on
  the strength of scaling ($\beta_{Latitude:Marine}$=0.009, $p$=0.047; 
  $\beta_{Latitude:Marine}$=0.011, $p$=0.028; 
  and $\beta_{Latitude:Marine}$=0.009, $p$=0.047 respectively),
  while in terrestrial food webs link density, generality, and vulnerability
  decreased more gradually towards the 
  equator ($\beta_{Latitude:Terrestrial}$=0.013, $p$=0.002; 
  $\beta_{Latitude:Terrestrial}$=0.015, $p$=0.001; 
  and $\beta_{Latitude:Terrestrial}$=0.013, $p$=0.002).


  In contrast to the proportion of basal resources, link density, generality,
  and vulnerability tended to increase with increasing proportions of
  intermediate consumers (Fig.~\ref{I}; ). The best-fit models for scaling of link
  density and vulnerability with the proportion of intermediate consumers did
  not include any effect of latitude on scaling in any ecosystem type. Generality
  increased more rapidly towards the equator in estuarine, lake, marine, and
  stream food webs ($\beta_{Latitude}$=-0.006, $p$=0.009) and varied little with
  latitude in terrestrial food webs ($\beta_{Latitude:Terrestrial}$=0.008,
  $p$=0.053). In lake and stream food webs, generality increased more slowly
  with the proporiton of intermediate consumers such that, at low proportions,
  the effect of latitude could negate or reverse the increase.


  As with the proportion of basal resources; link density, generality, and
  vulnerability decreased as the proportion of top predators increased. For
  estuarine, lake, and marine food webs scaling of link density, generality, and
  vulnerability did not vary significantly with latitude
  ($\beta_{Latitude}$=-0.001, $p$=0.527 for link density;
  $\beta_{Latitude}$=-0.001, $p$=0.521 for vulnerability; the best-fit model for
  generality did not include any terms for latitude in any ecosystem type). In
  stream and terrestrial food webs, the decrease in link density and
  vulnerability was less sharp towards the poles, although this trend was
  significant only in streams ($\beta_{Latitude:Stream}$=0.015,
  $p$\textless0.001 and $\beta_{Latitude:Terrestrial}$=0.006, $p$=0.069 for link
  density; $\beta_{Latitude:Stream}$=0.015, $p$\textless0.001 and
  $\beta_{Latitude:Terrestrial}$=0.006, $p$=0.067 for vulnerability).



\section*{Discussion}

  The tendency of food-web structure to exhibit scaling relationships with
  species richness has been well-established~\citep{Dunne2004,Riede2010}. As
  species richness in particular is also known to vary systematically over
  latitude~\citep{Schemske2009a,Macpherson2002,Kaufman1995}, intuitively one might suspect that any relationship
  between food-web properties such as generality might be due to the latitudinal
  gradient in species richness. In this dataset, however, we did not find
  overall latitudinal gradients in species richness, links per species, 
  generality, vulnerability, or the proportions of food webs accounted for by 
  basal resources, intermediate consumers, and top predators. 


  The lack of a latitudinal gradient in species richness in this dataset contrasts
  strongly with other studies~\citep{Schemske2009a,Macpherson2002,Kaufman1995}. As numbers of species and links included in a food web
  vary strongly with sampling effort as well as with the underlying diversity of the study
  area, it is possible that the lack of latitudinal trends here is a result of researchers
  tending to expend similar amounts of sampling effort across studies. This could result in
  food webs describing species-rich tropical communities omitting more species and links
  than species-poor arctic communities. In addition, it is worth noting that gradients in
  species richness are generally measured for a single taxonomic group at a time (e.g.~\citep{Kaufman1995}).
  It is possible that these taxa are not well-represented in our food webs and that the
  dominant taxa in them do not have an underlying latitudinal gradient in richness. In
  either case, the lack of association between species richness
  and latitude in any ecosystem type nevertheless means that any effect of latitude on scaling relationships
  between species richness and other properties is not being driven by underlying variation in
  species richness, allowing us to more clearly assess effects of latitude on scaling with 
  species richness and proportions of species in different trophic levels.


  Scaling of links per species, generality, and vulnerability with species
  richness varied strongly across ecosystem types. In estuarine, marine, and
  terrestrial food webs scaling of each property varied little with latitude.
  This is consistent with the idea that species' niche breadths do not vary
  systematically with temperature and productivity but that the niche space
  might be larger in species-rich communities~\citep{Davies2007}. Rather than
  niche space depending on temperature and productivity, it may be that species
  diversity itself affects the niche space available to species (although climate
  may affect speciation rates and therefore the diversity in a region~\citep{Currie2004}).


  Unlike other ecosystem types, in lake and stream food webs 
  the scaling of generality was stronger in higher-latitude food webs. 
  In lake food webs, this trend was echoed in the scaling relationships
  between species richness and links per species and vulnerability.
  This means that species in tropical freshwater
  communities gain fewer additional feeding links per additional species in the
  web and that species in tropical lakes also gain fewer predators, and fewer links
  in general, per additional species. These trends are consistent with the hypothesis
  that greater stability in the tropics leads to narrower niches~\citep{Brown2004} and a higher proportion of
  specialists. This may be partly due to high-latitude species
  tending to switch between different seasonally-available
  prey~\citep{Isaac2012,Wilhelm1999} while tropical freshwater ecosystems may have more stable composition.


\section*{Conclusion}

  Our results were consistent with the latitude-niche breadth hypothesis in estuarine, marine,
  and terrestrial communities but instead consistent with the hypothesis of greater specialization
  in the tropics in stream and lake food webs. This suggests that different mechanisms may structure
  food webs in different habitat types and that freshwater food webs in particular may be strongly
  affected by seasonal variation. In addition, different relationships between latitude and niche
  breadth in different habitat types goes some way towards explaining the equivocal support for
  the opposing hypotheses of~\citet{Brown2004} and ~\citet{Davies2007}; both have merit but appear
  to apply to different systems.



%\end{spacing}
\newpage
\bibliographystyle{jae}%Compile with jae.bst style file
\bibliography{noISN}% your .bib file(s)

\newpage

% \section*{Tables}

% \begin{table}[!h]
% \caption{Intercepts and slopes for the best-fitting model describing the relationship between species-richness, mean links per species, mean generality, and mean vulnerability and latitude. Note that none
% of the best-fitting models included a term for latitude or an interaction between latitude and ecosystem type. Each `NA' indicates a term that was not included in the corresponding best-fit model.}
% \label{Latlms}
% \begin{tabular}{l | l l  l l  l l  l l}

% \hline
% \multirow{2}{*}{Property} & \multicolumn{2}{|c}{Intercept} & \multicolumn{2}{|c}{Stream} & \multicolumn{2}{|c}{Lake} & \multicolumn{2}{|c}{Terrestrial}\\
% & $\alpha_{0}$ & $p$-value & $\alpha_{2}$ & $p$-value & $\alpha_{2}$ & $p$-value & $\alpha_{2}$ &$p$-value\\
% \hline
% Species richness  & 34.6 & \textless0.001 & 22.4 & 0.167 \textless0.001 & NA & NA & NA & NA \\
% Links per species & 3.53 & \textless0.001 & NA & NA & NA & NA & NA & NA \\
% Generality        & 7.32 & \textless0.001 & NA & NA & -1.98 & 0.044 & -3.77 & \textless0.001 \\
% Vulnerability     & 3.53 & \textless0.001 & NA & NA & NA & NA & NA & NA \\
% \hline
% \end{tabular}
% \end{table}

% \begin{table}[!h]
% \caption{Intercepts and slopes for the best-fitting model describing the relationship between species-richness, mean links per species, mean generality, and mean vulnerability and latitude in trophic-species versions of food webs. Note that none
% of the best-fitting models included a term for latitude or an interaction between latitude and ecosystem type. `NA' a term that was not included in the corresponding best-fit model.}
% \label{Latlms}
% \begin{tabular}{l | l l  l l  l l  l l}

% \hline
% \multirow{2}{*}{Property} & \multicolumn{2}{|c}{Intercept} & \multicolumn{2}{|c}{Stream} & \multicolumn{2}{|c}{Lake} & \multicolumn{2}{|c}{Terrestrial}\\
% & $\alpha_{0}$ & $p$-value & $\alpha_{2}$ & $p$-value & $\alpha_{2}$ & $p$-value & $\alpha_{2}$ &$p$-value\\
% \hline
% Species richness  & 34.6 & \textless0.001 & 22.4 & 0.167 \textless0.001 & NA & NA & NA & NA \\
% Links per species & 3.53 & \textless0.001 & NA & NA & NA & NA & NA & NA \\
% Generality        & 7.32 & \textless0.001 & NA & NA & -1.98 & 0.044 & -3.77 & \textless0.001 \\
% Vulnerability     & 3.53 & \textless0.001 & NA & NA & NA & NA & NA & NA \\
% \hline
% \end{tabular}
% \end{table}


% \begin{table}[!h]
% \caption{Intercepts and slopes for best-fit models describing the effect of latitude on the scaling of mean links per species, mean generality, and mean vulnerability with species richness. Each `NA' indicates a term that was not included in the corresponding best-fit model. These models refer to the original versions of food webs. }
% \label{Bestfits}
% \begin{tabular}{l | l l  l l  l l }
% Effect          & \multicolumn{2}{c|}{Links per species} & \multicolumn{2}{c|}{Generality} 
%                 & \multicolumn{2}{c}{Vulnerability}\\
%                 & $\beta$ & $p$-value     &  $\beta$ & $p$-value    &  $\beta$ & $p$-value  \\
% \hline
% Intercept               &-0.453 & \textless0.001 & -0.249 & \textless0.001 & -0.453 & \textless0.001 \\
% log(species)            & 0.419 & \textless0.001 &  0.663 & \textless0.001 &  0.419 & \textless0.001 \\
% log(species):Lake       & 0.119 & \textless0.001 & -0.185 & 0.016          &  0.119 & \textless0.001 \\
% log(species):Marine     & 0.255 & \textless0.001 &  0.090 & 0.002          &  0.255 & \textless0.001 \\
% log(species):Terrestrial& 0.207 & 0.002          & -0.122 & \textless0.001 &  0.207 & 0.002 \\
% log(species):Latitude   & 0.004 & \textless0.001 & -0.001 & 0.199          &  0.003 & \textless0.001 \\
% log(species):Latitude:Marine&-0.004 & 0.007      &  NA    &  NA            & -0.005 & 0.007 \\
% log(species):Latitude:Terrestrial&-0.005 & 0.002 &  NA    &  NA            & -0.005 & 0.002 \\
% log(species):Latitude:Lake & NA & NA             &  0.006 & 0.002          &  NA    & NA    \\
% \hline
% $R^{2} (adjusted)$   & \multicolumn{2}{c|}{0.706} & \multicolumn{2}{c|}{0.658} & \multicolumn{2}{c}{0.706} \\
% \hline
% \end{tabular}
% \end{table}


% \begin{table}[!h]
% \caption{Intercepts and slopes for best-fit models describing the effect of latitude on the scaling of mean links per species, mean generality, and mean vulnerability with species richness. Each `NA' indicates a term that was not included in the corresponding best-fit model. These models refer to the trophic species versions of food webs. }
% \label{Bestfits}
% \begin{tabular}{l | l l  l l  l l }
% Effect          & \multicolumn{2}{c|}{Links per species} & \multicolumn{2}{c|}{Generality} 
%                 & \multicolumn{2}{c}{Vulnerability}\\
%                 & $\beta$ & $p$-value     &  $\beta$ & $p$-value    &  $\beta$ & $p$-value  \\
% \hline
% Intercept               & -0.561 & \textless0.001 & -0.299 & \textless0.001 & -0.561 & \textless0.001 \\
% log(species)            &  0.733 & \textless0.001 &  0.622 & \textless0.001 &  0.733 & \textless0.001 \\
% log(species):Lake       & -0.118 & 0.047          & -0.115 & 0.078          & -0.118 & 0.047 \\
% log(species):Marine     &  0.051 & 0.040          &  0.120 & \textless0.001 &  0.051 & 0.040 \\
% log(species):Stream     & -0.161 & 0.014          & -0.057 & 0.421          & -0.161 & 0.014 \\
% log(species):Latitude   & -0.001 & 0.081          & -0.001 & 0.273          & -0.001 & 0.081 \\
% log(species):Lat:Lake   &  0.006 & \textless0.001 &  0.006 & \textless0.001 &  0.006 & \textless0.001 \\
% log(species):Lat:Stream &  0.004 & 0.018          &  0.003 & 0.059          &  0.004 & 0.018 \\
% \hline
% $R^{2} (adjusted)$   & \multicolumn{2}{c|}{0.706} & \multicolumn{2}{c|}{0.658} & \multicolumn{2}{c}{0.706} \\
% \hline
% \end{tabular}
% \end{table}

\newpage

\section*{Figures}

\begin{figure}[h]
\includegraphics[width=.85\textwidth]{Figures/by_TL/scaling_with_S/proportions/S_latlines_nonts.eps}
\caption{Scaling of link density, generality, and vulnerability with species richness
varies across ecosystem types and over latitude. For each property we show 
prediction curves for food webs at the equator (darkest line) and at 
20\degree, 40\degree, 60\degree, and 80\degree (lightest line) from the 
equator. We did not include direction (i.e., North or South) in our analyses. 
All curves are based on N=163 empirical food webs.}
\label{S}
\end{figure}


\begin{figure}[!H]
\includegraphics[width=.9\textwidth]{Figures/by_TL/scaling_with_S/proportions/B_latlines_nonts.eps}
\caption{Scaling of link density, generality, and vulnerability 
with the proportion of basal resources in a food web
varies across ecosystem types and over latitude. For each property we show 
prediction curves for food webs at the equator (darkest line) and at 
20\degree, 40\degree, 60\degree, and 80\degree (lightest line) from the 
equator. We did not include direction (i.e., North or South) in our analyses. 
All curves are based on N=163 empirical food webs.}
\label{B}
\end{figure}

\newpage


\begin{figure}[h]
\includegraphics[width=.9\textwidth]{Figures/by_TL/scaling_with_S/proportions/I_latlines_nonts.eps}
\caption{Scaling of link density, generality, and vulnerability 
with the proportion of intermediate consumers in a food web
varies across ecosystem types and over latitude. For each property we show 
prediction curves for food webs at the equator (darkest line) and at 
20\degree, 40\degree, 60\degree, and 80\degree (lightest line) from the 
equator. We did not include direction (i.e., North or South) in our analyses. 
All curves are based on N=163 empirical food webs.}
\label{I}
\end{figure}


\begin{figure}[h]
\includegraphics[width=.9\textwidth]{Figures/by_TL/scaling_with_S/proportions/T_latlines_nonts.eps}
\caption{Scaling of link density, generality, and vulnerability 
with the proportion of top predators in a food web
varies across ecosystem types and over latitude. For each property we show 
prediction curves for food webs at the equator (darkest line) and at 
20\degree, 40\degree, 60\degree, and 80\degree (lightest line) from the 
equator. We did not include direction (i.e., North or South) in our analyses. 
All curves are based on N=163 empirical food webs.}
\label{T}
\end{figure}

\newpage

\begin{figure}[H]
\includegraphics[width=.9\textwidth]{Figures/by_TL/scaling_with_S/proportions/fitlines_nonts.eps}
\caption{Scaling relationships for link density, generality (number of prey), 
and vulnerability (number of predators) 
against the proportions of basal resources (\% Basal), intermediate consumers (\% Intermediate), top 
predators (\% Top), and species richness of a food web. 
For each relationship, we show observed values corrected for
ecosystem type and latitude (grey circles), as well as the overall scaling relationship (N=163 food webs). }
\label{props_v_lat}
\end{figure}


\end{document}


>>>>>>> 7787e1ad724c4c9aa740fb54fce5a4669c120daf
