\section*{Introduction}

Food webs (networks of feeding links between species) have been used for decades to summarize the structure of ecological communities \citep{Martinez2000,earlierwork}.
Comparisons of community structure around the world and across ecosystem types 
have revealed that the structure of biological networks is generally tightly
constrained, particularly by relationships between species richness or connectance
(the proportion of possible links that are realized) and other food-web properties \citep{Riede2010}.
These relationships have suggested that the structure of food webs is highly scale-depended which here refers to the
strong dependence of network properties, such as fractions of
species in different trophic groups, on the species richness or connectance of
the networks \citep{Riede2010}. Therefore it is likely that much variation in network structure
can be explained by variation in these two properties (Vermaat et al., 2009).


As changes in species richness and connectance can explain much of the variation in network structure \citep{Vermaat2009}, 
it follows that drivers of change in these two properties may also affect other properties. Species richness in particular 
exhibits a strong latitudinal gradient, believed to be driven by varying productivity and temperature at different latitudes 
\citep{Brown2004,Cardillo2005,Thompson2005,Davies2007}. An alternative view
posits that species' niches are narrower in the tropics, 
reducing competition and permitting greater adaptive radiation \citep{}. Although this view is only equivocally supported 
\citep{Vazquez2004}, it is also possible that the higher productivity of the tropics \citep{Brown2004} may result in a broader 
niche space \citep{Davies2007} which could also allow greater diversification even if niche sizes are globally similar.


In addition to species richness, there are general and robust latitudinal gradients in metabolic rates \citep{Stegen2012} and 
rates and intensities of species interactions \citep{Marquis2005,Schemske2009}. These gradients, together with the strength and 
generality of associations between biological processes and environmental variables, suggest that variation in network structure 
may also be explained by ecological gradients \citep{Baiser2011}. Importantly, some of these gradients may act directly 
on food-web properties such that they change over latitude beyond the expected changes due to changing species richness. For 
example, if niche sizes vary with latitude then generality (mean number of prey per predator), vulnerability (mean number of 
predators per prey), and mean number of links per species may all vary with latitude beyond variation due to changes in species
richness.


Here we examine the effect of latitude on food-web structure by examining
its impact on the scaling relationships of food-web properties with species
richness and connectance. We expect that food-web structure will scale strongly
with both species richness and connectance, and that the scaling relationships
of properties measuring degree of specialization (i.e., generality, vulnerability, and
links per species) will change over latitude. The scaling relationships of other
properties (for example the proportions of top, intermediate, and basal resources) may
be less affected by latitude.


\section*{Methods}

We compiled a list of XX published food webs from [database details]. We
included only food webs with a clearly defined study site that did not cover
more than 10 degrees of latitude. Source and sink webs were also excluded as
they are likely to have different food web properties than webs attempting to
describe the entire community of a site. We also excluded webs describing highly
modified or artificial ecosystems, and those including humans, as these could also
have different structural properties to more ``natural'' food webs.


We then converted each web to a trophic-species version by aggregating species
with the same predators and prey. This helps to reduce variation in resolution
across different studies. We then classified them both according to latitude
and ecosystem type (terrestrial, [[stream, lake (lump?)]], marine, estuary). We analysed XX
food-web properties for each web (Table 1).


In our analysis, we first addressed whether the diversity (S) or complexity (C, L/S) of our
published food webs varied over latitude (expressed as degrees away from the
equator regardless of direction) or factorial ecosystem type. We also included
year published as in independent variable to control for changing standards
for food web studies over time. We fit independent generalized linear models
for each of the logarithms of species richness, mean links per species, and
connectance. These models were used to determine any latitudinal gradients in the food-web
properties known to drive scaling relationships with other properties.


Second, we analysed the scaling relationships between the remaining XX food-
web parameters and connectance and species richness. We used a series of
generalized linear models with a food-web parameter as the dependent variable
and latitude, ecosystem type, and their interactions with connectance and
species richness as independent variables. (parameter ~
latitude*ecotype*(log10(connectance)+log10(species richness))). For food web
properties that were proportions, we used general linear models with a
binomial error distribution. All other food web properties were log-transformed with a normal error distribution.
When modelling the standard deviation of generality, this
required the removal of 5 webs, all with 12 or fewer species, which had
standard deviations of generality of 0. These full models were then
systematically reduced using R function dredge \cite{} in package MuMIn
\cite{} in order to isolate the model with the highest AIC. These best-fitting models were 
then examined in detail. Significant slopes over species-richness or connectance were taken to indicate scaling
relationships. Significant interactions with latitude indicated that these scaling relationships varied from the tropics
to the poles. Different intercepts indicated differences between ecosystem types.


\section*{Results}

In this data set, none of the best fit models for species richness, connectance, or links per species included an 
effect of latitude, although all three properties varied across ecosystem types. [[Table or something]]


The best-fit models for most food-web properties exhibited scaling relationships with species-richness and/or 
connectance. Notable exceptions were the proportions of basal resources, herbivores, and omnivores, which varied only 
across year published; and the proportions of top predators and intermediate consumers, which varied with connectance 
and species richness and connectance (respectively) although not significantly.


For the food-web properties that did scale with species-richness or connectance, most of these relationships were 
constant over latitude. Latitude terms were not included in the best-fit models for vulnerability, SD of generality or 
SD of links. SD of vulnerability showed a weak negative relationship with latitude ($\Beta$=-0.001, $p$=0.056), 
and path length increased with increasing 
latitude ($\Beta$ = 0.005, $p$=0.045) but neither latitude effect affected any scaling with species richness or 
connectance.


The scaling rules of mean and maximum trophic level and generality varied over latitude. 
Maximum trophic level increased with 
latitude ($\Beta=0.011$, $p$=0.011), but this relationship was weakened as species richness increased ($\beta$=-0.008, $
p$=0.011). Similarly, maximum trophic level increased with species richness ($\beta$=0.430, $p$=0.003) except at very 
high latitudes. Maximum trophic level decreased with connectance for estuarine food webs ($\beta$=-0.458, $p$=0.021), 
varied little in lakes ($\beta$=-0.127, $p$=0.192), and increased with connectance in marine, stream, and terrestrial 
food webs ($\beta$=0.219, $p$=0.005; $\beta$=0.249, $p$<0.001; and $\beta$=0.337, $p$=0.002, respectively). Taken 
together, these effects resulted in an increase in maximum trophic level with latitude at low species richnesses across 
all ecotypes and decreasing maximum trophic level with latitude at high species richness across all ecotypes 
(Fig. \ref{}. At moderate levels of species richness, trophic level increased with latitude for estuarine webs, showed 
little change for lake webs, and decreased for all other ecotypes. 


Qualitatively similar trends were observed for mean trophic level. Mean trophic level increased with latitude 
($\beta$=0.0109, $p$<0.001) and with species richness ($\beta$=0.616, $p$<0.001), but both trends were weakened by an 
interaction between latitude and species-richness ($\beta$=-0.008, $p$<0.001). The best-fitting model for mean trophic 
level did not include a connectance term. The increase in mean trophic level with species richness was highest for 
estuarine food webs ($\beta$=0.616, $p$<0.001) and lower for lakes ($\beta$=-0.608, $p$=0.232), streams 
($\beta$=0.238, $p$<0.001), terrestrial food webs ($\beta$=0.217, $p$=0.002), and marine food webs 
($\beta$=0.133, $p$<0.001). 


Generality showed the opposite trend, decreasing with increasing latitude ($\beta$=-0.017, $p$=0.003). This decrease 
was counteracted by a positive interaction between latitude and species richness ($\beta$=0.013, $p$=0.003), with the net result that generality increased with latitude for most ecotypes and species richnesses (Fig. \ref{}).
The main effect of species richness on generality varied across ecotypes. Generality decreased with increasing species 
richness for estuarine webs ($\beta$=-0.058, $p$=0.003) and increased for lakes ($\beta$=0.220, $p$=0.245), 
streams ($\beta$=0.543, $p$<0.001), terrestrial ($\beta$=0.603, $p$=0.005) and marine webs ($\beta$=0.761, $p$<0.001). 
Generality increased with connectance equally for all ecotypes ($\beta$=0.875, $p$<0.001).


\section*{Discussion}

The tendency of food-web structure to exhibit scaling relationships with both species richness and connectance has been 
well-established \citep{,,,,}. As species richness in particular is also know to vary systematically over latitude 
\citep{,,,,}, intuitively one might suspect that the scaling relationships of other food web properties might also vary
over latitude \citep{}. However, we found that only four of XX food web properties varied significantly over latitude, 
and that only the scaling relationships of trophic level and generality varied from the tropics to the poles.


A study of plant-animal mutualistic networks found the opposite trend to that observed here, that generality decreased 
with increasing latitude while increasing with increasing plant diversity \citep{Schleuning2012}. Unfortunately this 
study did not examine changes in scaling relationships over latitude... Regardless, this may indicate that food webs 
and mutualistic networks vary differently over latitudinal gradients of productivity, temperature, etc. 
\citet{Schleuning2012} propose that the latitudinal specialization gradient in mutualists derives from the gradient in 
plant diversity, as high resource diversity may lead to more generalized consumers. 


A contrary argument was proposed in an earlier study \citep{Vazquez2004}. In this study, it was noted that bipartite 
networks such as plant-pollinator and host-parasite networks are typically highly nested (specialists interact with 
subsets of the interaction partners of generalists), and that nestedness tends to increase with network size 
\citep{Vazquez2004}. Therefore, as the number of species in a community increases the number of extreme specialists in 
a community should increase and generality should decrease. Importantly, this argument suggests that latitude can 
affect generality only indirectly via species richnes. As the networks in our dataset did not increase in size towards 
the tropics, and there was an effect of latitude as well as species richness, it seems unlikely that either of these 
two arguments holds.

As we also observed an increase in both mean and maximum trophic level with 
latitude, it is possible that high-trophic level species are also more apt to be generalists. Generalism may be an 
important strategy for high-level consumers as their prey may be scarce or seasonal \citep{}. These results also 
suggest that the tropics may indeed have narrower niches \citep{}. Similar increases in geographic range size with 
latitude (Rapoport's rule) have been found for mammals \citep{Letcher1994}, XX \citep{} and XX \citep{}. Large 
geographic ranges imply climactic generalization \citep{Letcher1994}, but may also promote dietary generalization 
(i.e., a species with a wide range may be more likely to eat a variety of different prey depending on what is 
available). This is supported by the fact tha feeding on many prey within a trophic level (non-omnivorous generality) 
has been shown to decrease population variability in zooplankton \citep{Romanuk2006}.


[[other scaling paper]]


The overall consistency of scaling laws over latitude is remarkable and suggests that food webs worldwide assemble 
along similar ``rules'' regardless of temperature, productivity, or other latitudinal gradients. However, the changing
scaling laws for generality and trophic level suggest that there are also key differences in the ways that food webs 
are configured in the tropics and at higher latitudes. Importantly, the opposing trends for generality in food webs 
and mutualistic networks may suggest that latitude imposes different pressures on mutualistic versus antagonistic 
interactions between species. [[Need a punchier conclusion]]



