\documentclass[12pt]{article}  
\usepackage{amsmath}
\usepackage{url}
\usepackage[dvips]{graphicx}
\usepackage{multirow}
\usepackage{geometry}
\usepackage{pdflscape}
%\usepackage{rotating}
% make Figure 1 etc bold
\usepackage[labelfont=bf]{caption}
\usepackage{setspace}

\usepackage[running]{lineno}

% let's get some nature formatted citations
%\usepackage{overcite}
\usepackage[round]{natbib}

% some cheats to reduce the need to type complicated bits and pieces
\newcommand{\expect}[1]{\left\langle #1 \right\rangle}
\newcommand{\etal}{\textit{et al.\ }}

\newcommand{\beginsupplement}{%
        \setcounter{table}{0}
        \renewcommand{\thetable}{S\arabic{table}}%
        \setcounter{figure}{0}
        \renewcommand{\thefigure}{S\arabic{figure}}%
     }

% the abstract formatting
\newenvironment{sciabstract}{%
\begin{quote} \bf}
{\end{quote}}
\renewcommand\refname{References}

% margin sizes`
\topmargin 0.0cm
\oddsidemargin 0.2cm

\textwidth 16cm 
\textheight 21cm
\footskip 1.0cm


\title{Generality in food webs scales with species richness, not latitude}
\author{Alyssa Cirtwill, Daniel Stouffer, Tamara Romanuk}
\date{$^1$School of Biological Sciences\\University of Canterbury\\
Private Bag 4800\\Christchurch 8140, New Zealand}

\begin{document}
\maketitle
\baselineskip=8.5mm
 
\vspace{0.4 in}

%%%%%%%%%%%%%%%%%%%%%%%%%%%%%%%%%%%%%%%%%%%%%%%%%%%%%%%%%%%%%%%%%%%%%%%%%%%%%%%%%%%%%%%%%%%%%%%%%%%%%%%%%%%%%%%%%%
%%%%%%%%%%%%%%%%%%%%%%%%%%%%%%%%%%%%%%%%%%%%%%%%%%%%%%%%%%%%%%%%%%%%%%%%%%%%%%%%%%%%%%%%%%%%%%%%%%%%%%%%%%%%%%%%%%
%%
%%    Look into non-linear regressions for all properties (nls?)
%%%%%%%%%%%%%%%%%%%%%%%%%%%% 

\section*{Introduction}

%Food webs are a thing. (Mention gen, vul, LS.) %Food webs have strong scaling structure.
Food webs --networks of feeding links between species-- have been used for over a century to summarize the structure of 
ecological communities \citep{Williams2000,earlierwork} and to understand how that structure relates to environmental variables 
such as land use \citep{Townsend1998,Townsend2004c,Digel2014}, primary productivity \citep{}, and climate\citep{}. The latter
variables in turn have strong gradients over latitude, with productivity and temperature both being higher in the tropics while 
climate is more variable at high latitudes \citep{}. These variables affect both the resources available and species' 
metabolisms~\citep{Stegen2012}, and have been proposed as influences on the strength of 
interspecific interactions~\citep{Marquis2005,Schemske2009}. 
By shaping interactions between species, latitudinal gradients may also shape food-web structure.
Indeed, these latitudinal environmental gradients have been put forward as potential driverd for the 
latitudinal gradient in species richness, one of the most general and robust patterns in ecology \citep{Schemske2009a,,}.

% These gradients, together with the strength and generality of associations between biological processes and environmental variables, suggest that variation in network structure 
% may also be explained by ecological gradients \citep{Baiser2012}. 


One proposed link between species richness and latitude is the ``latitude-niche breadth hypothesis''~\citep{,Vazquez2004}. 
This hypothesis predicts that decreased seasonality in the tropics should lead to more stable populations, which in
turn should evolve smaller niches \citep{Vazquez2004}. These narrow niches should therefore allow more species to 
coexist in the tropics than can at higher latitudes. Alternatively, higher productivity of the tropics \citep{Brown2004}
may result in a broader niche space \citep{Davies2007} which could also allow greater diversification even if niche 
sizes are globally similar. Although the assumptions of the latitude-niche breadth hypothesis are only equivocably 
supported \citep{Vazquez2004}, it remains a compelling potential mechanism for the latitudinal gradient in species 
richness \citep{Lappalainen2006,Krasnov2008,Slove2010}. 


If decreased seasonality or higher productivity is the cause of higher species-richness in the tropics, then 
there should be other effects of latitude on food-web structure. Attempts to unravel the effects, however,
are complicated by known effects of species richness on many other network properties~\citep{Riede2010}. In particular,
narrower niches imply fewer links per species (i.e., greater specialization in the tropics~\citep{}). However,
average numbers of links per species tend to increase in larger food webs~\citep{Dunne2006,Riede2010}. This means
that a latitudinal effect on specialization may be obscured by a latitudinal gradient in species richness. If this is
the case, it may still be possible to uncover effects of latitude on specialization by examining the shape of the scaling 
relationship between specialization and species richness over changing latitude. Here, we use three measures of specialization:
mean links per species, mean generality (number of prey), and mean vulnerability (number of predators). By testing whether
latitude affects the scaling of each property with species richness, we test effects of latitude on specialization implied by
the ``latitude-niche breadth hypothesis''.



\section*{Methods}

\subsection*{Data Set} 

We compiled a list of 263 empirical food webs from
multiple sources (see Supplemental Information for web origins and selection
criteria). We recorded study site latitude from the original source where
possible or, where study sites were described but exact coordinates were not
given, obtained estimated coordinates using Google Earth \citep{GoogleEarth}.

% [[list in supplemental (347 GlobalWeb webs \citep{GlobalWeb}), supplemented 
% with 31 food webs used to study the relationship between consumer and resource body sizes (SizeWebs \citep{Brose2006})
% and seven food webs used to study the impact of parasites on network structure (ParWebs \citep{Dunne2013}).
% Of the GlobalWeb webs, 122 were rejected because their original source could not be found or was unpublished (53), they were source or sink
% webs rather than descriptions of an entire community (13), were focused on plant-pollinator, host-parasitoid, or 
% competitive interactions (8, 34, and 2, respectively), described inferred interactions in an extinct community (2), were ``generalized 
% schemes'' rather than being based on empirical observation (8), or because it was not clear which of a variety of 
% described sites the published food web represented (2). (See Table S1 for a list of rejected webs and reasons for their
% exclusion.) For three GlobalWeb webs which included parasites as a minor component of the community, as well as the 
% seven ParWebs webs, we included a modified version of each web which included only free-living species and the 
% interactions between them \citep{Dunne2013}. 


As the food webs in this dataset are derived from a variety of sources and were compiled over many decades, it
is likely that they vary in their resolution and in the amount of sampling effort invested in their assembly.
Many analyses of food-web structure reduce this variation by aggregating species with identical predator and prey
sets to form ``trophic species'' webs (e.g. ~\citep{Martinez1991,Vermaat2009,Dunne2004,Dunne2013}). As this study
is concerned directly with the number of species at a particular latitude, however, we did not wish to ignore 
species with redundant sets of interactions. We therefore analysed both original and trophic-species versions
of the dataset. As the version of the dataset used did not qualitatively change the results, we present only
the results for the original webs in the main text. In each case, we counted the number of species ($S$) and 
links in each web and used these to calculate the mean number of links per species ($Z$), mean generality 
($G$), and mean vulnerability ($V$) of each web. 


\subsection*{Relationships with Latitude}

To determine whether there were latitudinal gradients in food-web structure,
we first examined simple linear relationships between latitude and $S$, $Z$, $G$, and $V$. 
We fit models of the form

\begin{equation}
\label{Latfull}
S_{i} = \alpha_{0} + \alpha_{1} L_{i} + \alpha_{2} E_{i} + \alpha_{3} L_{i} E_{i} + \epsilon_{i} 
\end{equation}

where $S_{i}$ is the species richness of web $i$, $L_{i}$ its absolute
latitude (degrees north or south  regardless of direction), $E_{i}$ is the
ecosystem type of network $i$ (comprising terms for stream, marine, lake, and terrestrial networks with estuarine
networks being the intercept) and $\epsilon_{i}$ a residual error term. 
We then simplified these maximal models. To do this, we calculated the AIC
of the chosen maximal model, as well as the AIC's of a suite of candidate simplified models identified
using R~\citep{R} function dredge from packages MuMIn~\citep{MuMIn}. Simplified models were obtained by
systematically removing all possible combinations of terms from the full model.
The best-fitting model was then determined to be the model with the fewest terms 
where $\Delta$AIC\textless2. If several models shared the fewest number of terms 
and had $\Delta$AIC\textless2, the model with the lowest AIC in that set was chosen as the best-fit
model.


\subsection*{Scaling Relationships with S}

Next, we examined the form of the scaling relationship between each 
property ($Z$, $G$, and $V$) and $S$. The scaling relationship between $Z$ and 
$S$ has been shown to be a power law \citep{Riede2010} of the form 

\begin{equation}
\label{Power}
Z_{i}~\alpha S_{i}^{\beta}
\end{equation}

which is often re-expreseed in logarithmic form 

\begin{equation}
\label{Loglog}
\log{Z_{i}} ~ \log{\alpha} + \beta\log{S_{i}}
\end{equation}


Although these relationships are very similar, they imply different error distributions~\citep{Xiao2011}.
Specifically,~\ref{Power} implies a normally-distributed, additive error and~\ref{LogLog} a lognormal,
multiplicative error. As we have no \emph{a priori} reason to believe that our dataset has one error distribution
over another, we follow the recommendations in \citet{Xiao2011} and compared the two
model formulations explicitly. The model with the error distribution most resembling that observed in the empirical
data was then used to test for potential effects of latitude.


Although scaling relationships between $S$, $G$, and $V$ have not been explicitly examined (but see scaling 
relationships for the standard devaitions of each property in \citet{Riede2010}), we expect that they will follow
power laws similar to that of the relationship between $S$ and $Z$. This is because the links taken into account in
calculating $G$ and $V$ are subsets of the total links included when calculating $Z$. As with $Z$, we explicitly 
compared the error distributions of models for $G$ and $V$ using both the power-law and logarithmic formulations. 
In each case, we used the best-fitting equation as a template when assessing the effect of latitude on scaling with
species richness.



\subsection*{Effect of Latitude on Scaling}

% Logarithm of a sum is awful:   log(a+c) = log(a) + log(1+b^[log(c)-log(a)]) But can probably feed that into R...


We then assessed the impact of latitude on the scaling relationships between $Z$, $G$, $V$, and $S$. 
Even if there is no direct effect of latitude on specialization, perhaps due to contrasting effects
on species richness as described above, there may still be an effect of latitude on the scaling of
each property with species richness. In fact, if the species-rich tropics truly have narrower 
niches~\citep{Brown2004}, then we would expect to see less increase in mean numbers of links, 
predators, and prey with increasing species richness in the tropics.
In the context of the scaling relationships above, we wish to determine the effect of latitude on
the scaling exponent $\beta$. As when examining the relationships between latitude and each food 
web property directly, we included a factorial variable for ecosystem type (stream, lake, 
terrestrial, marine, or estuary), as well as interactions between food web type and latitude.


We therefore began by considering models of the form:

\begin{equation}
\label{PowerLat}
Z_{i}=\alpha S_{i}^{\beta_{0}+\beta_{1}L_{i}+\beta_{2}E_{i}+\beta_{3}LE_{i}} + \epsilon_{i} 
\end{equation}

where $S_{i}$ is the number of species in food web $i$, $L_{i}$ is the absolute latitude of food web $i$ (degrees from the equator, regardless of direction),
$E_{i}$ indicates the ecosystem type of food web $i$.


The logarithmic formulation of this model is:

\begin{equation}
\label{LogLat}
\log{Z_{i}} = \log{\alpha}+\beta_{0}\log{S_{i}} + \beta_{1}L\log{S_{i}} +\beta_{2}E\log{S_{i}} +\beta_{3}LE\log{S_{i}} +\epsilon_{i}
\end{equation}

For each property in $Z$, $G$, and $V$, we used the form of the equation that was best supported
when describing the scaling of the property with species richness alone~\citep{Xiao2011}.
We then simplified the model. As with the models describing the relationships between latitude and food-web
properties, we calculated the AIC of the maximal and all reduced models, created by systematically removing
all possible combinations of terms. Note that if an interaction term was included in a reduced model, all
associated main effects were also retained. $S$ was included in all reduced models. Reduced models were
identified using the R~\citep{R} function dredge from package MuMIn~\citep{MuMIn}. The best-fitting model
was then determined to be the model with the fewest terms where $\Delta$AIC<2. Where there were several
models with the fewest terms where $\Delta$AIC<2, the model with the lowest AIC in this set was chosen as
the best-fitting model. The best-fitting model was then fit using R~\citep{R} function lm from package stats~\citep{stats}.

% of  the University of Canberra's GlobalWeb database (www.globalwebdb.com) 
% \citep{}. Of these, 302 webs had accessible original sources. 
% 18 source and sink webs, which describe links to or from a focal taxon, were excluded as they are likely 
% to have different food web properties than webs attempting to
% describe the entire community of a site \citep{Williams2002}, as were 8 plant-pollinator, 34 host-parasitoid, and 2 
% competition-focused networks
% \citep{Riede2010}. Of the remaining 230 food webs, we excluded two paleowebs (comprised of probable interactions 
% between extinct or prehistoric species) \citep{Simenstad1978} and five ``generalized schemes'' 
% \citep{Nybakken1982,Percival1929,Swan1961,Landry1977,Petipa1979,Harrison1963}, as 
% we wished to measure
% directly observed rather than inferred interactions. We further excluded a food web aggregated from several sites in 
% the Pacific Ocean which were widely spread in both latitude and longitude and at markedly different successional stages
% \citep{Vinogradov1978}, such that it is not clear which site the food web represents. 
% This left 223 food webs with widely varying levels of taxonomic resolution, 11 of which contained humans. 




\section*{Results}

\subsection*{Scaling Relationships with $S$ and Latitude}

For relationships of $Z$, $G$, and $V$ to $S$, equation~\ref{LogLog} had a lower AIC than the corresponding
equation~\ref{Power}. This indicates that the data support an assumption of multiplicative log-normal
error better than an assumption of additive normal error. That is, [[but what does it mean?]] . 
We therefore used logarithmic-form models when assessing the effect of latitude on scaling relationships 
with $S$. 


Contrary to the expected latitudinal gradient, the best-fit version of equation~\ref{Latfull} for $S$
did not include a term for latitude or any of its interactions with ecosystem type. Species richness was
higher for stream ecosystems ($\alpha_{2:Stream}$=22.4, $p$\textless0.001) than other ecosystem types. 
Similarly, the best-fit version of equation~\ref{Latfull} for $G$ did not include 
any terms for latitude or interactions between latitude and ecosystem type. $G$ was lower in terrestrial
and lake food webs than in other ecosystem types ($\alpha_{2:Terrestrial}$=-3.77, $p$\textless0.001 and 
$\alpha_{2:Lake}$=-1.98, $p$=0.044). The best-fit models for $Z$ and $V$ did not include any terms for
latitude, ecosystem type, or their interactions.
the best-fit version of equation~\ref{Latfull} for $Z$ did not include a term for latitude,
ecosystem type, or any of their interactions. Food webs tended to contain about 3.53 links per species
($p$\textless0.001), with the average species being preyed upon by 3.53 predators ($p$\textless0.001).


\subsection*{Effect of Latitude on Scaling with S}

The effect of latitude on scaling relationships varied across ecosystem types.
Overall, specialization in marine and terrestrial food webs increased towards the poles, 
while specialization in lake food webs increased towards the equator (Fig.~\ref{scalinfig}). 
In estuarine and stream food webs, there was a great deal of variation, with links per species
and vulnerability increasing towards the poles and generality increasing towards the equator.
More specifically, the scaling of mean links per species with species richness increased with
increasing latitude ($\beta_{Latitude}$=0.004, $p$\textless0.001; Table~\ref{Bestfits}). This
increase, however, was reversed for marine and terrestrial food webs
($\beta_{Latitude:Marine}$=-0.005, $p$=0.007 and
$\beta_{Latitude:Terrestrial}$=-0.005, $p$=0.001). Scaling of mean vulnerability with
species richness showed the same pattern ($\beta_{Latitude}$=0.004, $p$\textless0.001;
$\beta_{Latitude:Marine}$=-0.005, $p$=0.007; and $\beta_{Latitude:Terrestrial}$=-0.005, $p$=0.002).
In contrast, scaling of mean generality with species richnesss increased with latitude 
for lake food webs ($\beta_{Latitude:Lake}$=0.006, $p$=0.002) but did not vary with latitude for
other ecosystem types ($\beta_{Latitude}$=-0.001, $p$=0.199).

\section*{Discussion}

The tendency of food-web structure to exhibit scaling relationships with
species richness has been well-established \citep{Dunne2004,Riede2010}. As
species richness in particular is also known to vary systematically over
latitude \citep{}, intuitively one might suspect that any relationship
between food-web properties such as generality might be due to the latitudinal
gradient in species richness. In this dataset, however, we did not find
overall latitudinal gradients in species richness, links per species, 
generality, or vulnerability. 


The lack of latitudinal gradients in food-web properties in this dataset contrasts
strongly with other studies which have found consistent variation in species richness,
[[other properties]] \citep{}. As numbers of species and links included in a food web
vary strongly with sampling effort as well as with the underlying diversity of the study
area, it is possible that the lack of latitudinal trends here is a result of researchers
tending to expend similar amounts of sampling effort across studies. This could result in
food webs describing species-rich tropical communities omitting more species and links
than species-poor arctic communities. In addition, it is worth noting that gradients in
species richness are generally measured for a single taxonomic group at a time~\citep{}.
It is possible that these taxa are not well-represented in our food webs and that the
dominant taxa in them do not have an underlying latitudinal gradient in richness. In
either case, the lack of association between species richness
and latitude in any ecosystem type nevertheless means that any effect of latitude on scaling relationships
between species richness and other properties is not being driven by underlying variation in
species richness, allowing us to more clearly assess effects of latitude on scaling.


These effects varied strongly across ecosystem types. In marine and terrestrial food webs,
scaling of links per species, generality, and vulnerability was not strongly affected by
latitude. This is consistent with the idea that species' niche breadths do not vary systematically
with temperature and productivity but that the niche space might be larger in species-rich communities
~\citep{Davies2007}. In contrast, scaling in of lake, estuarine, and stream was generally stronger
towards the poles. This means that species in tropical communities gain fewer additional feeding links
per additional species in the web, supporting the hypothesis that greater stability in the tropics
leads to narrower niches~\citep{Brown2004} and a higher proportion of specialists~\citep{}. As food webs
are generally aggregated in space and time, the greater increase in links in high latitude stream, 
estuarine, and lake food webs might reflect prey switching as different food sources become
available~\citep{}. 


\section*{Conclusion}

Our results were consistent with the latitude-niche breadth hypothesis, but only in certain ecosystem
types. This suggests that different types of food webs respond differently to differences in climate
driven by latitudinal gradients. [[A sentence or two explaining why this is important.]]


%\end{spacing}
\newpage
\bibliographystyle{jae}%Compile with jae.bst style file
\bibliography{noISN.bib}% your .bib file(s)

\newpage

\section*{Tables}

\begin{table}[!h]
\caption{Intercepts and slopes for the best-fitting model describing the relationship between species-richness, mean links per species, mean generality, and mean vulnerability and latitude. Note that none
of the best-fitting models included a term for latitude or an interaction between latitude and ecosystem type. Each `NA' indicates a term that was not included in the corresponding best-fit model.}
\label{Latlms}
\begin{tabular}{l | l l  l l  l l  l l}

\hline
\multirow{2}{*}{Property} & \multicolumn{2}{|c}{Intercept} & \multicolumn{2}{|c}{Stream} & \multicolumn{2}{|c}{Lake} & \multicolumn{2}{|c}{Terrestrial}\\
& $\alpha_{0}$ & $p$-value & $\alpha_{2}$ & $p$-value & $\alpha_{2}$ & $p$-value & $\alpha_{2}$ &$p$-value\\
\hline
Species richness  & 34.6 & \textless0.001 & 22.4 & 0.167 \textless0.001 & NA & NA & NA & NA \\
Links per species & 3.53 & \textless0.001 & NA & NA & NA & NA & NA & NA \\
Generality        & 7.32 & \textless0.001 & NA & NA & -1.98 & 0.044 & -3.77 & \textless0.001 \\
Vulnerability     & 3.53 & \textless0.001 & NA & NA & NA & NA & NA & NA \\
\hline
\end{tabular}
\end{table}

\begin{table}[!h]
\caption{Intercepts and slopes for the best-fitting model describing the relationship between species-richness, mean links per species, mean generality, and mean vulnerability and latitude in trophic-species versions of food webs. Note that none
of the best-fitting models included a term for latitude or an interaction between latitude and ecosystem type. `NA' a term that was not included in the corresponding best-fit model.}
\label{Latlms}
\begin{tabular}{l | l l  l l  l l  l l}

\hline
\multirow{2}{*}{Property} & \multicolumn{2}{|c}{Intercept} & \multicolumn{2}{|c}{Stream} & \multicolumn{2}{|c}{Lake} & \multicolumn{2}{|c}{Terrestrial}\\
& $\alpha_{0}$ & $p$-value & $\alpha_{2}$ & $p$-value & $\alpha_{2}$ & $p$-value & $\alpha_{2}$ &$p$-value\\
\hline
Species richness  & 34.6 & \textless0.001 & 22.4 & 0.167 \textless0.001 & NA & NA & NA & NA \\
Links per species & 3.53 & \textless0.001 & NA & NA & NA & NA & NA & NA \\
Generality        & 7.32 & \textless0.001 & NA & NA & -1.98 & 0.044 & -3.77 & \textless0.001 \\
Vulnerability     & 3.53 & \textless0.001 & NA & NA & NA & NA & NA & NA \\
\hline
\end{tabular}
\end{table}


\begin{table}[!h]
\caption{Intercepts and slopes for best-fit models describing the effect of latitude on the scaling of mean links per species, mean generality, and mean vulnerability with species richness. Each `NA' indicates a term that was not included in the corresponding best-fit model. These models refer to the original versions of food webs. }
\label{Bestfits}
\begin{tabular}{l | l l  l l  l l }
Effect          & \multicolumn{2}{c|}{Links per species} & \multicolumn{2}{c|}{Generality} 
                & \multicolumn{2}{c}{Vulnerability}\\
                & $\beta$ & $p$-value     &  $\beta$ & $p$-value    &  $\beta$ & $p$-value  \\
\hline
Intercept               &-0.453 & \textless0.001 & -0.249 & \textless0.001 & -0.453 & \textless0.001 \\
log(species)            & 0.419 & \textless0.001 &  0.663 & \textless0.001 &  0.419 & \textless0.001 \\
log(species):Lake       & 0.119 & \textless0.001 & -0.185 & 0.016          &  0.119 & \textless0.001 \\
log(species):Marine     & 0.255 & \textless0.001 &  0.090 & 0.002          &  0.255 & \textless0.001 \\
log(species):Terrestrial& 0.207 & 0.002          & -0.122 & \textless0.001 &  0.207 & 0.002 \\
log(species):Latitude   & 0.004 & \textless0.001 & -0.001 & 0.199          &  0.003 & \textless0.001 \\
log(species):Latitude:Marine&-0.004 & 0.007      &  NA    &  NA            & -0.005 & 0.007 \\
log(species):Latitude:Terrestrial&-0.005 & 0.002 &  NA    &  NA            & -0.005 & 0.002 \\
log(species):Latitude:Lake & NA & NA             &  0.006 & 0.002          &  NA    & NA    \\
\hline
$R^{2} (adjusted)$   & \multicolumn{2}{c|}{0.706} & \multicolumn{2}{c|}{0.658} & \multicolumn{2}{c}{0.706} \\
\hline
\end{tabular}
\end{table}


\begin{table}[!h]
\caption{Intercepts and slopes for best-fit models describing the effect of latitude on the scaling of mean links per species, mean generality, and mean vulnerability with species richness. Each `NA' indicates a term that was not included in the corresponding best-fit model. These models refer to the trophic species versions of food webs. }
\label{Bestfits}
\begin{tabular}{l | l l  l l  l l }
Effect          & \multicolumn{2}{c|}{Links per species} & \multicolumn{2}{c|}{Generality} 
                & \multicolumn{2}{c}{Vulnerability}\\
                & $\beta$ & $p$-value     &  $\beta$ & $p$-value    &  $\beta$ & $p$-value  \\
\hline
Intercept               & -0.561 & \textless0.001 & -0.299 & \textless0.001 & -0.561 & \textless0.001 \\
log(species)            &  0.733 & \textless0.001 &  0.622 & \textless0.001 &  0.733 & \textless0.001 \\
log(species):Lake       & -0.118 & 0.047          & -0.115 & 0.078          & -0.118 & 0.047 \\
log(species):Marine     &  0.051 & 0.040          &  0.120 & \textless0.001 &  0.051 & 0.040 \\
log(species):Stream     & -0.161 & 0.014          & -0.057 & 0.421          & -0.161 & 0.014 \\
log(species):Latitude   & -0.001 & 0.081          & -0.001 & 0.273          & -0.001 & 0.081 \\
log(species):Lat:Lake   &  0.006 & \textless0.001 &  0.006 & \textless0.001 &  0.006 & \textless0.001 \\
log(species):Lat:Stream &  0.004 & 0.018          &  0.003 & 0.059          &  0.004 & 0.018 \\
\hline
$R^{2} (adjusted)$   & \multicolumn{2}{c|}{0.706} & \multicolumn{2}{c|}{0.658} & \multicolumn{2}{c}{0.706} \\
\hline
\end{tabular}
\end{table}

\newpage

\section*{Figures}

% \begin{figure}[!h]
% \label{Latfigs}
% \includegraphics[width=.9\textwidth]{Figures/properties_vs_lat.eps}
% \caption{Relationships between latitude and species richness, mean links per species, mean generality, and mean vulnerability. No food web property showed a significant latitudinal gradient in this dataset (N=259 food webs).}
% \end{figure}

\begin{figure}[!h]
\label{scalingfig}
\includegraphics[width=.9\textwidth]{Figures/scaling_with_S_grey.eps}
\caption{Relationships between species richness and mean links per species, mean generality, and mean 
vulnerability (N=259 food webs). Scaling of mean links per species and mean vulnerability with species 
richness strengthened with latitude for estuarine, stream, and lake food webs. This strengthening was
reversed for marine and terrestrial food webs. Scaling of mean generality with species richness 
strengthened with latitude for lake food webs, but did not significantly vary with latitude for other
ecosystem types (N=259 food webs). }
\end{figure}


\end{document}


