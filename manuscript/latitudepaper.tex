\documentclass[12pt]{article}  
\usepackage{amsmath}
\usepackage{url}
\usepackage[dvips]{graphicx}
\usepackage{multirow}
\usepackage{geometry}
\usepackage{pdflscape}
%\usepackage{rotating}
% make Figure 1 etc bold
\usepackage[labelfont=bf]{caption}
\usepackage{setspace}

\usepackage[running]{lineno}

% let's get some nature formatted citations
%\usepackage{overcite}
\usepackage[round]{natbib}

% some cheats to reduce the need to type complicated bits and pieces
\newcommand{\expect}[1]{\left\langle #1 \right\rangle}
\newcommand{\etal}{\textit{et al.\ }}

\newcommand{\beginsupplement}{%
        \setcounter{table}{0}
        \renewcommand{\thetable}{S\arabic{table}}%
        \setcounter{figure}{0}
        \renewcommand{\thefigure}{S\arabic{figure}}%
     }

% the abstract formatting
\newenvironment{sciabstract}{%
\begin{quote} \bf}
{\end{quote}}
\renewcommand\refname{References}

% margin sizes`
\topmargin 0.0cm
\oddsidemargin 0.2cm

\textwidth 16cm 
\textheight 21cm
\footskip 1.0cm


\title{Scaling laws of trophic level and generality vary across latitude}
\author{Alyssa Cirtwill, Tamara Romanuk?}
\date{$^1$School of Biological Sciences\\University of Canterbury\\
Private Bag 4800\\Christchurch 8140, New Zealand}

\begin{document}
\maketitle
\baselineskip=8.5mm
 
\vspace{0.4 in}


%%%%%%%%%%%%%%%%%%%%%%%%%%%%%%%%%%%%%%%%%%%%%%%%%%%%%%%%%%%%%%%%%%%%%%%%%%%%%%%%%%%%%%%%%%%%%%%%%%%%%%%%%%%%%%%%%%
%%%%%%%%%%%%%%%%%%%%%%%%%%%%%%%%%%%%%%%%%%%%%%%%%%%%%%%%%%%%%%%%%%%%%%%%%%%%%%%%%%%%%%%%%%%%%%%%%%%%%%%%%%%%%%%%%%
%%
%%    Look into non-linear regressions for all properties (nls?)
%%%%%%%%%%%%%%%%%%%%%%%%%%%% 

\section*{Introduction}

Food webs (networks of feeding links between species) have been used for decades to summarize the structure of ecological communities \citep{Martinez2000,earlierwork}.
Comparisons of community structure around the world and across ecosystem types 
have revealed that the structure of biological networks is generally tightly
constrained, particularly by relationships between species richness or connectance
(the proportion of possible links that are realized) and other food-web properties \citep{Riede2010}.
These relationships have suggested that the structure of food webs is highly scale-dependent. 
Specifically, network properties such as fractions of species in different trophic groups 
have been shown to strongly depend on the species richness or connectance of
the networks \citep{Vermaat2009,Riede2010}. 


As changes in species richness and connectance can explain much of the variation in network structure \citep{Vermaat2009}, 
it follows that drivers of change in these two properties may also affect other properties. Species richness in particular 
exhibits a strong latitudinal gradient, believed to be driven by varying productivity and temperature at different latitudes 
\citep{Brown2004,Cardillo2005,Thompson2005,Davies2007}. An alternative view
posits that species' niches are narrower in the tropics, 
reducing competition and permitting greater adaptive radiation \citep{}. Although this view is only equivocally supported 
\citep{Vazquez2004}, it is also possible that the higher productivity of the tropics \citep{Brown2004} may result in a broader 
niche space \citep{Davies2007} which could also allow greater diversification even if niche sizes are globally similar.


In addition to species richness, there are general and robust latitudinal gradients in metabolic rates \citep{Stegen2012} and 
rates and intensities of species interactions \citep{Marquis2005,Schemske2009}. These gradients, together with the strength and 
generality of associations between biological processes and environmental variables, suggest that variation in network structure 
may also be explained by ecological gradients \citep{Baiser2011}. Importantly, some of these gradients may act directly 
on food-web properties such that they change over latitude beyond the expected changes due to changing species richness. For 
example, if niche sizes vary with latitude then mean numbers of predators, prey, and links per species may also vary with 
latitude. Some of this variation may be explained by variation in species richness, but as niche sizes are believed to be 
related directly to latitude it is reasonable to expect that there may also be latitudinal effects unrelated to those of 
changing species richness.


Here we examine the effect of latitude on food-web structure by examining
its impact on the scaling relationships of food-web properties with species
richness and connectance. We expect that food-web structure will scale strongly
with both species richness and connectance, and that the scaling relationships
of properties measuring degree of specialization (i.e., generality, vulnerability, and
links per species) will change over latitude. The scaling relationships of other
properties (for example the proportions of top, intermediate, and basal resources) may
be less affected by latitude.


\section*{Methods}

We compiled a list of 89 published food webs from the University of Canberra's GlobalWeb database (www.globalwebdb.com) \citep{}. We
included only food webs with a clearly defined study site that did not cover
more than 10 degrees of latitude. Source and sink webs were also excluded as
they are likely to have different food web properties than webs attempting to
describe the entire community of a site. We also excluded webs describing highly
modified or artificial ecosystems, and those including humans, as these could also
have different structural properties to more ``natural'' food webs.
Where several food webs were constructed from the same study site, without obvious boundaries between ``sites'' (e.g. three food
webs from Lake Nyasa), we aggregated the webs to form one summary web for the site. That is, we included all species and
interactions observed in any of the the original webs for that site in the summary web.
Similarly, we aggregated food webs created for the same location in different years. Both types of
aggregation should help to combat a known problem with food webs, namely that infrequent interactions and
rare species tend to be under-represented.


We then converted each web to a trophic-species version by aggregating species
with the same predators and prey. This helps to reduce variation in resolution
across different studies.
We then classified them both according to latitude
and ecosystem type (terrestrial, freshwater, marine, or estuary). We then calculated 17
food-web properties for each web (Table \ref{symboltable}).


In our analysis, we first addressed whether the diversity (S) or complexity (C, L/S) of our
published food webs varied over latitude (expressed as degrees away from the
equator regardless of direction) or factorial ecosystem type. We also included
year published as in independent variable to control for changing standards
for food web studies over time. We fit independent generalized linear models
for each of the logarithms of species richness, mean links per species, and
connectance. These models were used to determine any latitudinal gradients in the food-web
properties known to drive scaling relationships with other properties.


Second, we analysed the scaling relationships between the remaining 14 food-
web parameters and connectance and species richness. We used a series of
generalized linear models with a food-web parameter as the dependent variable
and latitude, ecosystem type, and their interactions with connectance and
species richness as independent variables. (parameter ~
latitude*ecotype*(log10(connectance)+log10(species richness))). For food web
properties that were proportions, we used general linear models with a
binomial error distribution. All other food web properties were log-transformed with a normal error distribution.
When modelling the standard deviation of generality, this
required the removal of 5 webs, all with 12 or fewer species, which had
standard deviations of generality of 0. These full models were then
systematically reduced using R function dredge \cite{} in package MuMIn
\cite{} in order to isolate the model with the highest AIC. These best-fitting models were 
then examined in detail. Significant slopes over species-richness or connectance were taken to indicate scaling
relationships. Significant interactions with latitude indicated that these scaling relationships varied from the tropics
to the poles. Different intercepts indicated differences between ecosystem types.


\section*{Results}

In this data set, none of the best fit models for species richness, connectance, or links per species included an 
effect of latitude, although all three properties varied across ecosystem types (Table 2). 


The best-fit models for most food-web properties exhibited scaling relationships with species-richness and/or 
connectance. Notable exceptions were the proportions of basal resources, herbivores, and omnivores, which varied only 
across year published; and the proportions of top predators and intermediate consumers, which varied with connectance 
and species richness and connectance (respectively) although not significantly.


For the food-web properties that did scale with species-richness or connectance, most of these relationships were 
constant over latitude. Latitude terms were not included in the best-fit models for SD of generality or 
SD of links. Best-fit models for vulnerability, SD of vulnerability, and path length included latitude terms but showed 
very weak relationships ($\beta$=3.52 x $10^{-17}$, $p$=0.170; $\beta$=-0.001, $p$=0.050; and $\beta$=0.005, $p$=0.069; 
respectively). None of the above models included an effect of latitude on scaling with S or C.


The scaling rules of mean trophic level and generality, however, did vary over latitude. 
Mean trophic level increased with latitude ($\beta$=0.011, $p$<0.001) and with species richness 
($\beta$=0.633, $p$<0.001), but both trends were weakened by an interaction between latitude and species-richness 
($\beta$=-0.008, $p$=0.001). The best-fitting model for mean trophic 
level did not include a connectance term. The increase in mean trophic level with species richness was highest for 
estuarine food webs ($\beta$=0.633, $p$<0.001) and lower for freshwater ($\beta$=0.259, $p$=0.002), terrestrial 
($\beta$=0.234, $p$=0.005), and marine food webs ($\beta$=0.152, $p$<0.001). For food webs with high species richness, 
trophic level declined with latitude while trophic level in species-poor webs showed the opposite trend 
(Fig. \ref{TL_all}). The number of species for which the slope of trophic level over latitude changed from negative to 
positive varied across ecosystem types.


Generality showed the opposite trend, increasing with latitude for webs with high species richness and decreasing with
latitude for species-poor webs (Fig. \ref{gen_all}).
This increase was driven by a positive interaction between latitude and species richness ($\beta$=0.014, $p$=0.002), as 
the main effect of latitude was negative ($\beta$=-0.018, $p$=0.002). Generality also decreased with species richness 
for estuarine food webs ($\beta$=-0.150, $p$=0.002) but increased with species richness for freshwater, terrestrial, 
and marine webs ($\beta$=0.453, $p$=0.005; $\beta$=0.523, $p$=0.006; and $\beta$=0.675, $p$=0.001; respectively). 
Generality increased with connectance equally in all ecotypes and across all latitudes ($\beta$=0.799, $p$<0.001).


\section*{Discussion}

The tendency of food-web structure to exhibit scaling relationships with both species richness and connectance has been 
well-established \citep{,,,,}. As species richness in particular is also known to vary systematically over latitude 
\citep{,,,,}, intuitively one might suspect that the scaling relationships of other food web properties might also vary
over latitude \citep{}. However, we found that only five of 17 food web properties varied significantly over latitude, 
and that only, of these, only mean and maximum trophi level and generality showed different scaling laws at different latitudes.


The lack of latitudinal gradients in most of the food-web properties considered here may be due to the lack of a 
latitudinal gradient in species richness in this data set. If so, that strongly implies that the latitudinal gradients 
that were observed were driven by some other factor than latitudinal changes in species richness. As we observed 
increasing generality away from 
the equator in conjunction with decreasing trophic level (at least in high-S food webs), it appears that species nearer
the top of the food web may be feeding on a wider variety of prey at the poles, lowering their trophic levels. 
As both feeding on many prey within a trophic level (non-omnivorous generality) \citep{Romanuk2006} and feeding on many 
trophic levels \citep{} have been shown to decrease population variability, high generality at higher latitudes may act 
as ``insurance'' for upper-level consumers by allowing them to switch prey according to availability. 


This contrasts with an earlier study of plant-animal mutualistic networks which found that generality increased in the 
tropics \citep{Schleuning2012}. That increase in generality was attributed to increasing species richness, as high 
resource diversity might lead to generalized consumers \citep{Schleuning2012}. While we also found increasing 
generality with increasing species richness in our dataset, it seems unlikely that higher species richness at the poles 
is driving the observed trend in generality. This suggests that while mutualistic and antagonistic interactions may
be similarly affected by species richness, they respond differently to other latitudinally-associated variables.


[[High TL/low generality in species-poor food webs?]]

% A contrary argument was proposed in an earlier study \citep{Vazquez2004}. In this study, it was noted that bipartite 
% networks such as plant-pollinator and host-parasite networks are typically highly nested (specialists interact with 
% subsets of the interaction partners of generalists), and that nestedness tends to increase with network size 
% \citep{Vazquez2004}. Therefore, as the number of species in a community increases the number of extreme specialists in 
% a community should increase and generality should decrease. Importantly, this argument suggests that latitude can 
% affect generality only indirectly via species richnes. As the networks in our dataset did not increase in size towards 
% the tropics, and there was an effect of latitude as well as species richness, it seems unlikely that either of these 
% two arguments holds.


\section*{Conclusion}

While the overall consistency of scaling laws between species richness and connectance and other food-web properties is 
remarkable, there are nonetheless signification variations in the scaling laws of trophic level and generality. These food web 
properties vary both over latitude and across different network types. [[something punchy]]

\newpage

\section*{Tables}

\begin{center}
\begin{table}[!h]
\caption{Symbols for and descriptions of food-web properties.}
\label{symboltable}
\begin{tabular}{c l l}
\hline
Symbol      & Name & Description \\
\hline
S & Species richness & Number of trophic species \\
LS & Links per species & Mean number of links per taxon \\
C & Connectance & Number of observed links divided by the number of possible links ($L/S^{2}$) \\
\%Top & \% Top predators & Percentage of taxa without consumers \\
\%Int & \% Intermediate consumers & Percentage of taxa with both consumers and resources \\
\%Basal & \% Basal Resources & Percentage of taxa without resources \\
\%Herb	& \% Herbivores	& Percentage of herbivores and detrivores (taxa that feed on basal taxa) \\
\%Omni & \%Omnivores & Percentage of omnivores (taxa that feed on ≥2 taxa with different trophic levels) \\
Gen & Generality & Mean number of resources per taxon \\
GenSD & SD of generality & Standard deviation of resources per taxon \\
Vul & Vulnerability & Mean number of consumers per taxon \\
VulSD & SD of vulnerability & Standard deviation of consumers per taxon \\
LinkSD & SD of links per species & Standard deviation of links (consumers and resources) per taxon \\
SWTL & Mean short-weighted trophic level & Short-weighted trophic level averaged across taxa \\
Path & Path length & Characteristic path length, the mean shortest food chain between species pairs \\
\hline
\end{tabular}
\end{table}
\end{center}
% NODF & Average Nestedness & Degree to which interactions among specialists are subsets or interactions among generalists \\
% Chain & Chain length & Mean food chain length \\
% Cluster & Clustering coefficient & Mean clustering coefficient (probability that two taxa linked to the same taxon are also linked) \\

\begin{center}
\begin{table}[!h]
\caption{Full models for S, LS, and C.}
\label{symbols}
\begin{tabular}{c c c c c c c}
\hline
 & \multicolumn{2}{c}{S} & \multicolumn{2}{c}{C} & \multicolumn{2}{c}{LS} \\
\hline
Effect & Estimate & $p$-value & Estimate & $p$-value & Estimate & $p$-value \\
\hline
Latitude 			& 0.002  & 0.873  & -0.006 & 0.480  & -0.004 & 0.613 \\ 
Estuary 			& 1.078  & 0.022  & -0.809 & 0.012  & 0.269 & 0.362 \\
Freshwater 			& 1.347  & \textless0.001  & -1.229 & \textless0.001 & 0.118 & 0.394 \\
Marine 				& 1.127  & \textless0.001 & -0.761 & \textless0.001 & 0.366 & 0.012 \\
Terrestrial 		& 1.097  & \textless0.001 & -0.919 & \textless0.001 & 0.178 & 0.354\\
Latitude:Freshwater & 0.002  & 0.902  & 0.010  & 0.299  & 0.012 & 0.194\\
Latitude:Marine 	& 0.003  & 0.847  & 0.001  & 0.902  & 0.004 & 0.662 \\
Latitude:Terrestrial & 0.001 & 0.923  & 0.003  & 0.724  & 0.005 & 0.597 \\
\end{tabular}
\end{table}
\end{center}

\newpage

\section*{Figures}

\begin{figure}[!h]
\label{gen_all}
\includegraphics[width=.9\textwidth]{Figures/gen_all.eps}
\caption{Relationships between mean generality and latitude for food webs with moderate connectance (0.1, mean of empirical webs) at high (S=109, solid line), moderate (S=36, dashed line), and low (S=5, dotted line) species richness.}
\end{figure}


\begin{figure}[!h]
\label{TL_all}
\includegraphics[width=.9\textwidth]{Figures/TL_all.eps}
\caption{Relationships between mean trophic level and latitude for food webs with moderate connectance (0.1, mean of empirical webs) at high (S=109, solid line), moderate (S=36, dashed line), and low (S=5, dotted line) species richness.}
\end{figure}


\bibliographystyle{jae}%Compile with jae.bst style file
\bibliography{noISN}

\end{document}


