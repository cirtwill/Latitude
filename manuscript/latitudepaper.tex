\documentclass[12pt]{article}  
\usepackage{amsmath}
\usepackage{url}
\usepackage[dvips]{graphicx}
\usepackage{multirow}
\usepackage{geometry}
\usepackage{pdflscape}
%\usepackage{rotating}
% make Figure 1 etc bold
\usepackage[labelfont=bf]{caption}
\usepackage{setspace}

\usepackage[running]{lineno}

% let's get some nature formatted citations
%\usepackage{overcite}
\usepackage[round]{natbib}

% some cheats to reduce the need to type complicated bits and pieces
\newcommand{\expect}[1]{\left\langle #1 \right\rangle}
\newcommand{\etal}{\textit{et al.\ }}

\newcommand{\beginsupplement}{%
        \setcounter{table}{0}
        \renewcommand{\thetable}{S\arabic{table}}%
        \setcounter{figure}{0}
        \renewcommand{\thefigure}{S\arabic{figure}}%
     }

% the abstract formatting
\newenvironment{sciabstract}{%
\begin{quote} \bf}
{\end{quote}}
\renewcommand\refname{References}

% margin sizes`
\topmargin 0.0cm
\oddsidemargin 0.2cm

\textwidth 16cm 
\textheight 21cm
\footskip 1.0cm


\title{Generality in food webs scales with species richness, not latitude}
\author{Alyssa Cirtwill, Tamara Romanuk?, Daniel Stouffer?}
\date{$^1$School of Biological Sciences\\University of Canterbury\\
Private Bag 4800\\Christchurch 8140, New Zealand}

\begin{document}
\maketitle
\baselineskip=8.5mm
 
\vspace{0.4 in}

%%%%%%%%%%%%%%%%%%%%%%%%%%%%%%%%%%%%%%%%%%%%%%%%%%%%%%%%%%%%%%%%%%%%%%%%%%%%%%%%%%%%%%%%%%%%%%%%%%%%%%%%%%%%%%%%%%
%%%%%%%%%%%%%%%%%%%%%%%%%%%%%%%%%%%%%%%%%%%%%%%%%%%%%%%%%%%%%%%%%%%%%%%%%%%%%%%%%%%%%%%%%%%%%%%%%%%%%%%%%%%%%%%%%%
%%
%%    Look into non-linear regressions for all properties (nls?)
%%%%%%%%%%%%%%%%%%%%%%%%%%%% 

\section*{Introduction}

%Food webs are a thing. (Mention gen, vul, LS.) %Food webs have strong scaling structure.
Food webs (networks of feeding links between species) have been used for decades to summarize the structure of 
ecological communities \citep{Williams2000,earlierwork}.
One of the earliest goals of constructing food webs was to measure the relationship between the species richness of
a community and its stability \citep{Dunne2006}, a goal which is still pursued today \citep{robustness studies}.
Some authors have recognized the potential for environmental effects to influence both stability \citep{Shurin2007} and 
other food web properties such as food chain length \citep{Townsend1998,Thompson2003,2004c,Young2013,Digel2014}.
Latitude is a particularly intruiging contender for an environmental influence on food web structure \citep{Shurin2007},
as latitudinal gradients in temperature \citep{}, primary productivity \citep{}, body size \citep{}, and 
species richness \citep{Schemske2009a} are among the most robust patterns in ecology[[back up from original draft]]. 


Latitude's strong influence on species richness, however, presents a problem for the analysis of food web structure.
Although it was initially thought that food-web structure was small-world and scale invariant \citep{}, further 
analyses with higher-quality data have revealed that most food-web structural properties scale strongly with species 
richness, connectance (the proportion of possible links that are realized), or both \citep{Riede2010} 
(see \citet{Dunne2006} for a review of the history of scaling laws in food web theory). This means that relationships
between latitude and food-web structure must be considered in light of the latitudinal gradient in species richness.
This is especially true for food-web properties that both scale with species richness and are related to candidate
explanations for the latitudinal diversity gradient, such as measures of specialization.


% Explain niche space hypothesis, Schemske, Vazquez. Then make clear that niche space suggests opposite gradient to ~S^B.
%Then give hypotheses.
A latitudinal gradient in specialization lies behind the ``latitude-niche breadth hypothesis'' \citep{Vazquez2004}.
This hypothesis predicts that decreased seasonality in the tropics should lead to more stable populations, which in
turn should evolve smaller niches \citep{Vazquez2004}. These narrow niches should therefore allow more species to 
coexist in the tropics than can at higher latitudes. Alternatively, higher productivity of the tropics \citep{Brown2004}
may result in a broader niche space \citep{Davies2007} which could also allow greater diversification even if niche 
sizes are globally similar. Although the assumptions of the latitude-niche breadth hypothesis are only equivocably 
supported \citep{Vazquez2004}, it remains a compelling potential mechanism for the latitudinal gradient in species 
richness \citep{Lappalainen2006,Krasnov2008,Slove2010}. 


Importantly, the expectation of the latitude niche-breadth hypothesis that higher species richness is associated with 
reduced competition is contary to the expected increase in the mean number of interactions per species (one potential 
measure of specialization) as species richness increases \citep{Dunne2006,Riede2010}. If, therefore, there is a
distinct effect of latitude on specialization then this effect should be visible in changing scaling relationships.
Here, we tested the effect of latitude on scaling relationships between species richness and mean links per species, 
mean generality (number of prey per predator), and mean vulnerability (number of predators per prey). Together, these
properties provide a measure a species' biotic specialization - that is, its specialization in terms of interactions 
with other species.


% In addition to species richness, there are general and robust latitudinal gradients in metabolic rates \citep{Stegen2012} and 
% rates and intensities of species interactions \citep{Marquis2005,Schemske2009}. These gradients, together with the strength and 
% generality of associations between biological processes and environmental variables, suggest that variation in network structure 
% may also be explained by ecological gradients \citep{Baiser2012}. Importantly, some of these gradients may act directly 
% on food-web properties such that they change over latitude beyond the expected changes due to changing species richness. For 
% example, if niche sizes vary with latitude then mean numbers of predators, prey, and links per species may also vary with 
% latitude. Some of this variation may be explained by variation in species richness, but as niche sizes are believed to be 
% related directly to latitude it is reasonable to expect that there may also be latitudinal effects unrelated to those of 
% changing species richness.



\section*{Methods}

\subsection*{Data Set}
We compiled a list of XX empirical food webs from an online database (GlobalWeb webs \citep{GlobalWeb}), supplemented 
with XX food webs used to study the relationship between consumer and resource body sizes (SizeWebs \citep{Brose2006})
and seven food webs used to study the impact of parasites on network structure (ParWebs \citep{Dunne2013}).
Of the GlobalWeb webs, XX were rejected because their original source could not be found (XX), they were source or sink
webs rather than descriptions of an entire community (XX), were focused on plant-pollinator, host-parasitoid, or 
competitive interactions (XX), described inferred interactions in an extinct community (2), were ``generalized 
schemes'' rather than being based on empirical observation (5), or because it was not clear which of a variety of 
described sites the published food web represented (1). (See Table S1 for a list of rejected webs and reasons for their
exclusion.) For three GlobalWeb webs which included parasites as a minor component of the community, as well as the 
seven ParWebs webs, we included a modified version of each web which included only free-living species and the 
interactions between them \citep{Dunne2013}. For each of the XXX webs included in the final database, we recorded the
study site latitude if possible or, where no latitude was given, obtained estimated coordinates of the described
study site using Google Earth \citep{GoogleEarth}. Webs were also classified according to ecosystem type (terrestrial,
lake, stream, estuary, or marine), year of publication (to control for changing standards of data collection over time \citep{Dunne2006}), and whether or not humans were included (11 webs included humans).


We then converted each web to a trophic-species web by aggregating species with identical predator and prey sets.
Using trophic species webs helps to reduce variation in resolution across different studies and across taxa within
a study \citep{Martinez1991,Vermaat2009,Dunne2004,Dunne2013}. The number of species ($S$) and links ($L$) in each 
trophic species
web were counted and used to calculate mean links per species ($Z$), generality ($G$), and vulnerability ($V$).


\subsection*{Scaling Relationships with S}

We were then able to examine the form of the scaling relationship between each property ($Z$, $G$, and $V$) and $S$.
The scaling relationship between $Z$ and $S$ has been shown to be a power law \citep{Riede2010} of the form 

\begin{equation}
\label{Power}
$Z=\alpha S^{\beta} + \epsilon $ \,,
\end{equation}

which is often re-expreseed in logarithmic form 

\begin{equation}
\label{Loglog}
$log(Z) = log(\alpha) + \beta log(S) + \epsilon $ \,.
\end{equation}

However, the two forms of the relationship imply different error structures: normally-distributed, additive error in 
\ref{Power} and lognormal, multiplicative error in \ref{Loglog}. As there is no intuitive reason to believe that our
data have one error distribution over another, we follow the recommendations in \citet{Xiao2011} and compared the two
model formulations explicitly. The model with the error distribution most resembling that observed in the empirical
data was then used to test for potential effects of latitude.


Although scaling relationships between $S$, $G$, and $V$ have not been explicitly examined (but see scaling 
relationships for the standard devaitions of each property in \citet{Riede2010}), we expect that they will follow
power laws similar to that of the relationship between $S$ and $Z$. This is because the links taken into account in
calculating $G$ and $V$ are subsets of the total links included when calculating $Z$. As with $Z$, we explicitly 
compared the error distributions of models for $G$ and $V$ using both the power-law and logarithmic formulations. 


\subsection*{Effect of Latitude}

% Logarithm of a sum is awful:   log(a+c) = log(a) + log(1+b^[log(c)-log(a)]) But can probably feed that into R...


Once the best formulation of the relationship between each property and $S$ had been determined, we then assessed
the impact of latitude on this relationship. There are two ways that properties might be affected by latitude:
1) the slope of the scaling relationship with $S$ might vary over latitude or 2) the residuals $\epsilon$ might
vary over latitude. We were primarily interested in the first possibility, and so included a term for the effect of 
latitude on the scaling exponent $\beta$ in equation \ref{Power}. In addition, as other studies have suggested that 
food webs in different ecosystem types have characteristic structural differences, we included a dummy variable for 
each ecosystem type (stream, lake, terrestrial, or marine, with estuary used as the `intercept' ecosystem type). 
We also considered other trends in the data that could affect relationships between latitude and species richness.
To account for changing standards in data collection over time \citep{Dunne2006} we included an effect of year of
publication. As 11 of our XX webs included humans, and therefore might have different properties than other networks,
we included a factorial effect of the presence of humans. Finally, some of our webs describe the same area in different
seasons or different years. To account for likely correlation among these webs, we included a random effect of site ID.


We therefore began by considering models of the form:

\begin{equation}
\label{PowerLat}
$Z_{i}=\alpha S^{\beta_{0}+\beta_{1}\delta_{i}+\beta_{2}\Ecotype+\beta_{X}\delta_{i}\Ecotype}\Year_{i}\Humans_{i} + \epsilon_{i} $ \,,

\end{equation}

where $\delta$ is the absolute latitude of food web $i$ (degrees from the equator, regardless of direction), 
$\beta_{1}$ represents the effect of latitude on the scaling exponent $\beta$ in equation \ref{Power}. If the scaling
relationship with S for a property suggested that lognormal error was more appropriate, we used the logarithmic re-
expression:

\begin{equation}
\label{LogLat}
$Z_{i}= log(\alpha)+\beta_{0}log(S) + \beta_{1}log(S)\delta_{i} +\beta_{2}log(S)Ecotype + \beta_{X}log(S)\delta_{i}Ecotype +log(year_{i}) +log(Humans_{i})  +\epsilon_{i} $ \,.

\end{equation}


We calculated the AIC of the chosen maximal model, as well as the AIC's of a suite of candidate simplified models.
Simplified models were obtained by systematically removing all possible combinations of terms from the full model.
The term for species richness was not removed from a model if any of the terms affecting the exponent $\beta$ were 
included in the model. The model with the lowest AIC was selected as the best-fitting model. For power-law formulations
this simplification was performed by hand. For logarithmic formulations, this simplification was performed using the
R \citep{R} function dredge from package MuMIn \citep{MuMIn}. We then used the R \citep{R} function lmer
from the package lmerTest \citep{lmerTest} to estimate the standardized effects ($\beta$s) for each fixed effect in the 
best-fitting models, as well as their corresponding $p$-values. 



% of  the University of Canberra's GlobalWeb database (www.globalwebdb.com) 
% \citep{}. Of these, 302 webs had accessible original sources. 
% 18 source and sink webs, which describe links to or from a focal taxon, were excluded as they are likely 
% to have different food web properties than webs attempting to
% describe the entire community of a site \citep{Williams2002}, as were 8 plant-pollinator, 34 host-parasitoid, and 2 
% competition-focused networks
% \citep{Riede2010}. Of the remaining 230 food webs, we excluded two paleowebs (comprised of probable interactions 
% between extinct or prehistoric species) \citep{Simenstad1978} and five ``generalized schemes'' 
% \citep{Nybakken1982,Percival1929,Swan1961,Landry1977,Petipa1979,Harrison1963}, as 
% we wished to measure
% directly observed rather than inferred interactions. We further excluded a food web aggregated from several sites in 
% the Pacific Ocean which were widely spread in both latitude and longitude and at markedly different successional stages
% \citep{Vinogradov1978}, such that it is not clear which site the food web represents. 
% This left 223 food webs with widely varying levels of taxonomic resolution, 11 of which contained humans. 





\section*{Results}

\subsection*{Scaling Relationships with S}

For models of $Z$, $G$, and $V$, the assumption of multiplicative log-normal error was better supported (Table \ref{Errorfits}). We therefore use only models with logarithmic form for the remainder of the paper.

\begin{tabular}

\label{Errorfits}
\caption{AIC's for model \ref{Power}, assuming normal error, and model \ref{Loglog}, assuming lognormal error.}
\begin{table}{l | l l | c}
Property & Normal error & Lognormal error & Model selected \\
\hline
$Z$ & 1151 &  767 & \ref{Loglog} \\
$G$ & 1266 & 1037 & \ref{Loglog} \\
$V$ & 1151 & 767 & \ref{Loglog} \\
\end{table}
\end{tabular}




In this data set, none of the best fit models for species richness, connectance, or links per species included an 
effect of latitude, although all three properties varied across ecosystem types (Table 2). 


The best-fit models for most food-web properties exhibited scaling relationships with species-richness and/or 
connectance. Notable exceptions were the proportions of basal resources, herbivores, and omnivores, which varied only 
across year published; and the proportions of top predators and intermediate consumers, which varied with connectance 
and species richness and connectance (respectively) although not significantly.


For the food-web properties that did scale with species-richness or connectance, most of these relationships were 
constant over latitude. Latitude terms were not included in the best-fit models for SD of generality or 
SD of links. Best-fit models for vulnerability, SD of vulnerability, and path length included latitude terms but showed 
very weak relationships ($\beta$=3.52 x $10^{-17}$, $p$=0.170; $\beta$=-0.001, $p$=0.050; and $\beta$=0.005, $p$=0.069; 
respectively). None of the above models included an effect of latitude on scaling with S or C.


The scaling rules of mean trophic level and generality, however, did vary over latitude. 
Mean trophic level increased with latitude ($\beta$=0.011, $p$<0.001) and with species richness 
($\beta$=0.633, $p$<0.001), but both trends were weakened by an interaction between latitude and species-richness 
($\beta$=-0.008, $p$=0.001). The best-fitting model for mean trophic 
level did not include a connectance term. The increase in mean trophic level with species richness was highest for 
estuarine food webs ($\beta$=0.633, $p$<0.001) and lower for freshwater ($\beta$=0.259, $p$=0.002), terrestrial 
($\beta$=0.234, $p$=0.005), and marine food webs ($\beta$=0.152, $p$<0.001). For food webs with high species richness, 
trophic level declined with latitude while trophic level in species-poor webs showed the opposite trend 
(Fig. \ref{TL_all}). The number of species for which the slope of trophic level over latitude changed from negative to 
positive varied across ecosystem types.


Generality showed the opposite trend, increasing with latitude for webs with high species richness and decreasing with
latitude for species-poor webs (Fig. \ref{gen_all}).
This increase was driven by a positive interaction between latitude and species richness ($\beta$=0.014, $p$=0.002), as 
the main effect of latitude was negative ($\beta$=-0.018, $p$=0.002). Generality also decreased with species richness 
for estuarine food webs ($\beta$=-0.150, $p$=0.002) but increased with species richness for freshwater, terrestrial, 
and marine webs ($\beta$=0.453, $p$=0.005; $\beta$=0.523, $p$=0.006; and $\beta$=0.675, $p$=0.001; respectively). 
Generality increased with connectance equally in all ecotypes and across all latitudes ($\beta$=0.799, $p$<0.001).


\section*{Discussion}

The tendency of food-web structure to exhibit scaling relationships with both species richness and connectance has been 
well-established \citep{,,,,}. As species richness in particular is also known to vary systematically over latitude 
\citep{,,,,}, intuitively one might suspect that the scaling relationships of other food web properties might also vary
over latitude \citep{}. However, we found that only five of 17 food web properties varied significantly over latitude, 
and that only, of these, only mean and maximum trophi level and generality showed different scaling laws at different latitudes.


The lack of latitudinal gradients in most of the food-web properties considered here may be due to the lack of a 
latitudinal gradient in species richness in this data set. If so, that strongly implies that the latitudinal gradients 
that were observed were driven by some other factor than latitudinal changes in species richness. As we observed 
increasing generality away from 
the equator in conjunction with decreasing trophic level (at least in high-S food webs), it appears that species nearer
the top of the food web may be feeding on a wider variety of prey at the poles, lowering their trophic levels. 
As both feeding on many prey within a trophic level (non-omnivorous generality) \citep{Romanuk2006} and feeding on many 
trophic levels \citep{} have been shown to decrease population variability, high generality at higher latitudes may act 
as ``insurance'' for upper-level consumers by allowing them to switch prey according to availability. 


This contrasts with an earlier study of plant-animal mutualistic networks which found that generality increased in the 
tropics \citep{Schleuning2012}. That increase in generality was attributed to increasing species richness, as high 
resource diversity might lead to generalized consumers \citep{Schleuning2012}. While we also found increasing 
generality with increasing species richness in our dataset, it seems unlikely that higher species richness at the poles 
is driving the observed trend in generality. This suggests that while mutualistic and antagonistic interactions may
be similarly affected by species richness, they respond differently to other latitudinally-associated variables.


[[High TL/low generality in species-poor food webs?]]

% A contrary argument was proposed in an earlier study \citep{Vazquez2004}. In this study, it was noted that bipartite 
% networks such as plant-pollinator and host-parasite networks are typically highly nested (specialists interact with 
% subsets of the interaction partners of generalists), and that nestedness tends to increase with network size 
% \citep{Vazquez2004}. Therefore, as the number of species in a community increases the number of extreme specialists in 
% a community should increase and generality should decrease. Importantly, this argument suggests that latitude can 
% affect generality only indirectly via species richnes. As the networks in our dataset did not increase in size towards 
% the tropics, and there was an effect of latitude as well as species richness, it seems unlikely that either of these 
% two arguments holds.


\section*{Conclusion}

While the overall consistency of scaling laws between species richness and connectance and other food-web properties is 
remarkable, there are nonetheless signification variations in the scaling laws of trophic level and generality. These food web 
properties vary both over latitude and across different network types. [[something punchy]]

%\end{spacing}
\newpage
\bibliographystyle{jae}%Compile with jae.bst style file
\bibliography{noISN.bib}% your .bib file(s)

\newpage

\section*{Tables}

\begin{center}
\begin{table}[!h]
\caption{Symbols for and descriptions of food-web properties.}
\label{symboltable}
\begin{tabular}{c l l}
\hline
Symbol      & Name & Description \\
\hline
S & Species richness & Number of trophic species \\
LS & Links per species & Mean number of links per taxon \\
C & Connectance & Number of observed links divided by the number of possible links ($L/S^{2}$) \\
\%Top & \% Top predators & Percentage of taxa without consumers \\
\%Int & \% Intermediate consumers & Percentage of taxa with both consumers and resources \\
\%Basal & \% Basal Resources & Percentage of taxa without resources \\
\%Herb	& \% Herbivores	& Percentage of herbivores and detrivores (taxa that feed on basal taxa) \\
\%Omni & \%Omnivores & Percentage of omnivores (taxa that feed on ≥2 taxa with different trophic levels) \\
Gen & Generality & Mean number of resources per taxon \\
GenSD & SD of generality & Standard deviation of resources per taxon \\
Vul & Vulnerability & Mean number of consumers per taxon \\
VulSD & SD of vulnerability & Standard deviation of consumers per taxon \\
LinkSD & SD of links per species & Standard deviation of links (consumers and resources) per taxon \\
SWTL & Mean short-weighted trophic level & Short-weighted trophic level averaged across taxa \\
Path & Path length & Characteristic path length, the mean shortest food chain between species pairs \\
\hline
\end{tabular}
\end{table}
\end{center}
% NODF & Average Nestedness & Degree to which interactions among specialists are subsets or interactions among generalists \\
% Chain & Chain length & Mean food chain length \\
% Cluster & Clustering coefficient & Mean clustering coefficient (probability that two taxa linked to the same taxon are also linked) \\

\begin{center}
\begin{table}[!h]
\caption{Full models for S, LS, and C.}
\label{symbols}
\begin{tabular}{c c c c c c c}
\hline
 & \multicolumn{2}{c}{S} & \multicolumn{2}{c}{C} & \multicolumn{2}{c}{LS} \\
\hline
Effect & Estimate & $p$-value & Estimate & $p$-value & Estimate & $p$-value \\
\hline
Latitude 			& 0.002  & 0.873  & -0.006 & 0.480  & -0.004 & 0.613 \\ 
Estuary 			& 1.078  & 0.022  & -0.809 & 0.012  & 0.269 & 0.362 \\
Freshwater 			& 1.347  & \textless0.001  & -1.229 & \textless0.001 & 0.118 & 0.394 \\
Marine 				& 1.127  & \textless0.001 & -0.761 & \textless0.001 & 0.366 & 0.012 \\
Terrestrial 		& 1.097  & \textless0.001 & -0.919 & \textless0.001 & 0.178 & 0.354\\
Latitude:Freshwater & 0.002  & 0.902  & 0.010  & 0.299  & 0.012 & 0.194\\
Latitude:Marine 	& 0.003  & 0.847  & 0.001  & 0.902  & 0.004 & 0.662 \\
Latitude:Terrestrial & 0.001 & 0.923  & 0.003  & 0.724  & 0.005 & 0.597 \\
\end{tabular}
\end{table}
\end{center}

\newpage

\section*{Figures}

\begin{figure}[!h]
\label{gen_all}
\includegraphics[width=.9\textwidth]{Figures/gen_all.eps}
\caption{Relationships between mean generality and latitude for food webs with moderate connectance (0.1, mean of empirical webs) at high (S=109, solid line), moderate (S=36, dashed line), and low (S=5, dotted line) species richness.}
\end{figure}


\begin{figure}[!h]
\label{TL_all}
\includegraphics[width=.9\textwidth]{Figures/TL_all.eps}
\caption{Relationships between mean trophic level and latitude for food webs with moderate connectance (0.1, mean of empirical webs) at high (S=109, solid line), moderate (S=36, dashed line), and low (S=5, dotted line) species richness.}
\end{figure}


\end{document}


