\documentclass[12pt]{article}  
\usepackage{amsmath}
\usepackage{gensymb}
\usepackage{url}
\usepackage[dvips]{graphicx}
\usepackage{multirow}
\usepackage{geometry}
\usepackage{pdflscape}
%\usepackage{rotating}
% make Figure 1 etc bold
\usepackage[labelfont=bf]{caption}
\usepackage{setspace}

\usepackage[running]{lineno}

% let's get some nature formatted citations
%\usepackage{overcite}
% \usepackage[round]{natbib}
\usepackage[numbers,sort&compress]{natbib}

% some cheats to reduce the need to type complicated bits and pieces
\newcommand{\expect}[1]{\left\langle #1 \right\rangle}
\newcommand{\etal}{\textit{et al.\ }}

\newcommand{\beginsupplement}{%
        \setcounter{table}{0}
        \renewcommand{\thetable}{S\arabic{table}}%
        \setcounter{figure}{0}
        \renewcommand{\thefigure}{S\arabic{figure}}%
     }

% the abstract formatting
\newenvironment{sciabstract}{%
\begin{quote} \bf}
{\end{quote}}
\renewcommand\refname{References}

% margin sizes`
\topmargin 0.0cm
\oddsidemargin 0.2cm

\textwidth 16cm 
\textheight 21cm
\footskip 1.0cm
\begin{document}


\title{Latitudinal gradients in biotic niche breadth vary across ecosystem types.}
\author{Alyssa R. Cirtwill$^{1,2}$, Daniel B. Stouffer$^{1}$, Tamara N. Romanuk$^{2}$}
\date{\small$^1$Centre for Integrative Ecology\\School of Biological Sciences\\University of Canterbury\\
Private Bag 4800\\Christchurch 8140, New Zealand \\
\medskip$^2$Department of Biology\\
Life Science Centre, Dalhousie University\\1355 Oxford St., P0 BOX 15000\\
Halifax NS, B3H 4R2, Canada\\}

\maketitle
\baselineskip=8.5mm
 
\vspace{-0.3 in}

\begin{spacing}{1.0}
\section*{Abstract}

Several properties of food webs---the networks of feeding links between
species---are known to vary systematically with the species richness of the
underlying community.  Under the ``latitude-niche breadth hypothesis'', which
predicts that species in the tropics will tend to evolve narrower niches, one
might expect that these scaling relationships could also be affected by
latitude. To test this hypothesis, we analysed the scaling relationships
between species richness and average generality, vulnerability, and links per
species across a set of 196 empirical food webs. In estuarine, marine, and 
terrestrial food webs there was no effect of latitude on any scaling relationship,
suggesting constant niche breadth in these habitats. In freshwater communities,
on the other hand, there were strong effects of latitude on scaling relationships,
supporting the latitude-niche breadth hypothesis. These contrasting findings
indicate that it may be more important to account for habitat than latitude
when exploring gradients in food-web structure.

\end{spacing}

\linenumbers

\section*{Introduction} 

  %Food webs are a thing. (Mention gen, vul, LS.) %Food webs have strong scaling structure.

  Food webs --networks of feeding links between species-- have been used for
  several decades to summarise the structure of  ecological
  communities~\cite{Paine1966,Williams2000,Petchey2008} and to understand how
  that structure relates to environmental variables  such as habitat
  type~\cite{Briand1983,Shurin2006}, primary
  productivity~\cite{Townsend1998,Thompson2005a,Vermaat2009}, and
  climate~\cite{Petchey2010,Baiser2012}. The latter variables in turn have
  strong gradients over latitude, with productivity and temperature both being
  higher in the tropics while  climate is more variable at high
  latitudes~\cite{Field1998}. These variables affect both the resources
  available and species'
  metabolisms~\cite{White2007,OConnor2009,Hechinger2011,White2011}, and have
  been proposed as determinants of the strength of  interspecific
  interactions~\cite{Schemske2009,Lang2012,Schleuning2012}.  By modulating
  interactions between species, latitudinal gradients may also shape food-web
  structure. Indeed, these latitudinal environmental gradients have been put
  forward as potential drivers for the  latitudinal gradient in species
  richness, one of the most general and robust patterns in
  ecology~\cite{Kaufman1995,Macpherson2002,Schemske2009}.

  % These gradients, together with the strength and generality of associations between biological processes and environmental variables, suggest that variation in network structure 
  % may also be explained by ecological gradients \cite{Baiser2012}. 


  One proposed link between species richness and latitude is the 
  ``latitude-niche breadth hypothesis''~\cite{Vazquez2004}.  This hypothesis 
  predicts that decreased seasonality in the tropics should lead to more stable
  populations, which in turn should evolve smaller niches~\cite{Vazquez2004}.
  These narrow niches should therefore allow more species to  coexist in the
  tropics than at higher latitudes. Alternatively, the higher productivity of
  the tropics~\cite{Brown2004} may result in a broader niche
  space~\cite{Davies2007} which could also sustain greater biodiversity even
  if niche  sizes are globally similar. Although the assumptions of the
  latitude-niche breadth hypothesis are only equivocally
  supported~\cite{Vazquez2004}, it remains a compelling potential mechanism
  for the latitudinal gradient in species
  richness~\cite{Lappalainen2006,Krasnov2008,Slove2010}.


  If the latitude-niche breadth hypothesis is correct, there should also be
  direct relationships between latitude and the degree of specialisation
  (i.e., the breadth of the Eltonian niche; ~\cite{Elton1927,Leibold2010}) of
  species within food webs. Specifically, narrower niches in the tropics would
  equate to greater specialisation (narrower niches) while constant niche
  sizes but greater productivity would translate to constant specialisation
  and niche width across latitude (Fig.~\ref{concept}). Attempts to unravel
  these effects, however, are complicated by known relationships between
  species richness and many other network properties~\cite{Riede2010}.  For
  example, narrower niches imply fewer links per species (i.e., greater
  specialisation) in the tropics (\cite{Marra1997,Dyer2007}; but
  see~\cite{Schleuning2012}). However, average numbers of links per species
  tend to increase in larger food webs~\cite{Dunne2006,Riede2010}. This means
  that a latitudinal effect on specialisation may be obscured by a latitudinal
  gradient in species richness.


  If this is the case, it may still be possible to uncover effects of latitude
  on specialisation by examining the shape of the scaling relationship between
  specialisation and species richness over changing latitude.  By testing
  whether latitude affects the scaling of each property with species richness,
  we test for the effects of latitude on specialisation predicted by the
  ``latitude-niche breadth hypothesis'' (Fig.~\ref{concept}).  If the scaling
  of specialisation with species richness is weaker in the  tropics (i.e., if
  species gain fewer links, predators, or prey as the size  of the network
  increases), this will indicate narrower niches at the tropics.  If, however,
  the scaling of specialisation with species richness does not  vary over
  latitude, this will indicate that niches are similarly-sized worldwide but
  that there is a broader niche space in the tropics. Additionally, as food
  webs describing different ecosystem types may differ in their
  topology~\cite{Dunne2004,Shurin2006}, we also explored the differences in
  scaling relationships across ecosystem types. Here, we use three measures of
  specialisation; mean links per species, mean generality (number of prey),
  and mean vulnerability (number of predators).


\section*{Methods}

  \subsection*{Data Set} 

    We compiled a list of 196 empirical food webs from
    multiple sources (see \emph{Appendix S1} for web origins and selection
    criteria). We recorded study site latitude from the original source where
    possible or, where study sites were described but exact coordinates were not
    given, obtained estimated coordinates using Google Earth~\cite{GoogleEarth}.
    If a range of latitudes (e.g. $42-49^{\circ}N$) was provided, we used the midpoint
    of this range. We grouped food webs by ecosystem type (stream, N=71; lake, N=47; marine, 
    N=28; estuarine, N=18; and terrestrial, N=31) according to their designation in 
    previous aggregations of food webs (i.e.,~\cite{GlobalWeb,Riede2011,Dunne2013}).


    As the food webs in this dataset are derived from a variety of sources and were compiled over many decades, it
    is likely that they vary in their resolution and in the amount of sampling effort invested in their assembly.
    Many analyses of food-web structure attempt to reduce this variation by using food webs comprised of ``trophic
    species'' --aggregations of species with identical sets of predators and prey-- rather than  
    species \emph{per se}~\cite{Martinez1991,Dunne2004,Vermaat2009,Dunne2013}. As our study
    is concerned directly with the number of species at a particular latitude, however, we did not wish to ignore 
    species with identical sets of interactions. We therefore analysed both original (i.e., without aggregating 
    any species) and trophic-species (i.e., after aggregating species with identical predators and prey) versions
    of the dataset; in each case using the number of species and 
    feeding links in each web to calculate the mean link density (number of links per species), mean generality 
    (number of prey per species), and mean vulnerability (number of predators per species) of the web. 
    The version of the dataset used did not qualitatively change the results, suggesting that
    the scaling relationships between species richness, other food-web properties, and latitude are very 
    similar whether or not species with identical sets of predators and prey are included. For simplicity, 
    here we present only the results for the original (un-aggregated, original species) webs.


  \subsection*{Gradients over Latitude}

    To put our dataset in the context of other research on latitudinal gradients in species richness,
    we first examined simple linear relationships between latitude and each of 
    species richness, links per species, generality, vulnerability, and proportions
    of basal resources, intermediate consumers, and top predators. We fit models of the form

    \begin{equation}
    \label{Latfull}
    S_{i} = \alpha_{0} + \alpha_{1} L_{i} + \alpha_{2} E_{i} + \alpha_{3} L_{i} E_{i} + \epsilon_{i} ,
    \end{equation}

    \noindent where $S_{i}$ is the species richness of web $i$, $L_{i}$ its absolute
    latitude (degrees north or south  regardless of direction), $E_{i}$ is a categorical
    variable indicating the ecosystem type of network $i$ (comprising terms for stream, 
    marine, lake, and terrestrial networks with estuarine
    networks corresponding to $E_{i}=0$) and $\epsilon_{i}$ is a residual error term. 
    We next calculated the AIC
    of the maximal model as well as the AIC's of a suite of candidate simplified models identified
    using the R~\cite{R} function dredge from package MuMIn~\cite{MuMIn}. 
    Simplified models were obtained by
    systematically removing all possible combinations of terms from the full model.
    The best-fitting model was then determined to be the model with the fewest terms 
    where $\Delta$AIC\textless2, as this model is the least likely to suffer from over-fitting. 
    


  \subsection*{Scaling Relationships with Species Richness}

    The scaling relationship between link density ($Z$) and species richness ($S$)
    has been shown to be a power law~\cite{Riede2010} of the form 

    \begin{equation}
    \label{Power}
    Z_{i} \sim \alpha S_{i}^{\beta}  ,
    \end{equation}

    \noindent which is often re-expressed in logarithmic form 

    \begin{equation}
    \label{Loglog}
    \log{Z_{i}} \sim \log{\alpha} + \beta\log{S_{i}}  .
    \end{equation}

    \noindent As the two forms imply a statistical fit to the data to different error 
    distributions, neither of which has strong
    \emph{a priori} justification in our dataset, we followed the recommendations in~\cite{Xiao2011}
    to compare the two model formulations explicitly (see \emph{Appendix S2} for details). 
    The logarithmic form (equation~\ref{Loglog}) provided the better fit to the data,
    as did the logarithmic forms of similar models for the scaling of generality and vulnerability. 
    We therefore used and present logarithmic models throughout the rest of the analyses.


  \subsection*{Effect of Latitude on Scaling}


    After determining the appropriate form of the scaling relationship, we
    then assessed the impact of latitude on the scaling relationships between
    species richness and link density, generality, vulnerability. In the
    context of the scaling relationships above, note  that this implies that
    we wished to determine the effect of latitude on the scaling exponent
    $\beta$. We included a categorical variable for
    ecosystem type (stream, lake,  terrestrial, marine, or estuary), as well
    as interactions between food web type and latitude.


    We therefore began by considering models of the form

    \begin{equation}
    \label{PowerLat}
    Z_{i}=\alpha S_{i}^{\beta_{0}+\beta_{1}L_{i}+\beta_{2}E_{i}+\beta_{3}LE_{i}} + \epsilon_{i} ,
    \end{equation}

    \noindent where $S_{i}$ is the species richness of web $i$, $L_{i}$ its absolute
    latitude (degrees north or south  regardless of direction), $E_{i}$ is a categorical
    variable indicating the ecosystem type of network $i$ (comprising terms for stream, 
    marine, lake, and terrestrial networks with estuarine
    networks corresponding to $E_{i}=0$) and $\epsilon_{i}$ is a residual error term.
    The logarithmic formulation of this model is

    \begin{equation}
    \label{LogLat}
    \log{Z_{i}} = \log{\alpha}+\beta_{0}\log{S_{i}} + \beta_{1}L\log{S_{i}} +\beta_{2}E\log{S_{i}} +\beta_{3}LE\log{S_{i}} +\epsilon_{i} .
    \end{equation}

    We then simplified each model following the procedure described above.
    As a supplemental check to ensure that variation in sampling effort
    across food webs was not responsible for the trends we observed, we then
    repeated our analyses using jackknifed data sets in which we 1) sequentially
    removed each food web in the dataset and 2) sequentially removed sets of food webs
    that shared a common author. The first jackknife essentially controls for the
    influence of any single outlier, while the second controls for the influence
    of particular research groups, some of which contributed large numbers of food
    webs (up to 27) to the dataset. Parameter estimates for the simplified models 
    varied very little across either jackknife test (see \emph{Appendix S3} for details),
    indicating that the trends we observed were not due to either strong outliers or
    to substantial differences in sampling effort across research groups.
    % If several models shared the fewest
    % number of terms  and had $\Delta$AIC\textless2, the model with the lowest
    % AIC in that set was chosen as the best-fit model.


    We then repeated the above analyses to measure the 
    effect of latitude on the scaling of specialisation 
    with numbers and proportions of species at different 
    trophic levels (i.e., basal resources, intermediate 
    consumers, and top predators). As different trophic 
    levels represent different sections of the biotic niche 
    space, we expected that latitude might affect the 
    scaling of specialisation with the proportion of 
    species in each trophic level differently. The results
    of these supplemental analyses largely echo the the
    results of the species-richness analyses, and are 
    presented in~\emph{Appendix S4}.


\section*{Results}

  Link density (mean number of feeding links per species), generality (mean
  number of prey per species), and vulnerability (mean number of predators per
  species) were strongly and positively correlated ($R^2$=0.891 for link
  density and generality, $R^2$\textgreater0.999 for link density and
  vulnerability, and $R^2$=0.890 for generality and vulnerability). Contrary
  to the expected latitudinal gradient, the best-fit version of
  equation~(\ref{Latfull}) for species richness did not include a significant
  effect of latitude for any ecosystem type. Further, there were no significant
  relationships between link density, generality, or vulnerability with
  latitude for any ecosystem type.


  Each measure of specialisation increased with increasing
  species richness ($\beta_0$=0.637, $p$\textless0.001; $\beta_0$=0.553,
  $p$\textless0.001; and $\beta_0$=0.637, $p$\textless0.001, respectively;
  Fig.~\ref{props_v_lat}). For estuarine, marine, and terrestrial food webs the
  strength of this scaling did not vary with latitude
  ($\beta_{Latitude}$=-0.001, $p$=0.365 for link density;
  $\beta_{Latitude}$=-0.001, $p$=0.535 for generality; and
  $\beta_{Latitude}$=-0.001, $p$=0.366 for vulnerability; Fig.~\ref{S}). In
  lake food webs, however, the scaling of each property was stronger towards
  the poles ($\beta_{Latitude:Lake}$=0.004, $p$=0.019;
  $\beta_{Latitude:Lake}$=0.005, $p$=0.004; and
  $\beta_{Latitude:Lake}$=0.004, $p$=0.018, respectively). In stream food
  webs, generality increased more rapidly towards the poles
  ($\beta_{Latitude:Stream}$=0.007, $p$=0.001) while link density and
  vulnerability did not vary with latitude (i.e., the interaction term 
  $\beta_{Latitude:Stream}$ was not retained in the best-fit models).


\section*{Discussion}

  The tendency of food-web structure to exhibit scaling relationships with
  species richness has been well-established~\cite{Dunne2004,Riede2010}. As
  species richness in particular is also known to vary systematically over
  latitude~\cite{Kaufman1995,Macpherson2002,Hillebrand2004,Schemske2009},
  intuitively one might suspect that any relationship between food-web
  properties such as generality might be due to the latitudinal gradient in
  species richness. In this dataset, however, we found no evidence to support
  latitudinal gradients in species richness, links per species, generality, or
  vulnerability.
  % , or the proportions of food webs accounted for by basal
  % resources, intermediate consumers, and top predators, except in lake and
  % stream food webs where the proportions of top predators tended to decrease
  % towards the poles  (see \emph{Appendix S3} for details).


  The lack of a latitudinal gradient in species richness in this dataset
  contrasts strongly with other
  studies~\cite{Kaufman1995,Macpherson2002,Hillebrand2004,Schemske2009}. As
  numbers of species and links included in a food web vary strongly with
  sampling effort as well as with the underlying diversity of the study area,
  it is possible that the lack of latitudinal trends here is a result of
  researchers tending to expend similar amounts of sampling effort across
  studies. This could result in food webs describing species-rich tropical
  communities omitting more species and links than studies of species-poor
  arctic communities if research groups spend similar person-hours assembling
  webs and can observe similar numbers of species and links per person-hour.
  In addition, it is worth noting that gradients in species richness are
  generally measured for a single taxonomic group at a
  time~\cite{Kaufman1995,Macpherson2002,Hillebrand2004,Schemske2009}. It is
  possible that these taxa are not well-represented in our food webs and that
  the dominant taxa that are represented do not have an underlying latitudinal
  gradient in richness. In either case, the lack of a strong association
  between species richness and latitude in any ecosystem type means that any
  effect of latitude on other scaling relationships is not being driven by an
  underlying latitudinal gradient in species richness. This is fortunate since
  the lack of confounding effects of latitude allows us to more clearly assess
  effects of latitude on scaling with species richness.


  Scaling of links per species, generality, and vulnerability with species
  richness varied strongly across ecosystem types. In estuarine, marine, and
  terrestrial food webs scaling of each property varied little with latitude.
  This is consistent with the idea that species' niche breadths do not vary
  systematically with temperature and productivity but that the niche space
  might be larger in species-rich communities~\cite{Davies2007}. Rather than
  niche space depending on temperature and productivity, it may be that species
  diversity itself affects the biotic niche space available to species (although climate
  may affect speciation rates and therefore the diversity in a region~\cite{Currie2004}). 
  For example, as the plant diversity of a community increases both the 
  variety of food available to herbivores and the structural variety of the habitat will also increase.


  Unlike other ecosystem types, the scaling of generality in lake and stream
  food webs was stronger (i.e., generality increased more steeply with
  increasing species richness) in higher-latitude food webs. In lake food webs,
  this trend was echoed in the scaling relationships between species richness
  and vulnerability and links per species. This means that species in tropical
  freshwater communities gain fewer additional feeding links per additional
  species in the web and that species in tropical lakes also gain fewer
  predators, and fewer links in general, per additional species than species
  in high-latitude lakes. These trends are consistent with the hypothesis that
  greater stability in the tropics leads to narrower niches~\cite{Vazquez2004}
  and a higher proportion of specialists. 


  That freshwater food webs supported the hypothesis of narrower niches in the
  tropics --while other ecosystem types did not-- is noteworthy given that
  these ecosystems (especially streams) are known for being highly variable
  and that seasonal variability is one of the proposed drivers of the
  latitude-niche breadth hypothesis~\cite{Vazquez2004}. Both streams and lakes
  can experience severe changes in water temperature and volume (e.g., floods,
  drying, freezing) that remove food or other resources (notably oxygen during
  freezing events)~\cite{Winterbourn1997,Meding2001}. These events are often
  linked to seasonal events such as snowmelts or summer
  drought~\cite{Winterbourn1997}). Further, both temperate streams and lakes
  tend to experience seasonal strong pulses of allochthonous inputs (e.g.,
  fallen leaves,terrestrial
  invertebrates~\cite{Nakano2001,Lennon2004,Zeng2008}. These trends combined
  mean that, relative to estuarine and marine communities, freshwater food 
  webs may experience high turnover in both community composition and 
  productivity~\cite{Tilzer1988,Magalhaes1993,Baird1989}. Notable exceptions from the above
  trends are New Zealand stream communities (representing 31 of the 71 stream
  food webs in our dataset), which experience unpredictable flooding and
  drying throughout the year and do not receive seasonally pulsed
  subsidies~\cite{Winterbourn1997,Winterbourn1981}. However, as this subset of
  webs is very tightly grouped in latitude ($44.64-46.41^{\circ}$S, within an
  overall range of $23-69.02^{\circ}$), it is unlikely that they have greatly
  influenced our results (see also \emph{Appendix S3}). Moreover, just as in
  highly-variable communities where said variation is more seasonal, New
  Zealand communities are dominated by ecological
  generalists~\cite{Winterbourn1997,Winterbourn1981} implying that they appear
  to fit the general pattern of streams worldwide.


  Importantly, while terrestrial communities are also strongly seasonal at
  high latitudes and can receive significant allochthonous
  inputs~\cite{Nakano2001}, terrestrial consumers tend to be
  morphologically specialised for feeding on particular prey~\cite{Liem1990}.
  This means that primarily gape-limited aquatic consumers tend to be more
  general across all types of aquatic habitats~\cite{Liem1990,Shurin2006}. The
  key to this explanation of the differences between freshwater and
  marine and estuarine  ecosystems is whether the former experience more
  severe seasonal variation. Although we are not aware of any study explicitly
  comparing seasonal variation in freshwater and saltwater or brackish food
  webs in a similar location, we believe that freshwater ecosystems are indeed
  likely to  experience more severe changes because of their small size. While
  oceans and estuaries certainly vary in terms of water temperature and
  nutrients over the course of a year~\cite{Baird1989}, these changes are likely to be slower
  and milder than in freshwaters because marine and estuarine communities are buffered 
  by being open to the ocean rather than isolated
  in the midst of a terrestrial matrix . Net primary productivity in particular is
  much more variable over the course of a year in non-marine communities~\cite{Field1998},
  suggesting that niche breadths may also be more variable over the course of the year.

  % This may be partly due to high-latitude 
  % species tending to switch between different seasonally-available 
  % prey~\cite{Magalhaes1993,Wilhelm1999,Isaac2012} while tropical
  % freshwater ecosystems may have more stable composition. 


  % The scaling of link density, generality and vulnerability with proportions
  % of species at different trophic levels, however, did not differ between
  % lakes and most other food webs. Indeed, there was comparatively little
  % variation in scaling with the trophic-level breakdown of a web across
  % ecosystem types. Nevertheless, the negative relationships between the
  % proportion of top predators in a web and link density and vulnerability were
  % weaker (i.e., the scaling exponent was more strongly negative) in low-latitude 
  % stream ecosystems. This may imply that top predators in high-latitude 
  % streams tend to be more generalist than their low-latitude
  % equivalents~\cite{Winemiller2008}, perhaps due to prey-switching during
  % seasonal food shortages~\cite{Magalhaes1993}. More generally, the lack of
  % correspondence between the scaling of food-web properties with species
  % richness and with proportions of species at each trophic level suggests that
  % the size and trophic breakdowns of a community can each provide different
  % information~\cite{Downing2002}.


\section*{Conclusion}

  Overall, our results were inconsistent with the latitude-niche breadth
  hypothesis in estuarine, marine, and terrestrial communities but consistent
  with the hypothesis of greater specialisation in the tropics in stream and
  lake food webs. This suggests that different mechanisms may structure food
  webs in different habitat types and that freshwater food webs in particular
  may be strongly affected by seasonal variation. In addition, different
  relationships between latitude and niche breadth in different habitat types
  goes some way towards explaining the equivocal support for the opposing
  hypotheses of narrower niches in the tropics~\cite{Vazquez2004} and broader
  niche space in the tropics~\cite{Davies2007}. Our study indicates that both 
  have merit but would appear to apply to different systems.

\section*{Data Accessibility}

  Food webs used in this study were retrieved from the University of Canberra's 
  GlobalWeb database (\cite{GlobalWeb}; \emph{www.globalwebdb.com}) and from
  two papers (\cite{Riede2011}; \emph{http://dx.doi.org/10.1111/j.1461-0248.2010.01568.x}
  and \cite{Dunne2013}; \emph{http://dx.doi.org/10.1371/journal.pbio.1001579}). Original sources for the
  food webs are given in \emph{Appendix S1}.

\section*{Competing interests}
  
  We have no competing interests.

\section*{Authors' contributions}
  
  ARC, DBS, and TNR designed the study, ARC collected
  published data, performed the analyses, and wrote the
  first draft. DBS and TNR substantially revised the article;
  all authors approved the final version.


\section*{Acknowledgements}
  
  We thank members of the Romanuk, Stouffer, and Tylianakis labs for their comments on
  the manuscript. We also thank Angus McIntosh for valuable discussion on stream food webs.

\section*{Funding}
  This research was supported by an NSERC USRA undergraduate scholarship and NSERC PGS-D 
  graduate scholarship (to ARC), a Marsden Fund Fast-Start grant (UOC-1101) and a 
  Rutherford Discovery Fellowship, both administered by the Royal Society of New Zealand 
  (to DBS), and an NSERC Discovery Grant (to TNR).


\newpage
\bibliographystyle{prsb}%Compile with vancouver.bst style file
\bibliography{abbreviated}% your .bib file(s)

% \end{spacing}
\newpage

\section*{Figure Captions}

\begin{figure}[h]
% \centerline{\includegraphics*[height=.5\textheight]{Figures/Other/conceptual_figure.eps}}
\caption{In A) we show the known scaling relationship between link density (links per species) and 
species richness. This scaling relationship is a power law and therefore linear in a log-log plot.
In B) we show two versions of the latitudinal-niche breadth hypothesis that have been proposed to
explain this gradient. Hypothesis 1 posits that greater environmental stability in the tropics will 
allow species to evolve narrower niches (indicated by parabolas) than those at the poles. Hypothesis 2 
suggests that species will have constant niche sizes over latitude but that greater primary productivity 
in the tropics creates a larger niche space such that each species still occupies a smaller proportion 
of the total niche space. These two hypotheses have different implications for the scaling of food-web 
properties such as the number of feeding links per species with species richness. C) If hypothesis 1 is 
true, then the exponent of the scaling relationship between link density and species richness should 
be larger towards the poles, where each additional species in the food-web will have a larger niche 
(i.e., more feeding links). If hypothesis 2 is true, then the exponent of this distribution should 
not vary significantly over latitude.}
\label{concept}
\end{figure}


\begin{figure}[h]
% \centerline{\includegraphics*[height=.5\textheight]{Figures/by_TL/scaling_with_S/proportions/S_fitlines_nonts.eps}}
\caption{Scaling relationships for re-scaled link density, generality, and vulnerability 
relative to the species richness of a food web. Link density, generality,
and vulnerability were each re-scaled to remove the effects of latitude and ecosystem
type. As these relationships take the form of power laws, we did this by dividing the food-web
property (e.g. link density) by species richness raised to an exponent including the 
effects of latitude and, where applicable, ecosystem type and the interaction between ecosystem
type and latitude. Note that in all cases estuarine food webs were treated as the baseline 
ecosystem type, but that at most two ecosystem types had interactions between ecosystem type and
latitude retained in the best-fit model (see \emph{Results} for specifics). For each relationship, 
we show the re-scaled values (white circles) as well as the overall scaling relationship using estuarine
ecosystems as a baseline (black line, N=196 food webs). For a figure with the uncorrected values,
see Fig.~\emph{S7}, \emph{Appendix S5}.}
\label{props_v_lat}
\end{figure}


\begin{figure}[h]
% \centerline{\includegraphics*[width=.85\textwidth]{Figures/by_TL/marginal/S_marginal_latitude_proportions.eps}}
\caption{Changes to the scaling of link density, generality, and vulnerability with species richness across ecosystem
types and over latitude. We show the estimated scaling exponent for species richness (black
line) with its 95\% confidence interval (in grey), based on N=196 empirical food webs.
Latitude is given in degrees from the equator regardless of direction.} \label{S} \end{figure}


% \begin{figure}[h]
% % \centerline{\includegraphics*[height=.65\textheight]{Figures/by_TL/marginal/BIT_marginal_latitude.eps}}
% \caption{Changes to the scaling of link density with the proportions of basal resources, intermediate
% consumers, or top predators in a food web across ecosystem types and over latitude. For each proportion
% we show the estimated scaling exponent (black line) with its 95\% confidence interval (in grey), based
% on N=166 empirical food webs. Latitude is given in degrees from the equator regardless of direction. 
% The behaviour of exponents for the scaling relationships of generality and vulnerability with each 
% proportion was very similar to those of the scaling relationships with link density, except for the 
% scaling of generality with the proportion of top predators where there was no effect of latitude on the 
% size of the exponent in any ecosystem type. See Figs.~\emph{S3-S5, Appendix 5} for all scaling relationships.}
% \label{BIT}
% \end{figure}


\end{document}
