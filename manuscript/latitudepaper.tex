\documentclass[12pt]{article}  
\usepackage{amsmath}
\usepackage{gensymb}
\usepackage{url}
\usepackage[dvips]{graphicx}
\usepackage{multirow}
\usepackage{geometry}
\usepackage{pdflscape}
%\usepackage{rotating}
% make Figure 1 etc bold
\usepackage[labelfont=bf]{caption}
\usepackage{setspace}

\usepackage[running]{lineno}

% let's get some nature formatted citations
%\usepackage{overcite}
\usepackage[round]{natbib}

% some cheats to reduce the need to type complicated bits and pieces
\newcommand{\expect}[1]{\left\langle #1 \right\rangle}
\newcommand{\etal}{\textit{et al.\ }}

\newcommand{\beginsupplement}{%
        \setcounter{table}{0}
        \renewcommand{\thetable}{S\arabic{table}}%
        \setcounter{figure}{0}
        \renewcommand{\thefigure}{S\arabic{figure}}%
     }

% the abstract formatting
\newenvironment{sciabstract}{%
\begin{quote} \bf}
{\end{quote}}
\renewcommand\refname{References}

% margin sizes`
\topmargin 0.0cm
\oddsidemargin 0.2cm

\textwidth 16cm 
\textheight 21cm
\footskip 1.0cm
\begin{document}


\title{Specialisation in food webs scales with species richness but not with latitude}
\author{Alyssa R. Cirtwill$^{1,2}$, Daniel B. Stouffer$^{1,3}$, Tamara N. Romanuk$^{2}$}
\date{\small$^1$School of Biological Sciences\\University of Canterbury\\
Private Bag 4800\\Christchurch 8140, New Zealand \\
\medskip$^2$Department of Biology\\
Life Science Centre, Dalhousie University\\1355 Oxford St., P0 BOX 15000\\
Halifax NS, B3H 4R2, Canada\\
\medskip$^3$Centre for Integrative Ecology\\School of Biological Sciences\\University of Canterbury\\
Private Bag 4800\\Christchurch 8140, New Zealand \\}

\maketitle
\baselineskip=8.5mm
 
\vspace{-0.3 in}

\begin{spacing}{1.0}
\section*{Abstract}

Several properties of food webs---the networks of feeding links between
species---are known to vary systematically with the species richness of the underlying
community.  Under the ``latitude-niche breadth hypothesis'', which predicts
that species in the tropics will tend to evolve narrower niches, one might
expect that these scaling relationships could also be affected by latitude. To
test this hypothesis, we analysed the scaling relationships between species
richness and average generality, vulnerability, and links per species across a
set of 163 empirical food webs.  We also investigated scaling relationships
between the three food-web properties and the proportions of the web made up
by basal resources, intermediate consumers, and top predators. While we
observed no effect of latitude on scaling relationships with species richness
in the estuarine, marine, and terrestrial food webs, there were strong effects
of latitude on scaling relationships in the freshwater food webs. In these
communities, the latitude-niche breadth hypothesis appears to hold true while
in other habitat types niche breadth appears to vary little. These contrasting
findings indicate that it may be more important to account for habitat than
latitude when exploring gradients in food-web structure.

\end{spacing}

\section*{Introduction} 

  %Food webs are a thing. (Mention gen, vul, LS.) %Food webs have strong scaling structure.

  Food webs --networks of feeding links between species-- have been used for
  several decades to summarise the structure of  ecological
  communities~\citep{Paine1966,Williams2000,Petchey2008} and to understand how
  that structure relates to environmental variables  such as habitat
  type~\citep{Briand1983,Shurin2006}, primary
  productivity~\citep{Townsend1998,Thompson2005a,Vermaat2009}, and
  climate~\citep{Petchey2010,Baiser2012}. The latter variables in turn have
  strong gradients over latitude, with productivity and temperature both being
  higher in the tropics while  climate is more variable at high
  latitudes~\citep{Field1998}. These variables affect both the resources
  available and species'
  metabolisms~\citep{White2007,OConnor2009,Hechinger2011,White2011}, and have
  been proposed as determinants of the strength of  interspecific
  interactions~\citep{Schemske2009,Lang2012,Schleuning2012}.  By modulating
  interactions between species, latitudinal gradients may also shape food-web
  structure. Indeed, these latitudinal environmental gradients have been put
  forward as potential drivers for the  latitudinal gradient in species
  richness, one of the most general and robust patterns in
  ecology~\citep{Kaufman1995,Macpherson2002,Schemske2009}.

  % These gradients, together with the strength and generality of associations between biological processes and environmental variables, suggest that variation in network structure 
  % may also be explained by ecological gradients \citep{Baiser2012}. 


  One proposed link between species richness and latitude is the ``latitude-niche breadth hypothesis''~\citep{Vazquez2004}. 
  This hypothesis predicts that decreased seasonality in the tropics should lead to more stable populations, which in
  turn should evolve smaller niches~\citep{Vazquez2004}. These narrow niches should therefore allow more species to 
  coexist in the tropics than at higher latitudes. Alternatively, the higher productivity of the tropics~\citep{Brown2004}
  may result in a broader niche space~\citep{Davies2007} which could also sustain greater biodiversity even if niche 
  sizes are globally similar. Although the assumptions of the latitude-niche breadth hypothesis are only equivocally 
  supported~\citep{Vazquez2004}, it remains a compelling potential mechanism for the latitudinal gradient in species 
  richness~\citep{Lappalainen2006,Krasnov2008,Slove2010}. 


  If the latitude-niche breadth hypothesis is correct, there should also be direct relationships between latitude and the
  degree of specialisation (i.e., the breadth of the Eltonian niche; ~\citealp{Elton1927}) of species within food webs.
  Attempts to unravel these effects, however,
  are complicated by known relationships between species richness and many other network properties~\citep{Riede2010}. 
  For example,
  narrower niches imply fewer links per species (i.e., greater specialisation) in the tropics (\citealp{Marra1997,Dyer2007};
  but see~\citealp{Schleuning2012}). However,
  average numbers of links per species tend to increase in larger food webs~\citep{Dunne2006,Riede2010}. This means
  that a latitudinal effect on specialisation may be obscured by a latitudinal gradient in species richness. If this is
  the case, it may still be possible to uncover effects of latitude on specialisation by examining the shape of the scaling 
  relationship between specialisation and species richness over changing latitude. Here, we use three measures of specialisation:
  mean links per species, mean generality (number of prey), and mean vulnerability (number of predators). By testing whether
  latitude affects the scaling of each property with species richness, we test effects of latitude on specialisation implied by
  the ``latitude-niche breadth hypothesis''.


\section*{Methods}

  \subsection*{Data Set} 

    We compiled a list of 166 empirical food webs from
    multiple sources (see \emph{Appendix S1} for web origins and selection
    criteria). We recorded study site latitude from the original source where
    possible or, where study sites were described but exact coordinates were not
    given, obtained estimated coordinates using Google Earth~\citep{GoogleEarth}.
    We then divided the species in each web into basal resources (those species with
    consumers but no prey), top predators (those species with prey but no predators),
    and intermediate consumers (species with both predators and prey, including cannibalistic
    species). 


    As the food webs in this dataset are derived from a variety of sources and were compiled over many decades, it
    is likely that they vary in their resolution and in the amount of sampling effort invested in their assembly.
    Many analyses of food-web structure attempt to reduce this variation by aggregating species with identical predator and prey
    sets to form ``trophic species'' webs ~\citep{Martinez1991,Dunne2004,Vermaat2009,Dunne2013}. As our study
    is concerned directly with the number of species at a particular latitude, however, we did not wish to ignore 
    species with identical sets of interactions. We therefore analysed both original and trophic-species versions
    of the dataset; in each case using the number of species and 
    links in each web to calculate the mean link density (number of links per species), mean generality 
    (number of prey per species), and mean vulnerability (number of predators per species) of the web. 
    The version of the dataset used did not qualitatively change the results, suggesting that
    the scaling relationships between species richness, other food-web properties, and latitude are very 
    similar whether or not redundant species are included. For simplicity, here we present only
    the results for the original webs.


  \subsection*{Scaling Relationships with S}

    The scaling relationship between link density ($Z$) and species richness ($S$)
    has been shown to be a power law~\citep{Riede2010} of the form 

    \begin{equation}
    \label{Power}
    Z_{i} \sim \alpha S_{i}^{\beta}  ,
    \end{equation}

    \noindent which is often re-expressed in logarithmic form 

    \begin{equation}
    \label{Loglog}
    \log{Z_{i}} \sim \log{\alpha} + \beta\log{S_{i}}  .
    \end{equation}

    \noindent As the two forms imply a statistical fit to the data to different error 
    distributions, neither of which has strong
    \emph{a priori} justification in our dataset, we followed the recommendations in~\citet{Xiao2011}
    to compare the two model formulations explicitly (see \emph{Appendix S2} for details). 
    The logarithmic form (equation~\ref{Loglog}) provided the better fit to the data,
    as did the logarithmic forms of similar models for the scaling of generality and vulnerability. 
    We therefore used and present logarithmic models throughout the rest of the study.


  \subsection*{Effect of Latitude on Scaling}

    % Logarithm of a sum is awful:   log(a+c) = log(a) + log(1+b^[log(c)-log(a)]) But can probably feed that into R...

    After determining the appropriate form of the scaling relationship, we
    then assessed the impact of latitude on the scaling relationships between
    species richness and  link density, generality, vulnerability. In the
    context of the scaling relationships above, note  that this implies that
    we wished to determine the effect of latitude on the scaling exponent
    $\beta$. We included a categorical variable for
    ecosystem type (stream, lake,  terrestrial, marine, or estuary), as well
    as interactions between food web type and latitude.


    We therefore began by considering models of the form

    \begin{equation}
    \label{PowerLat}
    Z_{i}=\alpha S_{i}^{\beta_{0}+\beta_{1}L_{i}+\beta_{2}E_{i}+\beta_{3}LE_{i}} + \epsilon_{i} ,
    \end{equation}

    \noindent where $S_{i}$ is the species richness of web $i$, $L_{i}$ its absolute
    latitude (degrees north or south  regardless of direction), $E_{i}$ is a categorical
    variable indicating the ecosystem type of network $i$ (comprising terms for stream, 
    marine, lake, and terrestrial networks with estuarine
    networks providing the intercept) and $\epsilon_{i}$ is a residual error term.
    The logarithmic formulation of this model is

    \begin{equation}
    \label{LogLat}
    \log{Z_{i}} = \log{\alpha}+\beta_{0}\log{S_{i}} + \beta_{1}L\log{S_{i}} +\beta_{2}E\log{S_{i}} +\beta_{3}LE\log{S_{i}} +\epsilon_{i} .
    \end{equation}

    We then simplified each model. To do this, we calculated the AIC of the
    maximal model as well as the AIC's of a suite of candidate simplified
    models identified using the R~\citep{R} function dredge from package
    MuMIn~\citep{MuMIn}.  Simplified models were obtained by systematically
    removing all possible combinations  of terms from the full model with the
    restriction that species richness was retained in all reduced models. The
    best-fitting model was then determined to be the model with the fewest
    terms  where $\Delta$AIC\textless2. If several models shared the fewest
    number of terms  and had $\Delta$AIC\textless2, the model with the lowest
    AIC in that set was chosen as the best-fit model.


  \section*{Scaling by Trophic Levels}

    We were also interested in the ways that scaling relationships with species richness might
    be affected by changes to the distribution of species among trophic levels. To
    that end, we repeated all of the above analyses replacing species richness by
    proportion of basal resources, proportion of intermediate consumers,
    or proportion of top predators. All model fitting and model
    simplification procedures were identical to those described for species
    richness.


\section*{Results}

  Link density, generality, and vulnerability each increased with increasing
  species richness ($\beta_0$=0.637, $p$\textless0.001; $\beta_0$=0.553,
  $p$\textless0.001; and $\beta_0$=0.637, $p$\textless0.001, respectively;
  Fig~\ref{props_v_lat}).For estuarine, marine, and terrestrial food webs the
  strength of this scaling did not vary with latitude
  ($\beta_{Latitude}$=-0.001, $p$=0.365 for link density;
  $\beta_{Latitude}$=-0.001, $p$=0.535 for generality; and
  $\beta_{Latitude}$=-0.001, $p$=0.366 for vulnerability; Fig.~\ref{S}). In
  lake food webs, however, the scaling of each property was stronger towards
  the poles ($\beta_{Latitude:Lake}$=0.004, $p$=0.019;
  $\beta_{Latitude:Lake}$=0.005, $p$=0.004; and
  $\beta_{Latitude:Lake}$=0.004, $p$=0.018, respectively). In stream food
  webs, generality increased more rapidly towards the poles
  ($\beta_{Latitude:Stream}$=0.007, $p$=0.001) while link density and
  vulnerability did not vary with latitude.


  Unlike species richness, only generality showed an overall increase with
  increasing proportions of basal resources in a web ($\beta_0$=0.019,
  $p$=0.859; $\beta_0$=0.383, $p$=0.001; and $\beta_0$=0.019, $p$=0.855 for
  link density, generality, and vulnerability, respectively;
  Fig.~\ref{props_v_lat}). In each case, however, there were substantial
  effects of latitude on the scaling relationships. Thus, the scaling
  exponents for link density and vulnerability were negative for estuarine,
  lake, and stream communities and became more negative towards the poles
  ($\beta_{Latitude}$=-0.008, $p$\textless0.001 and $\beta_{Latitude}$=-0.008,
  $p$\textless0.001, respectively; Fig.~\ref{BIT}). For the scaling of link
  density and vulnerability in marine and terrestrial communities, the overall
  effect of latitude on scaling was positive ($\beta_{Latitude:Marine}$=0.011,
  $p$=0.020; $\beta_{Latitude:Terrestrial}$=0.016, $p$\textless0.001 and
  $\beta_{Latitude:Marine}$=0.011, $p$=0.020;
  $\beta_{Latitude:Terrestrial}$=0.016, $p$\textless0.001, respectively) such
  that there was weak scaling at most latitudes. In contrast, the scaling
  exponent of generality with the proportion of basal resources was positive
  near the equator and negative near the poles in estuarine, lake, and stream
  communities ($\beta_{Latitude}$=-0.010, $p$\textless0.001) while there was
  very little change in the scaling exponents for marine and terrestrial
  communities ($\beta_{Latitude:Marine}$=0.013, $p$=0.010;
  $\beta_{Latitude:Terrestrial}$=0.019, $p$\textless0.001).


  As with species richness, link density, generality, and vulnerability
  generally increased with the proportion of intermediate consumers in a web
  ($\beta_0$=0.771, $p$\textless0.001; $\beta_0$=0.349, $p$=0.235; and
  $\beta_0$=0.771, $p$\textless0.001 respectively; Fig.~\ref{props_v_lat}). 
  The best-fit models for link density and vulnerability did not include
  any effect of latitude. In contrast, the strength of the scaling relationship
  for generality increased weakly towards the poles in estuarine, marine, and
  terrestrial food webs ($\beta_{Latitude}$=0.008, $p$=0.204) but decreased 
  towards the poles in lakes and streams ($\beta_{Latitude:Lake}$=-0.021, $p$=0.018
  and $\beta_{Latitude:Stream}$=-0.035, $p$=0.001; Fig~\ref{BIT}).


  In contrast to the generally positive scaling relationships above, link
  density, generality, and vulnerability decreased as the proportion of top
  predators increased ($\beta_0$=-0.532, $p$\textless0.001; $\beta_0$=-0.454,
  $p$\textless0.001; and $\beta_0$=-0.532, $p$\textless0.001 respectively;
  Fig~\ref{props_v_lat}). For estuarine, lake, and marine food webs the
  strength of this scaling did not vary significantly with latitude
  ($\beta_{Latitude}$=-0.001, $p$=0.481 for link density;
  $\beta_{Latitude}$=-0.001, $p$=0.472 for vulnerability).
  In stream and terrestrial food webs, however, scaling was more strongly negative in the 
  tropics and near zero near the poles ($\beta_{Latitude:Stream}$=0.012, $p$=0.001 
  and $\beta_{Latitude:Terrestrial}$=0.009, $p$=0.015 for link density;
  $\beta_{Latitude:Stream}$=0.012, $p$=0.001 and $\beta_{Latitude:Terrestrial}$=0.009, $p$=0.015
  for vulnerability; Fig.~\ref{BIT}. The best-fit model for generality did not include any terms
  for latitude in any ecosystem type.


\section*{Discussion}

  The tendency of food-web structure to exhibit scaling relationships with
  species richness has been well-established~\citep{Dunne2004,Riede2010}. As
  species richness in particular is also known to vary systematically over
  latitude~\citep{Kaufman1995,Macpherson2002,Hillebrand2004,Schemske2009},
  intuitively one might suspect that any relationship between food-web
  properties such as generality might be due to the latitudinal gradient in
  species richness. In this dataset, however, we did not find overall
  latitudinal gradients in species richness, links per species,generality,
  vulnerability, or the proportions of food webs accounted for by basal
  resources, intermediate consumers, and top predators, except in lake and
  stream food webs where the proportions of top predators tended to decrease
  towards the poles  (see \emph{Appendix S3} for details).


  The lack of a latitudinal gradient in species richness in this dataset
  contrasts strongly with other
  studies~\citep{Kaufman1995,Macpherson2002,Hillebrand2004,Schemske2009}. As
  numbers of species and links included in a food web vary strongly with
  sampling effort as well as with the underlying diversity of the study area,
  it is possible that the lack of latitudinal trends here is a result of
  researchers tending to expend similar amounts of sampling effort across
  studies. This could result in food webs describing species-rich tropical
  communities omitting more species and links than species-poor arctic
  communities. In addition, it is worth noting that gradients in species
  richness are generally measured for a single taxonomic group at a
  time~\citep{Kaufman1995,Macpherson2002,Hillebrand2004,Schemske2009}. It is possible
  that these taxa are not well-represented in our food webs and that the
  dominant taxa in them do not have an underlying latitudinal gradient in
  richness.In either case, the lack of a strong association between species
  richness and latitude in any ecosystem type means that any effect of
  latitude on other scaling relationships is not being driven by an underlying
  latitudinal gradient in species richness. The lack of confounding effects of
  latitude allows us to more clearly assess effects of latitude on scaling
  with species richness and proportions of species in different trophic
  levels.


  Scaling of links per species, generality, and vulnerability with species
  richness varied strongly across ecosystem types. In estuarine, marine, and
  terrestrial food webs scaling of each property varied little with latitude.
  This is consistent with the idea that species' niche breadths do not vary
  systematically with temperature and productivity but that the niche space
  might be larger in species-rich communities~\citep{Davies2007}. Rather than
  niche space depending on temperature and productivity, it may be that species
  diversity itself affects the niche space available to species (although climate
  may affect speciation rates and therefore the diversity in a region~\citep{Currie2004}). 
  For example, as the plant diversity of a community increases both the 
  variety of food available to herbivores and the structural variety of the habitat will also increase.


  Unlike other ecosystem types, the scaling of generality in lake and stream
  food webs was stronger (i.e.,generality increased more steeply with
  increasing species richness) in higher-latitude food webs. In lake food webs,
  this trend was echoed in the scaling relationships between species richness
  and links per species and vulnerability. This means that species in tropical
  freshwater communities gain fewer additional feeding links per additional
  species in the web and that species in tropical lakes also gain fewer
  predators, and fewer links in general, per additional species than species
  in high-latitude lakes. These trends are consistent with the hypothesis that
  greater stability in the tropics leads to narrower niches~\citep{Vazquez2004}
  and a higher proportion of specialists. This may be partly due to high-latitude 
  species tending to switch between different seasonally-available 
  prey~\citep{Magalhaes1993,Wilhelm1999,Isaac2012} while tropical
  freshwater ecosystems may have more stable composition. 


  The scaling of link density, generality and vulnerability with proportions
  of species at different trophic levels, however, did not differ between
  lakes and most other food webs. Indeed, there was comparatively little
  variation in scaling with the trophic-level breakdown of a web across
  ecosystem types. Nevertheless, the negative relationships between the
  proportion of top predators in a web and link density and vulnerability were
  weaker (i.e., the scaling exponent was more strongly negative) in low-latitude 
  stream ecosystems. This may imply that top predators in high-latitude 
  streams tend to be more generalist than their low-latitude
  equivalents~\citep{Winemiller2008}, perhaps due to prey-switching during
  seasonal food shortages~\citep{Magalhaes1993}. More generally, the lack of
  correspondence between the scaling of food-web properties with species
  richness and with proportions of species at each trophic level suggests that
  the size and trophic breakdowns of a community can each provide different
  information~\citep{Downing2002}.

\section*{Conclusion}

  Overall, our results were inconsistent with the latitude-niche breadth
  hypothesis in estuarine, marine, and terrestrial communities but consistent
  with the hypothesis of greater specialisation in the tropics in stream and
  lake food webs. This suggests that different mechanisms may structure food
  webs in different habitat types and that freshwater food webs in particular
  may be strongly affected by seasonal variation. In addition, different
  relationships between latitude and niche breadth in different habitat types
  goes some way towards explaining the equivocal support for the opposing
  hypotheses of narrower niches in the tropics~\citep{Vazquez2004} and broader
  niche space in the tropics~\citep{Davies2007}; both have merit but appear to
  apply to different systems.

\section*{Conclusion}
  
  We thank members of the Romanuk, Stouffer, and Tylianakis labs for their comments on
  the manuscript.
  This research was supported by an NSERC USRA undergraduate scholarship and NSERC PGS-D 
  graduate scholarship (to ARC).

\newpage
\bibliographystyle{jae}%Compile with jae.bst style file
\bibliography{noISN}% your .bib file(s)

% \end{spacing}
\newpage

\section*{Figures}
\begin{figure}[h]
% \centerline{\includegraphics*[width=.85\textwidth]{Figures/by_TL/scaling_with_S/proportions/fitlines_nonts.eps}}
\caption{Scaling relationships for re-scaled link density, generality, and vulnerability 
relative to species richness and the proportions of basal resources (\% Basal), intermediate 
consumers (\% Intermediate), or top predators (\% Top) in a food web. Link density, generality,
and vulnerability were each re-scaled to remove the effects of latitude and ecosystem
type. As these relationships take the form of power laws, we did this by dividing the food-web
property (e.g. link density) by the predictor (e.g. \% Basal) raised to an exponent including the 
effects of latitude and, where applicable, ecosystem type and the interaction between ecosystem
type and latitude. Note that in all cases estuarine food webs were used as the baseline 
ecosystem type, but that at most two ecosystem types had interactions between ecosystem type and
latitude retained in the best-fit model (see \emph{results} for specifics). For each relationship, 
we show the re-scaled values (white circles) as well as the overall scaling relationship using estuarine
ecosystems as a baseline (black line, N=166 food webs). For a figure with the uncorrected values,
see Fig.~\emph{S1}, \emph{Appendix S4}.}
\label{props_v_lat}
\end{figure}


\begin{figure}[h]
% \centerline{\includegraphics*[height=.65\textheight]{Figures/by_TL/marginal/justS_marginal_latitude.eps}}
\caption{Changes to the scaling of link density with species richness across ecosystem
types and over latitude. We show the estimated scaling exponent for species richness (black
line) with its 95\% confidence interval (in grey), based on N=166 empirical food webs.
Latitude is given in degrees from the equator regardless of direction. The
behaviour of exponents for the scaling of generality and vulnerability with
species richness was very  similar to those for link density, except for the
scaling of generality in streams where  the size of the exponent increased
towards the poles. See Fig.~\emph{S2, Appendix 5} for  all scaling
relationships.} \label{S} \end{figure}


\begin{figure}[!H]
% \centerline{\includegraphics*[height=.65\textheight]{Figures/by_TL/marginal/BIT_marginal_latitude.eps}}
\caption{Changes to the scaling of link density with the proportions of basal resources, intermediate
consumers, or top predators in a food web across ecosystem types and over latitude. For each proportion
we show the estimated scaling exponent (black line) with its 95\% confidence interval (in grey), based
on N=166 empirical food webs. Latitude is given in degrees from the equator regardless of direction. 
The behaviour of exponents for the scaling relationships of generality and vulnerability with each 
proportion was very similar to those of the scaling relationships with link density, except for the 
scaling of generality with the proportion of top predators where there was no effect of latitude on the 
size of the exponent in any ecosystem type. See Figs.~\emph{S3-S5, Appendix 5} for all scaling relationships.}
\label{BIT}
\end{figure}


\end{document}
