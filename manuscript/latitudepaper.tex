\section*{Introduction}

[[Shorten this up]]


Creating networks describing ecologically important interactions (e.g. food
webs composed of feeding links) provides a powerful way to explore differences
in community structure as networks provide a rich set of metrics that can be
used to compare structure across ecosystems (Williams & Martinez, 2000, Olesen
& Jordano, 2002, Stouffer et al., 2005, Reide et al., 2010). Such comparisons
have revealed that the structure of biological networks is generally tightly
constrained, particularly by relationships between species richness or connectance
(the proportion of possible links that are realized)
which suggest scale-depence (Riede et al., 2010), which here refers to the
strong dependence of network properties, such as fractions of
species in different trophic groups, on the species richness or connectance of
the networks. Therefore it is likely that much variation in network structure
can be explained by variation in these two properties (Vermaat et al., 2009).


Species-richness in turn depends strongly on latitude via effects of
productivity and temperature (Brown et al., 2004, Cardillo et al., 2005,
Thompson & Townsend, 2005, Davies et al., 2007).  The strength and generality
of this pattern has generated a range of explanatory models (Whittaker et al.,
2001). Higher primary productivity has been shown emprically to affect species
richness at small and large scales  (Thompson & Townsend, 2005; Apellaniz
\emph{et al.}, 2012). Mechanistically, higher solar input (and therefor
higher productivity) may cause higher species richness (Brown et al., 2004,
Davies et al., 2007) by broadening the niche space, lessening competition
(Davies et al., 2007). Some have proposed that species' niches are generally
narrower in the tropics, and this greater specialization is what permits
higher species richness. This hypothesis is only equivocally supported
(Vazquez & Stevens, 2004) but even if niche sizes are similar at all latitudes
an increase in productivity, and by extension niche space, would be expected
to reduce competition and permit diversification. Adaptive radiation in the
tropics may also be facilitated (Cardillo et al., 2005) by the greater genetic
diversity btween populations, as well as between species, that has been
observed (Eo \emph{et al.}, 2008).


In addition to these potential effects of the changing magnitude of solar
radiation and productivity across latitudes,   both temperature (Field et al.,
1998) and primary productivity (Paine, 1966, Huston & Wolverton, 2011) are
more \emph{stable} at the equator than at higher latitudes. This lack of
seasonality in temperature and productivity is associated with lower
extinction rates, and therefore is also a potential mechanism for greater
species richness at lower latitudes (Krug \emph{et al.}, 2009). Less seasonality may also lead to greater specialization, as species do not need to switch food sources througout the year.[but can I support it?]



Due to the strength and generality of associations between biological
processes and environmental variables, variation in network structure may also
be explained by ecological gradients (Baiser et al., 2011). Predictable
changes in species richness (Davies et al., 2007, Schemske et al., 2009),
metabolic rates (Stegen et al., 2012), and rates and intensities of species
interactions (Marquis, 2005, Schemske et al., 2009) along latitudinal
gradients are among the most general and robust ecological patterns that have
been identified.




These explanations for latitudinal gradients in species richness  suggest that
other network properties should also vary over temperature and/or
productivity, independent of species richness. Temperature affects the rate at
which chemical reactions proceed, including metabolic reactions in organisms
(Gillooly et al., 2001). A faster metabolism implies greater energy demands on
an organism and therefore higher rates of herbivory, predation, and
decomposition (Brown et al., 2004, Schemske et al., 2009). This greater
activity tends to result in interactions with more species (Waser et al.,
1996, Bascompte & Jordano, 2007), increasing the number of links per species
and connectance at higher temperatures. Reduced competition due to either
narrower niches or a broader niche space (Davies et al., 2007) could also lead
to higher numbers of links per species and connectance, as species are free to
pursue links with many partners rather than being restricted to the most
profitable links. Specialist consumers tend to be more efficient at ingesting
and assimilating prey (Montoya et al., 2006), suggesting that specialization
is also likely at higher latitudes where productivity is low for much of the
year (Field et al., 1998). Higher productivity is expected to result in longer
food chains by relaxing the upper limits set by the inefficiency of energy
transfer across trophic levels (Brown et al., 2004). This should produce a web
with a lower proportion of basal species and more intermediate and top
predators in food webs. Bitrophic networks should also show more higher-
trophic-level and animal species, since they should experience the same
relaxed productivity in their feeding links, even if those links are not
explicitly considered in the networks used here.


To test whether variability in the network structure of ecological networks
can be reduced to a few ecologically meaningful dimensions, and whether these
dimensions are associated with potential environmental drivers, we preformed
principal components analyses on 97 multitrophic predator-prey (n=52) and
bitrophic parasite-host and mutualist networks (n=45) that range across a
latitudinal gradient from 52.35°S to 81.82°N. We ran additional analyses for
network properties that had been standardized for differences in species
richness and connectance to examine scale-independent variation in network
structure. We hypothesized that, in the analyses of scale-dependent network
properties, species richness and connectance would explain much of the
variation in network structure and that, because of this, temperature and
productivity would be strongly associated with major axes of variation.
Networks were expected to contain more species at higher trophic levels and
feature higher connectance and links per species closer to the equator. We
expected these trends to continue in the standardized network properties, as
many of the explanations for increasing species richness at lower latitudes
should also affect other network properties directly.


\section*{Methods}

We compiled a list of XX published food webs from [database details]. We
included only food webs with a clearly defined study site that did not cover
more than 10 degrees of latitude. Source and sink webs were also excluded as
they are likely to have different food web properties than webs attempting to
describe the entire community of a site. Also excluded were webs with poor
resolution, describing highly modified or artificial ecosystems, or including
humans.


We then converted each web to a trophic-species version by aggregating species
with the same predators and prey. This helps to reduce variation in resolution
across different studies. We then classified them both according to latitude
and ecosystem type (terrestrial, freshwater, marine, estuary). We analysed XX
food-web properties for each web (Table 1).


In our analysis, we first addressed whether the diversity or complexity of our
published food webs varies over latitude (expressed as degrees away from the
equator regardless of direction) or factorial ecosystem type. We also included
year published as in independent variable to control for changing standards
for food web studies over time. We fit independent generalized linear models
for each of the logarithms of species richness, mean links per species, and
connectance.


Second, we analysed the scaling relationships between the remaining XX food-
web parameters and connectance and species richness. We used a series of
generalized linear models with a food-web parameter as the dependent variable
and latitude, ecosystem type, and their interactions with connectance and
species richness as independent variables. (parameter ~
latitude*ecotype*(log10(connectance)+log10(species richness))). For food web
properties that were proportions, we used general linear models with a
binomial distribution. All other food web properties were log-transformed.
When modelling the standard deviation of generality, this
required the removal of 5 webs, all with 12 or fewer species, which had
standard deviations of generality of 0. These full models were then
systematically reduced using R function dredge \cite{} in package MuMIn
\cite{} in order to isolate the model with the highest AIC. These best-fitting models were 
then examined in detail.


Significant slopes in the reduced models were taken to indicate scaling
relationships with species richness or connectance. Significant interactions
indicated differences in the scaling relationships over latitude. Different
intercepts indicated differences between ecosystem types.


\section*{Results}

In this data set, none of the best fit models for species richness, connectance, or links per species included an 
effect of latitude, although all three properties varied across ecosystem types. [[Table or something]]


The best-fit models for most food-web properties exhibited scaling relationships with species-richness and/or 
connectance. Notable exceptions were the proportions of basal resources, herbivores, and omnivores, which varied only 
across year published; and the proportions of top predators and intermediate consumers, which varied with connectance 
and species richness and connectance (respectively) although not significantly.


For the food-web properties that did scale with species-richness or connectance, most of these relationships were 
constant over latitude. Latitude terms were not included in the best-fit models for vulnerability, SD of generality or 
SD of links. SD of vulnerability showed a weak negative relationship with latitude ($\Beta$=-0.001, $p$=0.056), 
and path length increased with increasing 
latitude ($\Beta$ = 0.005, $p$=0.045) but neither latitude effect affected any scaling with species richness or 
connectance.


The scaling rules of mean and maximum trophic level and generality varied over latitude. 
Maximum trophic level increased with 
latitude ($\Beta=0.011$, $p$=0.011), but this relationship was weakened as species richness increased ($\beta$=-0.008, $
p$=0.011). Similarly, maximum trophic level increased with species richness ($\beta$=0.430, $p$=0.003) except at very 
high latitudes. Maximum trophic level decreased with connectance for estuarine food webs ($\beta$=-0.458, $p$=0.021), 
varied little in lakes ($\beta$=-0.127, $p$=0.192), and increased with connectance in marine, stream, and terrestrial 
food webs ($\beta$=0.219, $p$=0.005; $\beta$=0.249, $p$<0.001; and $\beta$=0.337, $p$=0.002, respectively). Taken 
together, these effects resulted in an increase in maximum trophic level with latitude at low species richnesses across 
all ecotypes and decreasing maximum trophic level with latitude at high species richness across all ecotypes 
(Fig. \ref{}. At moderate levels of species richness, trophic level increased with latitude for estuarine webs, showed 
little change for lake webs, and decreased for all other ecotypes. 


Qualitatively similar trends were observed for mean trophic level. Mean trophic level increased with latitude 
($\beta$=0.0109, $p$<0.001) and with species richness ($\beta$=0.616, $p$<0.001), but both trends were weakened by an 
interaction between latitude and species-richness ($\beta$=-0.008, $p$<0.001). The best-fitting model for mean trophic 
level did not include a connectance term. The increase in mean trophic level with species richness was highest for 
estuarine food webs ($\beta$=0.616, $p$<0.001) and lower for lakes ($\beta$=-0.608, $p$=0.232), streams 
($\beta$=0.238, $p$<0.001), terrestrial food webs ($\beta$=0.217, $p$=0.002), and marine food webs 
($\beta$=0.133, $p$<0.001). 


Generality showed the opposite trend, decreasing with increasing latitude ($\beta$=-0.017, $p$=0.003). This decrease 
was counteracted by a positive interaction between latitude and species richness ($\beta$=0.013, $p$=0.003), with the net result that generality increased with latitude for most ecotypes and species richnesses (Fig. \ref{}).
The main effect of species richness on generality varied across ecotypes. Generality decreased with increasing species 
richness for estuarine webs ($\beta$=-0.058, $p$=0.003) and increased for lakes ($\beta$=0.220, $p$=0.245), 
streams ($\beta$=0.543, $p$<0.001), terrestrial ($\beta$=0.603, $p$=0.005) and marine webs ($\beta$=0.761, $p$<0.001). 
Generality increased with connectance equally for all ecotypes ($\beta$=0.875, $p$<0.001).


\section*{Discussion}

The tendency of food-web structure to exhibit scaling relationships with both species richness and connectance has been 
well-established \citep{,,,,}. As species richness in particular is also know to vary systematically over latitude 
\citep{,,,,}, intuitively one might suspect that the scaling relationships of other food web properties might also vary
over latitude \citep{}. However, we found that only four of XX food web properties varied significantly over latitude, 
and that only the scaling relationships of trophic level and generality varied from the tropics to the poles.


A study of plant-animal mutualistic networks found the opposite trend to that observed here, that generality decreased 
with increasing latitude while increasing with increasing plant diversity \citep{Schleuning2012}. Unfortunately this 
study did not examine changes in scaling relationships over latitude... Regardless, this may indicate that food webs 
and mutualistic networks vary differently over latitudinal gradients of productivity, temperature, etc. 
\citet{Schleuning2012} propose that the latitudinal specialization gradient in mutualists derives from the gradient in 
plant diversity, as high resource diversity may lead to more generalized consumers. As species richness in the food 
webs in our dataset did not vary over latitude, it seems unlikely that this is the mechanism driving increased 
generality in high-latitude food webs. As we also observed an increase in both mean and maximum trophic level with 
latitude, it is possible that high-trophic level species are also more apt to be generalists. Generalism may be an 
important strategy for high-level consumers as their prey may be scarce or seasonal \citep{}.


[[other scaling paper]]


The overall consistency of scaling laws over latitude is remarkable and suggests that food webs worldwide assemble 
along similar ``rules'' regardless of temperature, productivity, or other latitudinal gradients. However, the changing
scaling laws for generality and trophic level suggest that there are also key differences in the ways that food webs 
are configured in the tropics and at higher latitudes. Importantly, the opposing trends for generality in food webs 
and mutualistic networks may suggest that latitude imposes different pressures on mutualistic versus antagonistic 
interactions between species. [[Need a punchier conclusion]]





