\section*{Introduction}


Creating networks describing ecologically important interactions (e.g. food
webs composed of feeding links) provides a powerful way to explore differences
in community structure as networks provide a rich set of metrics that can be
used to compare structure across ecosystems (Williams & Martinez, 2000, Olesen
& Jordano, 2002, Stouffer et al., 2005, Reide et al., 2010). Such comparisons
have revealed that the structure of biological networks is generally tightly
constrained (Riede et al., 2010) and that it may be possible to reduce the
plethora of network variables to a few major descriptive dimensions (Vermaat
et al., 2009). In particular, characteristic changes in structural properties
with species richness and connectance (number of links/number of species2)
have suggested scale-dependence (Riede et al., 2010), which here refers to the
strong dependence of network properties, such as fractions of
species in different trophic groups, on the species richness or connectance of
the networks. Therefore it is likely that much variation in network structure
can be explained by variation in these two properties (Vermaat et al., 2009).


Species-richness in turn depends strongly on latitude via effects of
productivity and temperature (Brown et al., 2004, Cardillo et al., 2005,
Thompson & Townsend, 2005, Davies et al., 2007).  The strength and generality
of this pattern has generated a range of explanatory models (Whittaker et al.,
2001). Higher primary productivity has been shown emprically to affect species
richness at small and large scales  (Thompson & Townsend, 2005; Apellaniz
\emph{et al.}, 2012). Mechanistically, higher solar input (and therefor
higher productivity) may cause higher species richness (Brown et al., 2004,
Davies et al., 2007) by broadening the niche space, lessening competition
(Davies et al., 2007). Some have proposed that species' niches are generally
narrower in the tropics, and this greater specialization is what permits
higher species richness. This hypothesis is only equivocally supported
(Vazquez & Stevens, 2004) but even if niche sizes are similar at all latitudes
an increase in productivity, and by extension niche space, would be expected
to reduce competition and permit diversification. Adaptive radiation in the
tropics may also be facilitated (Cardillo et al., 2005) by the greater genetic
diversity btween populations, as well as between species, that has been
observed (Eo \emph{et al.}, 2008).


In addition to these potential effects of the changing magnitude of solar
radiation and productivity across latitudes,   both temperature (Field et al.,
1998) and primary productivity (Paine, 1966, Huston & Wolverton, 2011) are
more \emph{stable} at the equator than at higher latitudes. This lack of
seasonality in temperature and productivity is associated with lower
extinction rates, and therefore is also a potential mechanism for greater
species richness at lower latitudes (Krug \emph{et al.}, 2009). Less seasonality may also lead to greater specialization, as species do not need to switch food sources througout the year.[but can I support it?]



Due to the strength and generality of associations between biological
processes and environmental variables, variation in network structure may also
be explained by ecological gradients (Baiser et al., 2011). Predictable
changes in species richness (Davies et al., 2007, Schemske et al., 2009),
metabolic rates (Stegen et al., 2012), and rates and intensities of species
interactions (Marquis, 2005, Schemske et al., 2009) along latitudinal
gradients are among the most general and robust ecological patterns that have
been identified.




These explanations for latitudinal gradients in species richness  suggest that
other network properties should also vary over temperature and/or
productivity, independent of species richness. Temperature affects the rate at
which chemical reactions proceed, including metabolic reactions in organisms
(Gillooly et al., 2001). A faster metabolism implies greater energy demands on
an organism and therefore higher rates of herbivory, predation, and
decomposition (Brown et al., 2004, Schemske et al., 2009). This greater
activity tends to result in interactions with more species (Waser et al.,
1996, Bascompte & Jordano, 2007), increasing the number of links per species
and connectance at higher temperatures. Reduced competition due to either
narrower niches or a broader niche space (Davies et al., 2007) could also lead
to higher numbers of links per species and connectance, as species are free to
pursue links with many partners rather than being restricted to the most
profitable links. Specialist consumers tend to be more efficient at ingesting
and assimilating prey (Montoya et al., 2006), suggesting that specialization
is also likely at higher latitudes where productivity is low for much of the
year (Field et al., 1998). Higher productivity is expected to result in longer
food chains by relaxing the upper limits set by the inefficiency of energy
transfer across trophic levels (Brown et al., 2004). This should produce a web
with a lower proportion of basal species and more intermediate and top
predators in food webs. Bitrophic networks should also show more higher-
trophic-level and animal species, since they should experience the same
relaxed productivity in their feeding links, even if those links are not
explicitly considered in the networks used here.


To test whether variability in the network structure of ecological networks
can be reduced to a few ecologically meaningful dimensions, and whether these
dimensions are associated with potential environmental drivers, we preformed
principal components analyses on 97 multitrophic predator-prey (n=52) and
bitrophic parasite-host and mutualist networks (n=45) that range across a
latitudinal gradient from 52.35°S to 81.82°N. We ran additional analyses for
network properties that had been standardized for differences in species
richness and connectance to examine scale-independent variation in network
structure. We hypothesized that, in the analyses of scale-dependent network
properties, species richness and connectance would explain much of the
variation in network structure and that, because of this, temperature and
productivity would be strongly associated with major axes of variation.
Networks were expected to contain more species at higher trophic levels and
feature higher connectance and links per species closer to the equator. We
expected these trends to continue in the standardized network properties, as
many of the explanations for increasing species richness at lower latitudes
should also affect other network properties directly.


\section*{Methods}

We compiled a list of XX published food webs from [database details]. We
included only food webs with a clearly defined study site that did not cover
more than 10 degrees of latitude. Source and sink webs were also excluded as
they are likely to have different food web properties than webs attempting to
describe the entire community of a site. Also excluded were webs with poor
resolution, describing highly modified or artificial ecosystems, or including
humans.


We then converted each web to a trophic-species version by aggregating species
with the same predators and prey. This helps to reduce variation in resolution
across different studies. We then classified them both according to latitude
and ecosystem type (terrestrial, freshwater, marine, estuary). We analysed XX
food-web properties for each web (Table 1).


In our analysis, we first addressed whether the diversity or complexity of our
published food webs varies over latitude (expressed as degrees away from the
equator regardless of direction) or factorial ecosystem type. We also included
year published as in independent variable to control for changing standards
for food web studies over time. We fit independent generalized linear models
for each of the logarithms of species richness, mean links per species, and
connectance.


Second, we analysed the scaling relationships between the remaining XX food-
web parameters and connectance and species richness. We used a series of
generalized linear models with a food-web parameter as the dependent variable
and latitude, ecosystem type, and their interactions with connectance and
species richness as independent variables. (parameter ~
latitude*ecotype*(log10(connectance)+log10(species richness))). For food web
properties that were proportions, we used general linear models with a
binomial distribution. All other food web properties were log-transformed
except for motif frequencies, where many webs had zero frequencies of at least
some motifs. When modelling the standard deviation of generality, this
required the removal of 5 webs, all with 12 or fewer species, which had
standard deviations of generality of 0. These full models were then
systematically reduced using R function dredge \cite{} in package MuMIn
\cite{} in order to isolate the model with the highest AIC.


Significant slopes in the reduced models were taken to indicate scaling
relationships with species richness or connectance. Significant interactions
indicated differences in the scaling relationships over latitude. Different
intercepts indicated differences between ecosystem types.


\section*{Results}

Most food-web properties scaled with species-richness and/or connectance, and many of these relationships changed across latitude.
In the best-fit models, the percentages of species that were top predators, intermediate consumers, basal resources, herbivores, and omnivores did not significantly covary with either species-richness or connectance.
Mean trophic level scaled with species richness, mean shortest path scaled with connectance, and all other properties scaled with both species richness and connectance.


The strength of the scaling relationship between mean trophic level and species richness decreased with increasing latitude ($\beta=-0.008$,$p<0.001$). In addition, different ecosystem types had both different intercept trophic levels (except for lakes, which did not differ from estuaries) and different scaling relationships with species-richness (again except for lakes). Scaling of mean trophic level with species richness was strongest for estuarine food webs ($\beta=0.616$,$p<0.001$; intercept) and was progressively weaker for lake webs, stream webs, terrestrial webs, and marine webs.


Mean shortest path length decreased with increasing connectance ($\beta=-1.269$,$p=0.016$), and this relationship did not change with latitude. Although path length increased with latitude ($\beta=0.005$,$p=0.045$), the best-fitting model did not include a change to the scaling relationship with latitude. Different ecosystems, however, both had different intercept path lengths and different scaling relationships with connectance.


Mean and SD of vulnerability and SD of generality scaled with both species richness and connectance, but none of these scaling relationships varied either with latitude or with ecosystem type.


SD of links per species scaled with both species richness and connectance. Neither scaling relationship varied with latitude, and only the relationship with connectance varied across ecosystem types. LinkSD increased with increasing connectance ($\beta=0.999$,$p<0.001$), and this relationship was strongest for terrestrial and estuarine webs and weaker for marine, lake, and stream webs.


Similarly, mean links per species did not vary with latitude but did across ecosystem types. Streams had the highest mean L/S, followed by lakes, marine webs, terrestrial webs, and estuaries.


Gen, Max TL. 





The scaling of motif frequencies with connectance varied across ecotypes for
most motifs (specifically motifs 102, 108, 110, 14, 238, 46, 74, and 78) with
both intercept and slope effects. Scaling relationships with species-richness
did not change. Latitude had not effect on the scaling relationships of these
motifs. For three motifs (12, 36, and 6), scaling was not associated with
either latitude or ecotype. The frequency of the remaining motif (108) did not
scale with species richness. The scaling of this motif's frequency with
connectance did vary with latitude, although this relationship was not
significant.  Overall, therefore, we conclude that motif structure does not
change across latitude.


[Table of results]






