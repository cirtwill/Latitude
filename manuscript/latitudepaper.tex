\documentclass[12pt]{article}  
\usepackage{amsmath}
\usepackage{gensymb}
\usepackage{url}
\usepackage[dvips]{graphicx}
\usepackage{multirow}
\usepackage{geometry}
\usepackage{pdflscape}
%\usepackage{rotating}
% make Figure 1 etc bold
\usepackage[labelfont=bf]{caption}
\usepackage{setspace}

\usepackage[running]{lineno}

% let's get some nature formatted citations
%\usepackage{overcite}
\usepackage[round]{natbib}

% some cheats to reduce the need to type complicated bits and pieces
\newcommand{\expect}[1]{\left\langle #1 \right\rangle}
\newcommand{\etal}{\textit{et al.\ }}

\newcommand{\beginsupplement}{%
        \setcounter{table}{0}
        \renewcommand{\thetable}{S\arabic{table}}%
        \setcounter{figure}{0}
        \renewcommand{\thefigure}{S\arabic{figure}}%
     }

% the abstract formatting
\newenvironment{sciabstract}{%
\begin{quote} \bf}
{\end{quote}}
\renewcommand\refname{References}

% margin sizes`
\topmargin 0.0cm
\oddsidemargin 0.2cm

\textwidth 16cm 
\textheight 21cm
\footskip 1.0cm


\title{Generality in food webs scales with species richness, not latitude}
\author{Alyssa Cirtwill, Daniel Stouffer, Tamara Romanuk}
\date{$^1$School of Biological Sciences\\University of Canterbury\\
Private Bag 4800\\Christchurch 8140, New Zealand}

\begin{document}
\maketitle
\baselineskip=8.5mm
 
\vspace{0.4 in}


\section*{Introduction} [[Add something about ecosystem type since it's in the methods]]

  %Food webs are a thing. (Mention gen, vul, LS.) %Food webs have strong scaling structure.
  Food webs --networks of feeding links between species-- have been used for several decades to summarize the structure of 
  ecological communities \citep{Petchey2008,Williams2000,Paine1980} and to understand how that structure relates to environmental variables 
  such as habitat type \citep{Shurin2006,Briand1983}, primary productivity \citep{Vermaat2009,Thompson2005a,Townsend1998}, and climate\citep{Baiser2012,Petchey2010}. The latter
  variables in turn have strong gradients over latitude, with productivity and temperature both being higher in the tropics while 
  climate is more variable at high latitudes \citep{mappaper}. These variables affect both the resources available and species' 
  metabolisms~\citep{,Hechinger2011}, and have been proposed as determinants of the strength of 
  interspecific interactions~\citep{Lang2012,Schleuning2012,Schemske2009}. 
  By modulating interactions between species, latitudinal gradients may also shape food-web structure.
  Indeed, these latitudinal environmental gradients have been put forward as potential drivers for the 
  latitudinal gradient in species richness, one of the most general and robust patterns in ecology \citep{Schemske2009a,Macpherson2002,Kaufman1995}.

  % These gradients, together with the strength and generality of associations between biological processes and environmental variables, suggest that variation in network structure 
  % may also be explained by ecological gradients \citep{Baiser2012}. 


  One proposed link between species richness and latitude is the ``latitude-niche breadth hypothesis''~\citep{,Vazquez2004}. 
  This hypothesis predicts that decreased seasonality in the tropics should lead to more stable populations, which in
  turn should evolve smaller niches \citep{Vazquez2004}. These narrow niches should therefore allow more species to 
  coexist in the tropics than at higher latitudes. Alternatively, higher productivity of the tropics \citep{Brown2004}
  may result in a broader niche space \citep{Davies2007} which could also allow greater diversification even if niche 
  sizes are globally similar. Although the assumptions of the latitude-niche breadth hypothesis are only equivocably 
  supported \citep{Vazquez2004}, it remains a compelling potential mechanism for the latitudinal gradient in species 
  richness \citep{Lappalainen2006,Krasnov2008,Slove2010}. 


  If decreased seasonality or higher productivity is the cause of higher species-richness in the tropics, then 
  there should be other effects of latitudinal gradients on food-web structure. Attempts to unravel these effects, however,
  are complicated by known relationships between species richness and many other network properties~\citep{Riede2010}. In particular,
  narrower niches imply fewer links per species (i.e., greater specialization in the tropics~\citep{}). However,
  average numbers of links per species tend to increase in larger food webs~\citep{Dunne2006,Riede2010}. This means
  that a latitudinal effect on specialization may be obscured by a latitudinal gradient in species richness. If this is
  the case, it may still be possible to uncover effects of latitude on specialization by examining the shape of the scaling 
  relationship between specialization and species richness over changing latitude. Here, we use three measures of specialization:
  mean links per species, mean generality (number of prey), and mean vulnerability (number of predators). By testing whether
  latitude affects the scaling of each property with species richness, we test effects of latitude on specialization implied by
  the ``latitude-niche breadth hypothesis''.



\section*{Methods}

\subsection*{Data Set} 

  We compiled a list of 163 empirical food webs from
  multiple sources (see Supplemental Information for web origins and selection
  criteria). We recorded study site latitude from the original source where
  possible or, where study sites were described but exact coordinates were not
  given, obtained estimated coordinates using Google Earth \citep{GoogleEarth}.
  We then divided the species in each web into basal resources (those species with
  consumers but no prey), top predators (those species with prey but no predators),
  and intermediate consumers (species with both predators and prey, including cannibalistic
  species). 

  % [[list in supplemental (347 GlobalWeb webs \citep{GlobalWeb}), supplemented 
  % with 31 food webs used to study the relationship between consumer and resource body sizes (SizeWebs \citep{Brose2006})
  % and seven food webs used to study the impact of parasites on network structure (ParWebs \citep{Dunne2013}).
  % Of the GlobalWeb webs, 122 were rejected because their original source could not be found or was unpublished (53), they were source or sink
  % webs rather than descriptions of an entire community (13), were focused on plant-pollinator, host-parasitoid, or 
  % competitive interactions (8, 34, and 2, respectively), described inferred interactions in an extinct community (2), were ``generalized 
  % schemes'' rather than being based on empirical observation (8), or because it was not clear which of a variety of 
  % described sites the published food web represented (2). (See Table S1 for a list of rejected webs and reasons for their
  % exclusion.) For three GlobalWeb webs which included parasites as a minor component of the community, as well as the 
  % seven ParWebs webs, we included a modified version of each web which included only free-living species and the 
  % interactions between them \citep{Dunne2013}. 
  % As we were considering the log of proportions of basal resources, top predators, and intermediate consumers, each web used in our analysis 
  % had to have at least one species in each trophic group.

  % of  the University of Canberra's GlobalWeb database (www.globalwebdb.com) 
  % \citep{}. Of these, 302 webs had accessible original sources. 
  % 18 source and sink webs, which describe links to or from a focal taxon, were excluded as they are likely 
  % to have different food web properties than webs attempting to
  % describe the entire community of a site \citep{Williams2002}, as were 8 plant-pollinator, 34 host-parasitoid, and 2 
  % competition-focused networks
  % \citep{Riede2010}. Of the remaining 230 food webs, we excluded two paleowebs (comprised of probable interactions 
  % between extinct or prehistoric species) \citep{Simenstad1978} and five ``generalized schemes'' 
  % \citep{Nybakken1982,Percival1929,Swan1961,Landry1977,Petipa1979,Harrison1963}, as 
  % we wished to measure
  % directly observed rather than inferred interactions. We further excluded a food web aggregated from several sites in 
  % the Pacific Ocean which were widely spread in both latitude and longitude and at markedly different successional stages
  % \citep{Vinogradov1978}, such that it is not clear which site the food web represents. 
  % This left 223 food webs with widely varying levels of taxonomic resolution, 11 of which contained humans. 


  As the food webs in this dataset are derived from a variety of sources and were compiled over many decades, it
  is likely that they vary in their resolution and in the amount of sampling effort invested in their assembly.
  Many analyses of food-web structure attempt to reduce this variation by aggregating species with identical predator and prey
  sets to form ``trophic species'' webs (e.g. ~\citep{Martinez1991,Vermaat2009,Dunne2004,Dunne2013}). As this study
  is concerned directly with the number of species at a particular latitude, however, we did not wish to ignore 
  species with redundant sets of interactions. We therefore analysed both original and trophic-species versions
  of the dataset; in each case using the number of species ($S$) and 
  links ($L$) in each web to calculate the mean number of links per species ($Z$), mean generality 
  ($G$), and mean vulnerability ($V$) of the web. 
  The version of the dataset used did not qualitatively change the results, suggesting that
  the scaling relationships between species richness, other food-web properties, and latitude are very 
  similar whether or not redundant species are included. For simplicity, we present only
  the results for the original webs in the main text. 


\subsection*{Relationships with Latitude}

  To determine whether there were latitudinal gradients in food-web structure,
  we first examined simple linear relationships between latitude and each of 
  $S$, $Z$, $G$, $V$, $B$, $I$, and $T$. We fit models of the form

  \begin{equation}
  \label{Latfull}
  S_{i} = \alpha_{0} + \alpha_{1} L_{i} + \alpha_{2} E_{i} + \alpha_{3} L_{i} E_{i} + \epsilon_{i} ,
  \end{equation}

  \noindent where $S_{i}$ is the species richness of web $i$, $L_{i}$ its absolute
  latitude (degrees north or south  regardless of direction), $E_{i}$ is a categorical
  variable indicating the ecosystem type of network $i$ (comprising terms for stream, 
  marine, lake, and terrestrial networks with estuarine
  networks providing the intercept) and $\epsilon_{i}$ a residual error term. 
  We next calculated the AIC
  of the maximal model, as well as the AIC's of a suite of candidate simplified models identified
  using the R~\citep{R} function dredge from package MuMIn~\citep{MuMIn}. 
  Simplified models were obtained by
  systematically removing all possible combinations of terms from the full model.
  The best-fitting model was then determined to be the model with the fewest terms 
  where $\Delta$AIC\textless2. If several models shared the fewest number of terms 
  and had $\Delta$AIC\textless2, the model with the lowest AIC in that set was chosen as the best-fit
  model.


\subsection*{Scaling Relationships with S}

  Next, we examined the form of the scaling relationship between each 
  property ($Z$, $G$, and $V$) and $S$. The scaling relationship between $Z$ and 
  $S$ has been shown to be a power law \citep{Riede2010} of the form 

  \begin{equation}
  \label{Power}
  Z_{i} \sim \alpha S_{i}^{\beta}  ,
  \end{equation}

  \noindent which is often re-expressed in logarithmic form 

  \begin{equation}
  \label{Loglog}
  \log{Z_{i}} \sim \log{\alpha} + \beta\log{S_{i}}  .
  \end{equation}


  \noindent Although these relationships are very similar, they imply different error distributions~\citep{Xiao2011}.
  Specifically, equation~(\ref{Power}) implies a normally-distributed, additive error and equation~(\ref{Loglog}) a lognormal,
  multiplicative error. As we have no \emph{a priori} reason to believe that our dataset has one error distribution
  over another, we follow the recommendations in \citet{Xiao2011} and compared the two
  model formulations explicitly. The model with the error distribution most resembling that observed in the empirical
  data was then used to test for potential effects of latitude.


  Although scaling relationships between $S$ and $G$ or $S$ and $V$ have not been explicitly examined (but see scaling 
  relationships for the standard devaitions of each property in \citet{Riede2010}), we expect that they will follow
  power laws similar to that of the relationship between $S$ and $Z$. This is because the links taken into account in
  calculating $G$ and $V$ are subsets of the total links included when calculating $Z$. As with $Z$, we explicitly 
  compared the error distributions of models for $G$ and $V$ using both the power-law and logarithmic formulations. 
  In each case, we used the best-fitting equation as a template when assessing the effect of latitude on scaling with
  species richness.



\subsection*{Effect of Latitude on Scaling}

  % Logarithm of a sum is awful:   log(a+c) = log(a) + log(1+b^[log(c)-log(a)]) But can probably feed that into R...


  We then assessed the impact of latitude on the scaling relationships between species richness and 
  link density, generality, vulnerability.
  [[If the species-rich tropics truly have narrower 
  niches~\citep{Brown2004}, then we would expect to see less increase in mean numbers of links, 
  predators, and prey with increasing species richness in the tropics. ]]
  In the context of the scaling relationships above, note that this implies that we aim to determine the effect of latitude on
  the scaling exponent $\beta$. As when examining the relationships between latitude and each food 
  web property directly, we included a categorical variable for ecosystem type (stream, lake, 
  terrestrial, marine, or estuary), as well as interactions between food web type and latitude.


  We therefore began by considering models of the form

  \begin{equation}
  \label{PowerLat}
  Z_{i}=\alpha S_{i}^{\beta_{0}+\beta_{1}L_{i}+\beta_{2}E_{i}+\beta_{3}LE_{i}} + \epsilon_{i} ,
  \end{equation}

  \noindent where $S_{i}$, $L_{i}$, and $E_{i}$ are as defined previously. The logarithmic formulation of this model is

  \begin{equation}
  \label{LogLat}
  \log{Z_{i}} = \log{\alpha}+\beta_{0}\log{S_{i}} + \beta_{1}L\log{S_{i}} +\beta_{2}E\log{S_{i}} +\beta_{3}LE\log{S_{i}} +\epsilon_{i} .
  \end{equation}

  For each property in link density, generality, and vulnerability, we used the form of the equation that was best supported
  when describing the scaling of the property with species richness alone~\citep{Xiao2011}.
  We then simplified each model following the same procedure as for the relationships between
  latitude and food-web properties, except that species richness was retained in all reduced models. 


\section*{Scaling by Trophic Levels}

  We were also interested in the ways that scaling relationships with species richness might
  be affected by changes to the distribution of species among trophic levels. To
  that end, we repeated all of the above analyses replacing species richness by
  proportion of basal resources $B$, proportion of intermediate consumers $I$,
  or proportion of top predators $T$. All model fitting and model
  simplification procedures were identical to those described for species
  richness.



\section*{Results}

\subsection*{Relationships with Latitude}

  Contrary to the expected latitudinal gradient, the best-fit version of
  equation~(\ref{Latfull}) for species richness did not include a signficant effect of latitude for any ecosystem type except streams
  ($\beta_{Latitude}$=0.095, $p$=0.626; $\beta_{Latitude:Stream}$=-1.69, $p$=0.007).  This
  relationship was robust to the removal of three outliers (based on Cook's Distance). In the best-fit versions of
  equation~(\ref{Latfull}) for link density and vulnerability, there were once again
  significant relationships with latitude in stream food webs, but after the removal of three outliers these trends
  were no longer significant. Link density and vulnerability did not vary with latitude in any other ecosystem type.
  Similarly, the best-fit version of equation~(\ref{Latfull}) for generality did not include any effect of
  latitude in any ecosystem type. 


  The relationships between the proportions of species at each trophic level with latitude were broadly similar
  to the relationship between species richness and latitude. 
  The proportion of basal resources increased towards the poles in stream webs ($\beta_{Latitude:Stream}$=0.012, $p$\textless0.001) 
  but did not vary with latitude in estuarine, marine, lake, or terrestrial 
  food webs ($\beta_{Latitude}$=0.001, $p$=0.139).
  The proportion of intermediate consumers did not vary with latitude in any ecosystem type.
  The proportion of top predators did not vary with latitude in 
  estuarine, marine, or terrestrial food webs ($\beta_{Latitude}$=0.001, $p$=0.136) but decreased towards the poles in stream and lake food 
  webs ($\beta_{Latitude:Stream}$=-0.006, $p$=0.030; 
  $\beta_{Latitude:Lake}$=-0.006, $p$=0.001). 
  All of these trends were robust to the removal of outliers.


\subsection*{Form of scaling relationships}

  When considering the relationships between species richness and proportions 
  of basal resources, intermediate consumers, and top predators and all response variables 
  (link density, generality, vulnerability), equation~(\ref{Loglog}) had a
  lower AIC than did equation~(\ref{Power}). This indicates that the
  data support an assumption of multiplicative lognormal error better than an
  assumption of additive normal error. That is, models where $\epsilon$ is
  modelled as an additive term on
  the logarithmic scale provide a better description of the data than models
  where $\epsilon$ is modelled as an additive term on the arithmetic scale.  
  We therefore used logarithmic-form models when assessing the
  effect of latitude on scaling relationships  with species richness.



\subsection*{Effect of Latitude on Scaling with S, B, I, and T}

Link density, generality, and vulnerability each increased with increasing
species richness (Fig.~\ref{S}; $\beta_0$=0.666, $p$\textless0.001; $\beta_0$=0.623,
$p$\textless0.001; and $\beta_0$=0.666, $p$\textless0.001 respectively).  For estuarine,
marine, and terrestrial food webs this increase did not vary with latitude
($\beta_{Latitude}$=-0.001, $p$=0.267 for link density; $\beta_{Latitude}$=-0.001, $p$=0.363 for generality;
and $\beta_{Latitude}$=-0.001, $p$=0.267 for vulnerability). In lake food webs, link density,
generality, and vulnerability all increased more quickly towards the poles
($\beta_{Latitude:Lake}$=0.005, $p$=0.008; $\beta_{Latitude:Lake}$=0.005,
$p$=0.002; and $\beta_{Latitude:Lake}$=0.005, $p$=0.008, respectively) In
stream food webs generality increased more rapidly towards the poles 
($\beta_{Latitude}$=0.007, $p$=0.002) while link density and vulnerability did not vary with latitude.


In general, link density, generality, and vulnerability decreased as the
proportion of basal resources in a web increased (Fig.~\ref{B}). For
estuarine, lake, and stream food webs these decreases were more gradual towards the
poles ($\beta_{Latitude}$=-0.007, $p$=0.006 for link density;
$\beta_{Latitude}$=-0.009, $p$=0.002 for generality; and
$\beta_{Latitude}$=-0.007, $p$=0.006 for vulnerability).
In marine food webs there was very little effect of latitude on
the strength of scaling ($\beta_{Latitude:Marine}$=0.009, $p$=0.047; 
$\beta_{Latitude:Marine}$=0.011, $p$=0.028; 
and $\beta_{Latitude:Marine}$=0.009, $p$=0.047 respectively),
while in terrestrial food webs link density, generality, and vulnerability
decreased more gradually towards the 
equator ($\beta_{Latitude:Terrestrial}$=0.013, $p$=0.002; 
$\beta_{Latitude:Terrestrial}$=0.015, $p$=0.001; 
and $\beta_{Latitude:Terrestrial}$=0.013, $p$=0.002).


In contrast to the proportion of basal resources, link density, generality,
and vulnerability tended to increase with increasing proportions of
intermediate consumers (Fig.~\ref{I}; ). The best-fit models for scaling of link
density and vulnerability with the proportion of intermediate consumers did
not include any effect of latitude on scaling in any ecosystem type. Generality
increased more rapidly towards the equator in estuarine, lake, marine, and
stream food webs ($\beta_{Latitude}$=-0.006, $p$=0.009) and varied little with
latitude in terrestrial food webs ($\beta_{Latitude:Terrestrial}$=0.008,
$p$=0.053). In lake and stream food webs, generality increased more slowly
with the proporiton of intermediate consumers such that, at low proportions,
the effect of latitude could negate or reverse the increase.


As with the proportion of basal resources; link density, generality, and
vulnerability decreased as the proportion of top predators increased. For
estuarine, lake, and marine food webs scaling of link density, generality, and
vulnerability did not vary significantly with latitude
($\beta_{Latitude}$=-0.001, $p$=0.527 for link density;
$\beta_{Latitude}$=-0.001, $p$=0.521 for vulnerability; the best-fit model for
generality did not include any terms for latitude in any ecosystem type). In
stream and terrestrial food webs, the decrease in link density and
vulnerability was less sharp towards the poles, although this trend was
significant only in streams ($\beta_{Latitude:Stream}$=0.015,
$p$\textless0.001 and $\beta_{Latitude:Terrestrial}$=0.006, $p$=0.069 for link
density; $\beta_{Latitude:Stream}$=0.015, $p$\textless0.001 and
$\beta_{Latitude:Terrestrial}$=0.006, $p$=0.067 for vulnerability).



\section*{Discussion}

The tendency of food-web structure to exhibit scaling relationships with
species richness has been well-established \citep{Dunne2004,Riede2010}. As
species richness in particular is also known to vary systematically over
latitude \citep{}, intuitively one might suspect that any relationship
between food-web properties such as generality might be due to the latitudinal
gradient in species richness. In this dataset, however, we did not find
overall latitudinal gradients in species richness, links per species, 
generality, vulnerability, or the proportions of food webs accounted for by 
basal resources, intermediate consumers, and top predators. 


The lack of latitudinal gradients in food-web properties in this dataset contrasts
strongly with other studies which have found consistent variation in species richness,
[[other properties]] \citep{}. As numbers of species and links included in a food web
vary strongly with sampling effort as well as with the underlying diversity of the study
area, it is possible that the lack of latitudinal trends here is a result of researchers
tending to expend similar amounts of sampling effort across studies. This could result in
food webs describing species-rich tropical communities omitting more species and links
than species-poor arctic communities. In addition, it is worth noting that gradients in
species richness are generally measured for a single taxonomic group at a time~\citep{}.
It is possible that these taxa are not well-represented in our food webs and that the
dominant taxa in them do not have an underlying latitudinal gradient in richness. In
either case, the lack of association between species richness
and latitude in any ecosystem type nevertheless means that any effect of latitude on scaling relationships
between species richness and other properties is not being driven by underlying variation in
species richness, allowing us to more clearly assess effects of latitude on scaling with 
species richness and proportions of species in different trophic levels.


Scaling of links per species, generality, and vulnerability with species
richness varied strongly across ecosystem types. In estuarine, marine, and
terrestrial food webs scaling of each property varied little with latitude.
This is consistent with the idea that species' niche breadths do not vary
systematically with temperature and productivity but that the niche space
might be larger in species-rich communities~\citep{Davies2007}. In lake food
webs, in contrast, scaling of each property was much stronger in high-latitude
food webs. In stream food webs, scaling of links per species and vulnerability
varied little over latitude while scaling of generality was stronger in high-
latitude food webs. This means that species in trophical freshwater
communities gain fewer additional feeding links per additional species in the
web and is consistent with the hypothesis that greater stability in the
tropics leads to narrower niches~\citep{Brown2004} and a higher proportion of
specialists~\citep{}.


In lake food webs, species in tropical food webs also gained fewer predators,
and therefore fewer links overall. Conversely, for species in high-latitude
food webs numbers of predators, prey, and links per species all increased
rapidly with increasing species richness.
As there were proportionally fewer top predators in high-latitude lakes yet
generality, vulnerability, and links per species all increased more rapidly than
near the equator, it appears that high-latitude predators are more likely to be 
generalists. This may reflect prey switching as
different food sources become available in different seasons~\citep{}.


Surprisingly species in high-latitude stream food webs, while gaining many
additional prey species per additional species in the food web, did not
receive any more predators or links than species in low-latitude streams. 
This may be partly due to the increasing proportion of the food web made up of
basal resources at higher latitudes. If any additional species is more likely to
be a basal resource at high latitudes, it will have predators but no prey.
Combined with the potential for prey-switching between seasonal resources, this 
could account for the sharper increase in generality with increasing species richness
in high-latitude streams. Vulnerability and mean links per species, on the other hand,
increased at the same rate across latitude. [[Still working on why]]


\section*{Conclusion}

% Our results were consistent with the latitude-niche breadth hypothesis, but only in certain ecosystem
% types. This suggests that different types of food webs respond differently to differences in climate
% driven by latitudinal gradients. [[A sentence or two explaining why this is important.]]


%\end{spacing}
\newpage
\bibliographystyle{jae}%Compile with jae.bst style file
\bibliography{noISN.bib}% your .bib file(s)

\newpage

% \section*{Tables}

% \begin{table}[!h]
% \caption{Intercepts and slopes for the best-fitting model describing the relationship between species-richness, mean links per species, mean generality, and mean vulnerability and latitude. Note that none
% of the best-fitting models included a term for latitude or an interaction between latitude and ecosystem type. Each `NA' indicates a term that was not included in the corresponding best-fit model.}
% \label{Latlms}
% \begin{tabular}{l | l l  l l  l l  l l}

% \hline
% \multirow{2}{*}{Property} & \multicolumn{2}{|c}{Intercept} & \multicolumn{2}{|c}{Stream} & \multicolumn{2}{|c}{Lake} & \multicolumn{2}{|c}{Terrestrial}\\
% & $\alpha_{0}$ & $p$-value & $\alpha_{2}$ & $p$-value & $\alpha_{2}$ & $p$-value & $\alpha_{2}$ &$p$-value\\
% \hline
% Species richness  & 34.6 & \textless0.001 & 22.4 & 0.167 \textless0.001 & NA & NA & NA & NA \\
% Links per species & 3.53 & \textless0.001 & NA & NA & NA & NA & NA & NA \\
% Generality        & 7.32 & \textless0.001 & NA & NA & -1.98 & 0.044 & -3.77 & \textless0.001 \\
% Vulnerability     & 3.53 & \textless0.001 & NA & NA & NA & NA & NA & NA \\
% \hline
% \end{tabular}
% \end{table}

% \begin{table}[!h]
% \caption{Intercepts and slopes for the best-fitting model describing the relationship between species-richness, mean links per species, mean generality, and mean vulnerability and latitude in trophic-species versions of food webs. Note that none
% of the best-fitting models included a term for latitude or an interaction between latitude and ecosystem type. `NA' a term that was not included in the corresponding best-fit model.}
% \label{Latlms}
% \begin{tabular}{l | l l  l l  l l  l l}

% \hline
% \multirow{2}{*}{Property} & \multicolumn{2}{|c}{Intercept} & \multicolumn{2}{|c}{Stream} & \multicolumn{2}{|c}{Lake} & \multicolumn{2}{|c}{Terrestrial}\\
% & $\alpha_{0}$ & $p$-value & $\alpha_{2}$ & $p$-value & $\alpha_{2}$ & $p$-value & $\alpha_{2}$ &$p$-value\\
% \hline
% Species richness  & 34.6 & \textless0.001 & 22.4 & 0.167 \textless0.001 & NA & NA & NA & NA \\
% Links per species & 3.53 & \textless0.001 & NA & NA & NA & NA & NA & NA \\
% Generality        & 7.32 & \textless0.001 & NA & NA & -1.98 & 0.044 & -3.77 & \textless0.001 \\
% Vulnerability     & 3.53 & \textless0.001 & NA & NA & NA & NA & NA & NA \\
% \hline
% \end{tabular}
% \end{table}


% \begin{table}[!h]
% \caption{Intercepts and slopes for best-fit models describing the effect of latitude on the scaling of mean links per species, mean generality, and mean vulnerability with species richness. Each `NA' indicates a term that was not included in the corresponding best-fit model. These models refer to the original versions of food webs. }
% \label{Bestfits}
% \begin{tabular}{l | l l  l l  l l }
% Effect          & \multicolumn{2}{c|}{Links per species} & \multicolumn{2}{c|}{Generality} 
%                 & \multicolumn{2}{c}{Vulnerability}\\
%                 & $\beta$ & $p$-value     &  $\beta$ & $p$-value    &  $\beta$ & $p$-value  \\
% \hline
% Intercept               &-0.453 & \textless0.001 & -0.249 & \textless0.001 & -0.453 & \textless0.001 \\
% log(species)            & 0.419 & \textless0.001 &  0.663 & \textless0.001 &  0.419 & \textless0.001 \\
% log(species):Lake       & 0.119 & \textless0.001 & -0.185 & 0.016          &  0.119 & \textless0.001 \\
% log(species):Marine     & 0.255 & \textless0.001 &  0.090 & 0.002          &  0.255 & \textless0.001 \\
% log(species):Terrestrial& 0.207 & 0.002          & -0.122 & \textless0.001 &  0.207 & 0.002 \\
% log(species):Latitude   & 0.004 & \textless0.001 & -0.001 & 0.199          &  0.003 & \textless0.001 \\
% log(species):Latitude:Marine&-0.004 & 0.007      &  NA    &  NA            & -0.005 & 0.007 \\
% log(species):Latitude:Terrestrial&-0.005 & 0.002 &  NA    &  NA            & -0.005 & 0.002 \\
% log(species):Latitude:Lake & NA & NA             &  0.006 & 0.002          &  NA    & NA    \\
% \hline
% $R^{2} (adjusted)$   & \multicolumn{2}{c|}{0.706} & \multicolumn{2}{c|}{0.658} & \multicolumn{2}{c}{0.706} \\
% \hline
% \end{tabular}
% \end{table}


% \begin{table}[!h]
% \caption{Intercepts and slopes for best-fit models describing the effect of latitude on the scaling of mean links per species, mean generality, and mean vulnerability with species richness. Each `NA' indicates a term that was not included in the corresponding best-fit model. These models refer to the trophic species versions of food webs. }
% \label{Bestfits}
% \begin{tabular}{l | l l  l l  l l }
% Effect          & \multicolumn{2}{c|}{Links per species} & \multicolumn{2}{c|}{Generality} 
%                 & \multicolumn{2}{c}{Vulnerability}\\
%                 & $\beta$ & $p$-value     &  $\beta$ & $p$-value    &  $\beta$ & $p$-value  \\
% \hline
% Intercept               & -0.561 & \textless0.001 & -0.299 & \textless0.001 & -0.561 & \textless0.001 \\
% log(species)            &  0.733 & \textless0.001 &  0.622 & \textless0.001 &  0.733 & \textless0.001 \\
% log(species):Lake       & -0.118 & 0.047          & -0.115 & 0.078          & -0.118 & 0.047 \\
% log(species):Marine     &  0.051 & 0.040          &  0.120 & \textless0.001 &  0.051 & 0.040 \\
% log(species):Stream     & -0.161 & 0.014          & -0.057 & 0.421          & -0.161 & 0.014 \\
% log(species):Latitude   & -0.001 & 0.081          & -0.001 & 0.273          & -0.001 & 0.081 \\
% log(species):Lat:Lake   &  0.006 & \textless0.001 &  0.006 & \textless0.001 &  0.006 & \textless0.001 \\
% log(species):Lat:Stream &  0.004 & 0.018          &  0.003 & 0.059          &  0.004 & 0.018 \\
% \hline
% $R^{2} (adjusted)$   & \multicolumn{2}{c|}{0.706} & \multicolumn{2}{c|}{0.658} & \multicolumn{2}{c}{0.706} \\
% \hline
% \end{tabular}
% \end{table}

\newpage

\section*{Figures}

\begin{figure}[h]
\includegraphics[width=.85\textwidth]{Figures/by_TL/scaling_with_S/proportions/S_latlines_nonts.eps}
\caption{Scaling of link density, generality, and vulnerability with species richness
varies across ecosystem types and over latitude. For each property we show 
prediction curves for food webs at the equator (darkest line) and at 
20\degree, 40\degree, 60\degree, and 80\degree (lightest line) from the 
equator. We did not include direction (i.e., North or South) in our analyses. 
All curves are based on N=163 empirical food webs.}
\label{S}
\end{figure}


\begin{figure}[!H]
\includegraphics[width=.9\textwidth]{Figures/by_TL/scaling_with_S/proportions/B_latlines_nonts.eps}
\caption{Scaling of link density, generality, and vulnerability 
with the proportion of basal resources in a food web
varies across ecosystem types and over latitude. For each property we show 
prediction curves for food webs at the equator (darkest line) and at 
20\degree, 40\degree, 60\degree, and 80\degree (lightest line) from the 
equator. We did not include direction (i.e., North or South) in our analyses. 
All curves are based on N=163 empirical food webs.}
\label{B}
\end{figure}

\newpage


\begin{figure}[h]
\includegraphics[width=.9\textwidth]{Figures/by_TL/scaling_with_S/proportions/I_latlines_nonts.eps}
\caption{Scaling of link density, generality, and vulnerability 
with the proportion of intermediate consumers in a food web
varies across ecosystem types and over latitude. For each property we show 
prediction curves for food webs at the equator (darkest line) and at 
20\degree, 40\degree, 60\degree, and 80\degree (lightest line) from the 
equator. We did not include direction (i.e., North or South) in our analyses. 
All curves are based on N=163 empirical food webs.}
\label{I}
\end{figure}


\begin{figure}[h]
\includegraphics[width=.9\textwidth]{Figures/by_TL/scaling_with_S/proportions/T_latlines_nonts.eps}
\caption{Scaling of link density, generality, and vulnerability 
with the proportion of top predators in a food web
varies across ecosystem types and over latitude. For each property we show 
prediction curves for food webs at the equator (darkest line) and at 
20\degree, 40\degree, 60\degree, and 80\degree (lightest line) from the 
equator. We did not include direction (i.e., North or South) in our analyses. 
All curves are based on N=163 empirical food webs.}
\label{T}
\end{figure}

\newpage

\begin{figure}[H]
\includegraphics[width=.9\textwidth]{Figures/by_TL/scaling_with_S/proportions/fitlines_nonts.eps}
\caption{Scaling relationships for link density, generality (number of prey), 
and vulnerability (number of predators) 
against the proportions of basal resources (\% Basal), intermediate consumers (\% Intermediate), top 
predators (\% Top), and species richness of a food web. 
For each relationship we show observed values corrected for
ecosystem type and latitude (grey circles), as well as the overall scaling relationship (N=163 food webs). }
\label{props_v_lat}
\end{figure}


\end{document}


