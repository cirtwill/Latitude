\documentclass[12pt]{article}  
\usepackage{amsmath}
\usepackage{gensymb}
\usepackage{url}
\usepackage[dvips]{graphicx}
\usepackage{multirow}
\usepackage{geometry}
\usepackage{pdflscape}
\usepackage[labelfont=bf]{caption}
\usepackage{setspace}
\usepackage{soul}

\usepackage[running]{lineno}

\usepackage[numbers,sort&compress]{natbib}

\newcommand{\expect}[1]{\left\langle #1 \right\rangle}
\newcommand{\etal}{\textit{et al.\ }}

\newcommand{\beginsupplement}{%
        \setcounter{table}{0}
        \renewcommand{\thetable}{S\arabic{table}}%
        \setcounter{figure}{0}
        \renewcommand{\thefigure}{S\arabic{figure}}%
     }

\newenvironment{sciabstract}{%
\begin{quote} \bf}
{\end{quote}}
\renewcommand\refname{References}

\topmargin 0.0cm
\oddsidemargin 0.2cm

\textwidth 16cm 
\textheight 21cm
\footskip 1.0cm
\begin{document}


\title{Latitudinal gradients in biotic niche breadth vary across ecosystem types}
\author{Alyssa R. Cirtwill$^{1,2}$, Daniel B. Stouffer$^{1}$, Tamara N. Romanuk$^{2}$}
\date{\small$^1$Centre for Integrative Ecology\\School of Biological Sciences\\University of Canterbury\\
Private Bag 4800\\Christchurch 8140, New Zealand \\
\medskip$^2$Department of Biology\\
Life Science Centre, Dalhousie University\\1355 Oxford St., P0 BOX 15000\\
Halifax NS, B3H 4R2, Canada\\}

\maketitle
\baselineskip=8.5mm
 
\vspace{-0.3 in}

% \linenumbers
\begin{spacing}{2.0}

\section*{Purpose for correction}

    A subset of food webs analysed for this manuscript (75 out of 263; see \emph{Table S2, Appendix S1} for details) had formatting errors or irregularities that caused them to be incorrectly processed by our original code. Some food webs had inconsistent labels that prevented sets of predators and prey from being correctly linked to the same species. Others listed interactions using categorical symbols; these interactions were not included in our original analyses.


    We have corrected these inconsistencies and repeated our analyses to ensure that our conclusions did not change. After re-analysis, our results were qualitatively identical to those originally published and our original conclusions hold. However, the specific values for correlations and scaling relationships given in the original results and discussion are incorrect. Below, we present the results section and the beginning of the last paragraph of the discussion (the only portion to require correction). All corrections are highlighted in bold. The original and corrected values may be compared in Tables~\ref{corr_comparison} and~\ref{beta_comparison}. We also show the original and corrected Figures 2-3.  All supplemental figures have also been updated; as with the main-text figures below, any differences were minor. Reference numbers are as in the original main text.


\section*{Results}

    Link density (mean number of feeding links per species), generality (mean
    number of prey per species), and vulnerability (mean number of predators per
    species) were strongly and positively correlated 
    \textbf{($R^2$=0.915 for link
    density and generality, $R^2$\textgreater0.999 for link density and
    vulnerability, and $R^2$=0.912 for generality and vulnerability; Table~\ref{corr_comparison})}
    . Contrary
    to the expected latitudinal gradient, the best-fit version of
    equation (1) did not include a significant
    effect of latitude on species richness for any ecosystem type. Further, there were no significant
    relationships between link density, generality, or vulnerability with
    latitude for any ecosystem type.


    Each measure of specialisation increased with increasing
    species richness 
    \textbf{($\beta_0$=0.688, $p$\textless0.001; $\beta_0$=0.658,
    $p$\textless0.001; and $\beta_0$=0.695, $p$\textless0.001, respectively; Table~\ref{beta_comparison}}
    Fig.~\ref{props_v_lat1}). For estuarine, marine, and terrestrial food webs the
    strength of this scaling did not vary with latitude
    \textbf{($\beta_{Latitude}$=-0.001, $p$=0.210 for link density;
    $\beta_{Latitude}$=-0.001, $p$=0.315 for generality; and
    $\beta_{Latitude}$=-0.001, $p$=0.205 for vulnerability; Table~\ref{beta_comparison};} 
    Fig.~\ref{S1}). In
    lake food webs, however, the scaling of each property was stronger towards
    the poles 
    \textbf{($\beta_{Latitude:Lake}$=0.003, $p$=0.031;
    $\beta_{Latitude:Lake}$=0.005, $p$=0.005; and
    $\beta_{Latitude:Lake}$=0.004, $p$=0.030, respectively; Table~\ref{beta_comparison})}
    . In stream food
    webs, generality increased more rapidly towards the poles
    \textbf{($\beta_{Latitude:Stream}$=0.005, $p$=0.008; Table~\ref{beta_comparison})}
    while link density and
    vulnerability did not vary with latitude (i.e., the interaction term 
    $\beta_{Latitude:Stream}$ was not retained in the best-fit models).    


\section*{Discussion}

  Importantly, while terrestrial communities are also strongly seasonal at
  high latitudes and can receive significant allochthonous
  inputs [46], terrestrial consumers tend to be
  morphologically specialised for feeding on particular prey [53].
  This means that primarily gape-limited aquatic consumers tend to be more
  generalist across all types of aquatic habitats than terrestrial 
  consumers [5, 53]. This trend also held in our data, although the difference between generality was not significant \textbf{(
    $\mu_{Aquatic}$=5.86, $\mu_{Terrestrial}$=3.75; $p$\textless0.001 for 
    $\mu_{Aquatic}$\textgreater$\mu_{Terrestrial}$; Table~\ref{corr_comparison})}. 
  The key to this argument is therefore whether freshwater ecosystems experience
  more severe seasonal variation than marine and estuarine ecosystems. 


% \bibliographystyle{prsb}
% \bibliography{abbreviated}

\clearpage

\section*{Tables}

\begin{table}[h!]
\caption{Original and corrected values for correlations, mean generalities, and a student's T-test for differences in mean generalities reported in the main text. The 'Location' column indicates the section in which the relationship is described. Original values are those given in the original publication of this article. Corrected values are those obtained after correcting formatting inconsistencies in several food webs.}
\label{corr_comparison}
\small
\begin{tabular}{l l | r | r}
Relationship & Location & Original & Corrected \\
\hline
$R^2_{link density:generality}$ & Results & 0.891 & 0.915 \\
$R^2_{link density:vulnerability}$ & Results & \textgreater0.999 & \textgreater0.999 \\
$R^2_{generality:vulnerability}$ & Results & 0.890 & 0.912 \\
$\mu_{generality_{Aquatic}}$ & Discussion & 5.47 & 5.86 \\
$\mu_{generality_{Terrestrial}}$ & Discussion & 3.82 & 3.75 \\
P($\mu_{generality_{Aquatic}}>\mu_{generality_{Terrestrial}}$) & Discussion & 0.007 & \textless0.001 \\
\end{tabular}
\end{table}


\begin{table}[h!]
\caption{Original and corrected values for scaling relationships between specialisation and species richness and the effect of latitude on the strength of these relationships. Original values are those given in the original publication of this article. Corrected values are those obtained after correcting formatting inconsistencies in several food webs. All values are referred to in the Results section of the main text.}
\label{beta_comparison}
\small
\begin{tabular}{l | r r | r r}
\multirow{2}{*}{Parameter} & \multicolumn{2}{c|}{Original} & \multicolumn{2}{c}{Corrected} \\
& $\beta$ & $p$ & $\beta$ & $p$ \\
\hline
\multicolumn{5}{c}{Scaling of specialisation with species richness} \\
\hline
$\beta_{0_{link density}}$ & 0.637 & \textless0.001 & 0.688 & \textless0.001 \\
$\beta_{0_{generality}}$ & 0.553 & \textless0.001 & 0.658 & \textless0.001 \\
$\beta_{0_{vulnerability}}$ & 0.637 & \textless0.001 & 0.695 & \textless0.001 \\
\hline
\multicolumn{5}{c}{Strength of scaling and latitude: estuarine, marine, terrestrial} \\
\hline
$\beta_{{Latitude}_{link density}}$ & -0.001 & 0.365 & -0.001 & 0.210 \\
$\beta_{{Latitude}_{generality}}$ & -0.001 & 0.535 & -0.001 & 0.315 \\
$\beta_{{Latitude}_{vulnerability}}$ & -0.001 & 0.366 & -0.001 & 0.205 \\
\hline
\multicolumn{5}{c}{Strength of scaling and latitude: lakes} \\
\hline
$\beta_{{Latitude}_{link density}}$ & 0.004 & 0.019 & 0.003 & 0.031 \\
$\beta_{{Latitude}_{generality}}$ & 0.005 & 0.004 & 0.005 & 0.005 \\
$\beta_{{Latitude}_{vulnerability}}$ & 0.004 & 0.018 & 0.004 & 0.030 \\
\hline
\multicolumn{5}{c}{Strength of scaling and latitude: streams} \\
\hline
$\beta_{{Latitude}_{link density}}$ & \multicolumn{2}{c|}{NA} & \multicolumn{2}{c}{NA} \\
$\beta_{{Latitude}_{generality}}$ & 0.007 & 0.001 & 0.005 & 0.008 \\
$\beta_{{Latitude}_{vulnerability}}$ & \multicolumn{2}{c|}{NA} & \multicolumn{2}{c}{NA} \\
\end{tabular}
\end{table}


\clearpage

\section*{Figures}

\setcounter{figure}{1}

\begin{figure}[h]
% \includegraphics*[width=\textwidth]{Figures/Figure2.eps}
\includegraphics*[width=.48\textwidth]{Figures/Figure2.eps}
\hspace{.02\textwidth}
\includegraphics*[width=.48\textwidth]{Figures/Figure2_updated.eps}

\caption{Scaling relationships for re-scaled link density, generality, and vulnerability 
relative to the species richness of a food web. Here we show the original relationships (left)
and the relationships after correcting several food webs (right). Link density, generality,
and vulnerability were each re-scaled to remove the effects of latitude and ecosystem
type. As these relationships take the form of power laws, we did this by dividing the food-web
property (e.g. link density) by species richness raised to an exponent including the 
effects of latitude and, where applicable, ecosystem type and the interaction between ecosystem
type and latitude. Note that in all cases estuarine food webs were treated as the baseline 
ecosystem type, but that at most two ecosystem types had interactions between ecosystem type and
latitude retained in the best-fit model (see \emph{Results} for specifics). For each relationship, 
we show the re-scaled values (white circles) as well as the overall scaling relationship using estuarine
ecosystems as a baseline (black line, N=196 food webs). For a figure with the uncorrected values,
see Fig.~\emph{S7}, \emph{Appendix S4}. Note that the corrected and original relationships are 
very similar, but the re-scaled values for some food webs have changed.
}
\label{props_v_lat1}
\end{figure}


\begin{figure}[h]
\includegraphics*[width=.48\textwidth]{Figures/Figure3.eps}
\hspace{.02\textwidth}
\includegraphics*[width=.48\textwidth]{Figures/Figure3_updated.eps}
% \includegraphics*[width=\textwidth]{Figure3.eps}
\caption{Changes to the scaling of link density, generality, and vulnerability with species richness 
across ecosystem types and over latitude. Here we show these changes for the original dataset (left) 
and after correcting several food webs (right). We show the estimated scaling exponent for species 
richness (black line) with its 95\% confidence interval (in grey), based on N=196 empirical food 
webs. Latitude is given in degrees from the equator regardless of direction. Note that the scaling
relationships with latitude were very similar for the original and corrected datasets.} 
\label{S1} 
\end{figure}

% \clearpage

% \section*{Corrected Figures}

% \setcounter{figure}{1}
% \begin{figure}[h]
% % \includegraphics*[width=\textwidth]{Figures/Figure2_updated.eps}
% \includegraphics*[width=\textwidth]{Figure2_updated.eps}
% \caption{
% Figure 2 caption correct in original.}
% \label{props_v_lat}
% \end{figure}


% \begin{figure}[h]
% % \includegraphics*[width=\textwidth]{Figures/Figure3_updated.eps}
% \includegraphics*[width=\textwidth]{Figure3_updated.eps}
% \caption{Figure 3 caption correct in original.} \label{S} \end{figure}


\end{spacing}

\end{document}
