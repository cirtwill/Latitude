\documentclass[12pt]{letter}

%\usepackage[britdate]{canterbury-letter}
\usepackage[britdate,alyssa-signature]{canterbury-letter}
\usepackage{times}
\usepackage{letterbib}
\usepackage{geometry}
\usepackage[round]{natbib}
\usepackage{graphicx}
\geometry{a4paper}
\usepackage[T1]{fontenc}
\usepackage[utf8]{inputenc}
\usepackage{authblk}
\usepackage[running]{lineno}
\usepackage{amsmath,amsfonts,amssymb}
% \usepackage[margin=10pt,font=small,labelfont=bf]{caption}

%\usepackage{natbib}
% \bibpunct[; ]{(}{)}{;}{a}{,}{;}

\newenvironment{refquote}{\bigskip \begin{it}}{\end{it}\smallskip}

\newenvironment{figure}{}


\position{PhD Candidate}
\department{School of Biological Sciences}
\location{Private Bag 4800}
\telephone{+64 3 364 2729}
\fax{+64 3 364 2590}
\email{alyssa.cirtwill@pg.canterbury.ac.nz}
\url{http://stoufferlab.org}
\name{Alyssa R. Cirtwill}

% \position{PhD student}
% \department{School of Biological Sciences}
% \location{Private Bag 4800}
% \telephone{+64 3 364 2729}
% \fax{+64 3 364 2590}
% \email{alyssa.cirtwill@pg.canterbury.ac.nz}
% \url{http://stoufferlab.org}
% \name{Ms. Alyssa R. Cirtwill}


\newcommand{\mytitle}{\emph{Specialisation in food webs scales with species richness but not with latitude}}
\newcommand{\myjournal}{\emph{Proceedings of the Royal Society B}}

\begin{document}

\begin{letter}{\bf Professor Spencer C.H. Barrett\\
               Editor-in-Chief\\
               Proceedings of the Royal Society B\\
               6-9 Carlton House Terrace\\
               London, UK\\
               SW1Y 5AG\\
                }

\opening{Dear Prof. Barrett:}

We are happy to invite you to consider our manuscript 
``\mytitle'' for publication in the Population and Community Ecology section of the \myjournal. 

Our study aims to combine two well-established but disparate ecological
themes, namely the analysis of macroecological gradients and the search for
general laws governing the structure of food webs (networks of who eats whom
within a community). In particular, we explored the possibility of a link
between the well-known latitudinal gradients in species richness of some taxa
(e.g., birds, terrestrial plants) and the scaling relationships between
species richness and food-web structure. With regard to the former, one
hypothesised mechanism for greater species richness in the tropics is the
latitude-niche-breadth hypothesis, which predicts that the greater stability
of temperature and productivity in the tropics will permit species to evolve
narrower niches and therefore more finely partition habitats. If this
hypothesis holds true, then tropical species should be more specialised---have
fewer feeding links---than their high-latitude counterparts. However, the
number of feeding links in which a species is involved is also known to
increase with the number of species in the community food web following a
power law. Therefore we also naively expect that species in tropical food webs
will participate in relatively many feeding links by virtue of their speciose
communities. To address this apparent paradox, we tested for an effect of
latitude on the strength of the relationship between specialisation and species 
richness.


In general, we found little evidence of an effect of latitude on the
relationship between specialisation and species richness, implying that
latitudinal variation in food-web structure may be driven purely by
latitudinal gradients in species richness. If diet breadth may be taken as a
measure of a species' Eltonian (biotic) niche, this means that food webs
in general do not conform to the latitude-niche-breadth hypothesis. Ecosystem
type, on the other hand, had a greater impact on scaling exponents, suggesting
that different processes may structure food webs in different habitats.
Indeed, freshwater food webs departed from the overall trend and showed strong
effects of latitude on scaling relationships. Only in these food webs was the
latitude-niche-breadth hypothesis.


By linking macroecology and food-web ecology, we believe that our study will
be of interest to ecologists, theory-focused biologists, and to general
readers of \myjournal. Our results are potentially applicable to researchers
exploring community structure and specialisation in a variety of different
systems and also provide a new community-wide perspective on the validity of
the latitude-niche-breadth hypothesis. We hope you will agree that this broad
relevance and the novelty of our results justify consideration of our
manuscript for \myjournal.


Thank you again for your consideration.

\closing{Regards,}


\end{letter}

% \newpage

% \setcounter{page}{1}

\end{document}
