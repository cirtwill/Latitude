\documentclass[12pt]{letter}

\usepackage[britdate,alyssa-signature]{canterbury-letter}
\usepackage{times}
\usepackage{letterbib}
\usepackage{geometry}
\usepackage[round]{natbib}
\usepackage{graphicx}
\geometry{a4paper}
\usepackage[T1]{fontenc}
\usepackage[utf8]{inputenc}
\usepackage{authblk}
\usepackage[running]{lineno}
\usepackage{amsmath,amsfonts,amssymb}
% \usepackage[margin=10pt,font=small,labelfont=bf]{caption}

%\usepackage{natbib}
% \bibpunct[; ]{(}{)}{;}{a}{,}{;}

\newenvironment{refquote}{\bigskip \begin{it}}{\end{it}\smallskip}

\newenvironment{figure}{}


\position{PhD Candidate}
\department{School of Biological Sciences}
\location{Private Bag 4800}
\telephone{+64 3 364 2729}
\fax{+64 3 364 2590}
\email{alyssa.cirtwill@pg.canterbury.ac.nz}
\url{http://stoufferlab.org}
\name{Alyssa R. Cirtwill}

% \position{PhD student}
% \department{School of Biological Sciences}
% \location{Private Bag 4800}
% \telephone{+64 3 364 2729}
% \fax{+64 3 364 2590}
% \email{alyssa.cirtwill@pg.canterbury.ac.nz}
% \url{http://stoufferlab.org}
% \name{Ms. Alyssa R. Cirtwill}


\newcommand{\mytitle}{\emph{Generality in food webs scales with species richness but not with latitude}}
\newcommand{\myjournal}{\emph{Proceedings of the Royal Society B}}

\begin{document}

\begin{letter}{\bf RE: \mytitle\\
               ~~~~~~~Cirtwill, Alyssa R; Stouffer, Daniel B; Romanuk, Tamara N\\
               \vspace*{20pt}
               Professor Spencer C.H. Barrett\\
               Editor-in-Chief, Proceedings of the Royal Society B\\
               6-9 Carlton House Terrace\\
               London, UK\\
               SW1Y 5AG\\
                }

\opening{Dear Prof. Barrett:}

We are happy to invite you to consider our manuscript 
``\mytitle'' for publication in the Population and Community Ecology section of the \emph{\myjournal}. 



Our study aims to combine two well-established but disparate ecological themes, namely the analysis of macroecological gradients and the search for general laws governing the structure of communities, as measured through food webs (networks of who eats whom). In particular, we were interested in the possibility of a link between the well-known latitudinal gradients in species richness of some taxa (e.g., birds, land plants) and the scaling relationships between species richness and food-web structure. With regard to the former, one  hypothesized mechanism for greater species richness in the tropics is the latitude-niche-breadth hypothesis, which predicts that the greater stability of temperature and productivity in the tropics will permit species to evolve narrower niches and therefore more finely partition habitats. If this hypothesis holds true, then tropical species should be more specialized than their high-latitude counterparts.

In terms of a food web, this means that we would naively expect that tropical species will have relatively fewer feeding links than polar species. However, the number of feeding links in which a species is involved is known to scale with the number of species in the community food web following a power law. Therefore we also expect that species in tropical food webs will participate in many feeding links by virtue of their speciose communities. This apparent paradox can be resolved by adding an effect of latitude to the exponent of the power law. We tested for the impact of such an exponent, as well as potential interactions between the effects of latitude and ecosystem type on scaling laws. 

We found that, in general, there was little effect of latitude on scaling relationships between diet breadth and species richness. If diet breadth may be taken as a measure of a species' Eltonian (biotic) niche, this means that our results are inconsistent with the latitude-niche-breadth hypothesis. Ecosystem type, on the other hand, had a greater impact on scaling exponents, suggesting that different processes may structure food webs in different habitats. 

Our results are the pinnacle of interestingness and you should publish them before we decide to give them to Nature instead. Everyone will love us and then the world will shower me with options for my future. That's how this works, right?




We believe that our study will be of interest not only to theory-focused biologists but also to general readers
of \emph{\myjournal} as it deals with global trends in the structure of ecological communities. 
Specifically, we address the impact of latitude and habitat type on known scaling relationships between empirical
food-web structure and species richness. Although latitude (as a proxy for temperature, seasonality, and primary 
productivity) is expected to drive trends in niche size (here measured as the number of feeding links for any 
given species), we instead show that habitat type has the stronger influence on scaling between niche size and 
species richness. This suggests that there may indeed be general rules governing the structure of food webs but 
that these rules differ between different habitat types.


Our paper thus contributes to the discussions of both the latitude-niche-breadth hypothesis and scaling
laws in food-web structure, as well as being potentially applicable to researchers exploring community 
structure and specialization in a variety of different systems. We hope you will agree that this broad relevance
and the novelty of our results justify our manuscript's inclusion in \emph{\myjournal}.

Thank you again for your consideration.

\closing{Regards,}


\end{letter}

% \newpage

% \setcounter{page}{1}

\end{document}