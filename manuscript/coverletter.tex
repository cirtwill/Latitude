\documentclass[12pt]{letter}

\usepackage[britdate,alyssa-signature]{canterbury-letter}
\usepackage{times}
\usepackage{letterbib}
\usepackage{geometry}
\usepackage[round]{natbib}
\usepackage{graphicx}
\geometry{a4paper}
\usepackage[T1]{fontenc}
\usepackage[utf8]{inputenc}
\usepackage{authblk}
\usepackage[running]{lineno}
\usepackage{amsmath,amsfonts,amssymb}
\usepackage[margin=10pt,font=small,labelfont=bf]{caption}

%\usepackage{natbib}
% \bibpunct[; ]{(}{)}{;}{a}{,}{;}

\newenvironment{refquote}{\bigskip \begin{it}}{\end{it}\smallskip}

\newenvironment{figure}{}


\position{PhD Candidate}
\department{School of Biological Sciences}
\location{Private Bag 4800}
\telephone{+64 3 364 2729}
\fax{+64 3 364 2590}
\email{alyssa.cirtwill@pg.canterbury.ac.nz}
\url{http://stoufferlab.org}
\name{Alyssa R. Cirtwill}

% \position{PhD student}
% \department{School of Biological Sciences}
% \location{Private Bag 4800}
% \telephone{+64 3 364 2729}
% \fax{+64 3 364 2590}
% \email{alyssa.cirtwill@pg.canterbury.ac.nz}
% \url{http://stoufferlab.org}
% \name{Ms. Alyssa R. Cirtwill}


\newcommand{\mytitle}{\emph{Title title title}}
\newcommand{\myjournal}{\emph{Proceedings of the Royal Society B}}

\begin{document}

\begin{letter}{{\bf RE: \mytitle\\
               ~~~~~~~Cirtwill, Alyssa R; Stouffer, Daniel B; Romanuk, Tamara N\\
               \vspace*{20pt}
               Professor Spencer C.H. Barrett\\
               Editor-in-Chief, Proceedings of the Royal Society B\\
               6-9 Carlton House Terrace\\
               London, UK\\
               SW1Y 5AG\\
                }

\opening{Dear Prof. Barrett:}

We are happy to invite you to consider our manuscript 
``\mytitle'' for publication in the Theoretical Biology section of the \emph{\myjournal}. 


We believe that our study will be of interest not only to theory-focused biologists but also to general readers
of \emph{\myjournal}} as it deals with global trends in the structure of ecological communities. 
Specifically, we address the impact of latitude and habitat type on known scaling relationships between empirical
food-web structure and species richness. Although latitude (as a proxy for temperature, seasonality, and primary 
productivity) is expected to drive trends in niche size (here measured as the number of feeding links for any 
given species), we instead show that habitat type has the stronger influence on scaling between niche size and 
species richness. This suggests that there may indeed be general rules governing the structure of food webs but 
that these rules differ between different habitat types.


Our paper thus contributes to the discussions of both the latitude-niche-breadth hypothesis of () and scaling
laws in food-web structure, as well as being potentially applicable to researchers exploring community 
structure and specialization in a variety of different systems. We hope you will agree that this broad relevance
and the novelty of our results justify our manuscript's inclusion in \emph{\myjournal}.

Thank you again for your consideration.

\closing{Regards,}


\end{letter}

% \newpage

% \setcounter{page}{1}

\end{document}