\documentclass[12pt]{article}  
\usepackage{amsmath}
\usepackage{gensymb}
\usepackage{url}
\usepackage[dvips]{graphicx}
\usepackage{multirow}
\usepackage{geometry}
\usepackage{pdflscape}
\usepackage[labelfont=bf]{caption}
\usepackage{setspace}
\usepackage[round]{natbib}
\usepackage{float}
\usepackage[running]{lineno}

% some cheats to reduce the need to type complicated bits and pieces
\newcommand{\expect}[1]{\left\langle #1 \right\rangle}
\newcommand{\etal}{\textit{et al.\ }}

\newcommand{\beginsupplement}{%
        \setcounter{table}{0}
        \renewcommand{\thetable}{S\arabic{table}}%
        \setcounter{figure}{0}
        \renewcommand{\thefigure}{S\arabic{figure}}%
     }

% the abstract formatting
\newenvironment{sciabstract}{%
\begin{quote} \bf}
{\end{quote}}

\renewcommand\refname{References}
% the abstract formatting

% margin sizes`
\topmargin 0.0cm
\oddsidemargin 0.2cm


% let's get some nature formatted citations
%\usepackage{overcite}
\textwidth 16cm 
\textheight 21cm
\footskip 1.0cm

\begin{document}

\title{Supplementary Information}
\author{Alyssa R. Cirtwill$^{1,2}$, Daniel B. Stouffer$^{1,3}$, Tamara N. Romanuk$^{2}$}
\date{\small$^1$School of Biological Sciences\\University of Canterbury\\
Private Bag 4800\\Christchurch 8140, New Zealand \\
\medskip$^2$Department of Biology\\
Life Science Centre, Dalhousie University\\1355 Oxford St., P0 BOX 15000\\
Halifax NS, B3H 4R2, Canada\\
\medskip$^3$Centre for Integrative Ecology\\School of Biological Sciences\\University of Canterbury\\
Private Bag 4800\\Christchurch 8140, New Zealand \\}



\maketitle
\baselineskip=8.5mm
% \linenumbers

\vspace{0.4 in}
\beginsupplement
% \begin{landscape}
\subsection*{Appendix S1: Details of Food Web Sources and Selection}

  We combined published food webs from~\citet{GlobaWeb},~\citet{Riede2011},
  and~\citet{Dunne2013}. Of the 358 food webs available
  from~\citet{Globalweb}, 59 were rejected because their original source was
  unpublished or could not be retrieved. We further eliminated 13 source or
  sink webs (which describe feeding links flowing from or to a particular
  resource or top predator rather than the interactions of an entire
  community), 8 plant-pollinator webs, 33 host-parasitoid webs, and 2 networks
  describing competitive interactions. Finally, we eliminated 2 webs
  describing  inferred interactions in an extinct community, 10 ``generalized
  schemes'' that were not based on direct field data, and 4 webs where it was
  not clear which of several described study sites were used to construct the
  food web (and therefore the latitude of the food web could not be
  included). This left us with 226 webs from~\citet{GlobalWeb}.


  To these, we added 7 webs from~\citet{Dunne2013} that were originally used
  to assess the roles of parasites within food webs. As all but 3 of the webs
  from~\citet{GlobalWeb} contained only free-living species, we removed the
  the parasites (and all interactions involving them) from these webs (and the
  3 webs containing parasites from~\citet{GlobalWeb}) leaving free-living
  species only. In addition, we included 30 of the 40 webs
  from~\citet{Riede2011}, which were originally collected to test the
  relationship between consumer and resources body sizes. Nine of the
  remaining webs in the~\citet{Riede2011} dataset were also included in
  the~\citet{GlobalWeb}  dataset, and the last web was derived from an
  unpublished source. This gave us a database of 263 webs. As we wished to use
  the logarithmic form of power-law relationships, we then eliminated any food
  web which did not have at least one each of basal resources, intermediate
  consumers, and top predators (to obtain  non-zero proportions of each
  category). Our final sample size was thereby reduced to 196 food webs.


% Tree hole and pitcher plant webs counted as "lakes", wetlands could be either lakesor streams depending on the study

 See Table S1 for a list of rejected webs and reasons for their
  exclusion. 


\subsection*{Appendix S2: Supplemental Methods and Results - scaling Relationships with S}

  \subsubsection*{Methods}

    The scaling relationships between link density ($Z$) and species richness ($S$)
    has been shown to be a power law~\citep{Riede2010} of the form 

    \begin{equation}
    \label{Power}
    Z_{i} \sim \alpha S_{i}^{\beta}  ,
    \end{equation}

    \noindent which is often re-expressed in logarithmic form 

    \begin{equation}
    \label{Loglog}
    \log{Z_{i}} \sim \log{\alpha} + \beta\log{S_{i}}  .
    \end{equation}

    \noindent Although these relationships are very similar, they imply different error distributions~\citep{Xiao2011}.
    Specifically, equation~(\ref{Power}) implies a normally-distributed, additive error and equation~(\ref{Loglog}) a lognormal,
    multiplicative error. As we have no \emph{a priori} reason to believe that our dataset has one error distribution
    over another, we follow the recommendations in~\citet{Xiao2011} and compared the two
    model formulations explicitly. The model with the error distribution most resembling that observed in the empirical
    data was then used to test for potential effects of latitude.


    Although scaling relationships between species richness and generality or
    species richness and vulnerability have not been explicitly examined (but see scaling 
    relationships for the standard devaitions of each property in~\citet{Riede2010}), we expect that they will follow
    power laws similar to that of the relationship between species richness and links per species.
    This is because the links taken into account in calculating generality and vulnerability are subsets 
    of the total links included when calculating links per species. As with links per species, we explicitly 
    compared the error distributions of models for generality and vulnerability using
    both the power-law and logarithmic formulations. 
    In each case, we used the best-fitting equation as a template when assessing the effect of latitude on scaling with
    species richness. We then repeated this analysis 
    to model the scaling of link density, generality, 
    and vulnerability with the proportions of basal 
    resources, intermediate consumers, and top predators
    in a food web.


  \subsubsection*{Results}

    When considering the relationships between species richness or the proportions 
    of basal resources, intermediate consumers, or top predators, and all response variables 
    (link density, generality, vulnerability), equation~(\ref{Loglog}) had a
    lower AIC than did equation~(\ref{Power}). This indicates that the
    data support an assumption of multiplicative lognormal error better than an
    assumption of additive normal error. That is, models where $\epsilon$ is
    modelled as an additive term on
    the logarithmic scale provide a better description of the data than models
    where $\epsilon$ is modelled as an additive term on the arithmetic scale.  
    We therefore used logarithmic-form models when assessing the
    effect of latitude on scaling relationships  with species richness.

\subsection*{Appendix S3: Supplemental Methods and Results - relationships with latitude}

  \subsubsection*{Methods}

    To determine whether there were latitudinal gradients in food-web structure,
    we first examined simple linear relationships between latitude and each of 
    species richness, links per species, generality, vulnerability, and proportions
    of basal resources, intermediate consumers, and top predators. We fit models of the form

    \begin{equation}
    \label{Latfull}
    S_{i} = \alpha_{0} + \alpha_{1} L_{i} + \alpha_{2} E_{i} + \alpha_{3} L_{i} E_{i} + \epsilon_{i} ,
    \end{equation}

    \noindent where $S_{i}$ is the species richness of web $i$, $L_{i}$ its absolute
    latitude (degrees north or south  regardless of direction), $E_{i}$ is a categorical
    variable indicating the ecosystem type of network $i$ (comprising terms for stream, 
    marine, lake, and terrestrial networks with estuarine
    networks providing the intercept) and $\epsilon_{i}$ a residual error term. 
    We next calculated the AIC
    of the maximal model as well as the AIC's of a suite of candidate simplified models identified
    using the R~\citep{R} function dredge from package MuMIn~\citep{MuMIn}. 
    Simplified models were obtained by
    systematically removing all possible combinations of terms from the full model.
    The best-fitting model was then determined to be the model with the fewest terms 
    where $\Delta$AIC\textless2. If several models shared the fewest number of terms 
    and had $\Delta$AIC\textless2, the model with the lowest AIC in that set was chosen as the best-fit
    model.


  \subsubsection*{Results}
    Contrary to the expected latitudinal gradient, the best-fit version of
    equation~(\ref{Latfull}) for species richness did not 
    include a signficant effect of latitude for any 
    ecosystem type. Further, there were no relationships
    between link density, generality, or vulnerability
    with latitude for any ecosystem type.


    The relationships between latitude and the proportions of species at each trophic level were more complex. 
    The proportion of basal resources increased towards the poles in stream webs ($\beta_{Latitude:Stream}$=0.011, $p$\textless0.001) 
    but did not vary with latitude in estuarine, marine, lake, or terrestrial 
    food webs ($\beta_{Latitude}$=0.001, $p$=0.080),
    while the proportion of intermediate consumers decreased towards the poles in all ecosystem types
    except lakes ($\beta_{Latitude}$=-0.003, $p$=0.015 and
    $\beta_{Latitude:Lake}$=0.005, $p$=0.016).
    The proportion of top predators in a food web, 
    meanwhile, decreased towards the poles in lake and 
    stream food webs but did not vary in other ecosystem
    types ($\beta_{Latitude}$=0.002, $p$=0.151; 
    $\beta_{Latitude:Lake}$=-0.006, $p$=0.001; and
    $\beta_{Latitude:Stream}$=-0.007, $p$=0.006).

    % Does this need to be taken into account? Do I mention
    % it later?
    % All of these trends were robust to the removal of outliers.
    % Was it? Do I even care?

\newpage

\subsection*{Appendix S4: Supplemental Figures}

\begin{figure}[h]
\centerline{\includegraphics*[width=.75\textwidth]{Figures/by_TL/scaling_with_S/proportions/fitlines_nonts_observed.eps}}
\caption{Scaling relationships for link density, generality, 
and vulnerability relative to species richness and the proportions of basal resources (\% Basal),
intermediate consumers (\% Intermediate), top predators (\% Top), and species richness of a food web. 
For each relationship, we show observed values (white circles) and 
a simplified form of the scaling relationship described in equation XX (\emph{Main Text}), neglecting 
any effects of habitat type or latitude (black line, N=163 food webs). See Fig. 1 (\emph{Main Text}) to compare with 
observed values correcting for the effects of habitat type and latitude. }
\label{props_v_lat_obs}
\end{figure}


\newpage


\begin{figure}[h]
\centerline{\includegraphics*[width=.8\textwidth]{Figures/by_TL/marginal/S_marginal_latitude_proportions.eps}}
\caption{Changes to the scaling of link density, generality, and vulnerability across ecosystem
types and over latitude. For each property we show the estimated scaling exponent for species richness (black
line) with its 95\% confidence interval (in grey),
based on N=163 empirical food webs. Latitude is 
given in degrees from the equator
regardless of direction.}
\label{S}
\end{figure}


\begin{figure}[!h]
\centerline{\includegraphics*[width=.8\textwidth]{Figures/by_TL/marginal/B_marginal_latitude_proportions.eps}}
\caption{Changes to the scaling of link density, generality, and vulnerability across ecosystem
types and over latitude. For each property we show the scaling exponent for the proportion of
basal resources (black line) with its 95\% confidence interval (in grey), based on N=163 
empirical food webs. Latitude is given in degrees
from the equator regardless of direction.}
\label{B}
\end{figure}

\newpage


\begin{figure}[h]
\centerline{\includegraphics*[width=.8\textwidth]{Figures/by_TL/marginal/I_marginal_latitude_proportions.eps}}
\caption{Changes to the scaling of link density, generality, and vulnerability across ecosystem
types and over latitude. For each property we show the scaling exponent for the proportion of
intermediate consumers (black line) with its 95\% confidence interval (in grey), based on 
N=163 empirical food webs. Latitude is given in degrees
from the equator regardless of direction.}
\label{I}
\end{figure}

\begin{figure}[!h]
\centerline{\includegraphics*[width=.8\textwidth]{Figures/by_TL/marginal/T_marginal_latitude_proportions.eps}}
\caption{Changes to the scaling of link density, generality, and vulnerability across ecosystem
types and over latitude. For each property we show the scaling exponent for the proportion of
top predators (black line) with its 95\% confidence interval (in grey), based on N=163 
empirical food webs. Latitude is given in degrees
from the equator regardless of direction.}
\label{T}
\end{figure}


\newpage


\bibliographystyle{jae}%Compile with jae.bst style file
\bibliography{noISN}% your .bib file(s)


\end{document}

