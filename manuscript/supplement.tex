\documentclass[12pt]{article}  
\usepackage{amsmath}
\usepackage{gensymb}
\usepackage{url}
\usepackage[dvips]{graphicx}
\usepackage{multirow}
\usepackage{geometry}
\usepackage{pdflscape}
\usepackage[labelfont=bf]{caption}
\usepackage{setspace}
\usepackage[round]{natbib}
\usepackage{float}
\usepackage[running]{lineno}

% some cheats to reduce the need to type complicated bits and pieces
\newcommand{\expect}[1]{\left\langle #1 \right\rangle}
\newcommand{\etal}{\textit{et al.\ }}

\newcommand{\beginsupplement}{%
        \setcounter{table}{0}
        \renewcommand{\thetable}{S\arabic{table}}%
        \setcounter{figure}{0}
        \renewcommand{\thefigure}{S\arabic{figure}}%
     }

% the abstract formatting
\newenvironment{sciabstract}{%
\begin{quote} \bf}
{\end{quote}}

\renewcommand\refname{References}
% the abstract formatting

% margin sizes`
\topmargin 0.0cm
\oddsidemargin 0.2cm


% let's get some nature formatted citations
%\usepackage{overcite}
\textwidth 16cm 
\textheight 21cm
\footskip 1.0cm


\title{Supplementary Information}
\author{Alyssa R. Cirtwill$^{1,2}$, Daniel B. Stouffer$^{1}$, Tamara N. Romanuk$^{2}$}
\date{$^1$School of Biological Sciences\\University of Canterbury\\
Private Bag 4800\\Christchurch 8140, New Zealand}
\medskip$^2$Department of Biology\\Life Science Centre\\
Dalhousie University\\1355 Oxford St.\\P0 BOX 15000\\
Halifax, NS B3H 4R2\\Canada\\}


\begin{document}
\maketitle
\baselineskip=8.5mm
% \linenumbers

\vspace{0.4 in}
\beginsupplement
\begin{landscape}
\subsection*{Appendix S1: Details of Food Web Sources and Selection}

  [[list in supplemental (347 GlobalWeb webs \citep{GlobalWeb}), supplemented 
  with 31 food webs used to study the relationship between consumer and resource body sizes (SizeWebs \citep{Brose2006})
  and seven food webs used to study the impact of parasites on network structure (ParWebs \citep{Dunne2013}).
  Of the GlobalWeb webs, 122 were rejected because their original source could not be found or was unpublished (53), they were source or sink
  webs rather than descriptions of an entire community (13), were focused on plant-pollinator, host-parasitoid, or 
  competitive interactions (8, 34, and 2, respectively), described inferred interactions in an extinct community (2), were ``generalized 
  schemes'' rather than being based on empirical observation (8), or because it was not clear which of a variety of 
  described sites the published food web represented (2). (See Table S1 for a list of rejected webs and reasons for their
  exclusion.) For three GlobalWeb webs which included parasites as a minor component of the community, as well as the 
  seven ParWebs webs, we included a modified version of each web which included only free-living species and the 
  interactions between them \citep{Dunne2013}. 
  As we were considering the log of proportions of basal resources, top predators, and intermediate consumers, each web used in our analysis 
  had to have at least one species in each trophic group.

  of  the University of Canberra's GlobalWeb database (www.globalwebdb.com) 
  \citep{}. Of these, 302 webs had accessible original sources. 
  18 source and sink webs, which describe links to or from a focal taxon, were excluded as they are likely 
  to have different food web properties than webs attempting to
  describe the entire community of a site \citep{Williams2002}, as were 8 plant-pollinator, 34 host-parasitoid, and 2 
  competition-focused networks
  \citep{Riede2010}. Of the remaining 230 food webs, we excluded two paleowebs (comprised of probable interactions 
  between extinct or prehistoric species) \citep{Simenstad1978} and five ``generalized schemes'' 
  \citep{Nybakken1982,Percival1929,Swan1961,Landry1977,Petipa1979,Harrison1963}, as 
  we wished to measure
  directly observed rather than inferred interactions. We further excluded a food web aggregated from several sites in 
  the Pacific Ocean which were widely spread in both latitude and longitude and at markedly different successional stages
  \citep{Vinogradov1978}, such that it is not clear which site the food web represents. 
  This left 223 food webs with widely varying levels of taxonomic resolution, 11 of which contained humans. 

\subsection*{Appendix S2: Supplemental Figures}

\newpage

\begin{figure}[H]
\includegraphics[width=.9\textwidth]{}
\caption{ }
\label{}
\end{figure}


\end{document}

