\documentclass[12pt]{article}  
\usepackage{amsmath}
\usepackage{gensymb}
\usepackage{url}
\usepackage[dvips]{graphicx}
\usepackage{multirow}
\usepackage{geometry}
\usepackage{pdflscape}
\usepackage[labelfont=bf]{caption}
\usepackage{setspace}
\usepackage[round]{natbib}
\usepackage{float}
\usepackage[running]{lineno}

% some cheats to reduce the need to type complicated bits and pieces
\newcommand{\expect}[1]{\left\langle #1 \right\rangle}
\newcommand{\etal}{\textit{et al.\ }}

\newcommand{\beginsupplement}{%
        \setcounter{table}{0}
        \renewcommand{\thetable}{S\arabic{table}}%
        \setcounter{figure}{0}
        \renewcommand{\thefigure}{S\arabic{figure}}%
     }

% the abstract formatting
\newenvironment{sciabstract}{%
\begin{quote} \bf}
{\end{quote}}

\renewcommand\refname{References}
% the abstract formatting

% margin sizes`
\topmargin 0.0cm
\oddsidemargin 0.2cm


% let's get some nature formatted citations
%\usepackage{overcite}
\textwidth 16cm 
\textheight 21cm
\footskip 1.0cm


\title{Supplementary Information}
\author{Alyssa R. Cirtwill$^{1,2}$, Daniel B. Stouffer$^{1}$, Tamara N. Romanuk$^{2}$}
\date{$^1$School of Biological Sciences\\University of Canterbury\\
Private Bag 4800\\Christchurch 8140, New Zealand}
\medskip$^2$Department of Biology\\Life Science Centre\\
Dalhousie University\\1355 Oxford St.\\P0 BOX 15000\\
Halifax, NS B3H 4R2\\Canada\\}


\begin{document}
\maketitle
\baselineskip=8.5mm
% \linenumbers

\vspace{0.4 in}
\beginsupplement
\begin{landscape}
\subsection*{Appendix S1: Details of Food Web Sources and Selection}

  We combined published food webs from~\citet{GlobaWeb} (GlobalWeb),~\citet{Brose2006} (Brose), 
  and~\citet{Dunne2013} (Dunne).
  Of the 385 food webs available from~\citet{Globalweb}, 122 were rejected because their original
  source was unpublished or could not be retrieved. We further eliminated 13 source or sink webs
  (which describe feeding links flowing from or to a particular resource or top predator rather
  than the interactions of an entire community), 8 plant-pollinator webs, 34 host-parasitoid webs,
  and 2 networks describing competitive interactions. Finally, we eliminated 2 webs describing 
  inferred interactions in an extinct community, 8 ``generalized schemes'' that were not based
  on direct field data, and 2 webs where it was not clear which of several described study sites
  were used to construct the food web (and therefore the latitude of the food web could not be 
  included). This left us with 156 webs from~\citet{GlobalWeb}.


  To these, we added 7 webs from~\citet{Dunne2013} that were originally used to assess the roles
  of parasites within food webs. As all but 3 of the webs from~\citet{GlobalWeb} contained only
  free-living species, we removed the the parasites (and all interactions involving them) from
  these webs (and the 3 webs containing parasites from~\citet{GlobalWeb}) leaving free-living 
  species only. In addition, we included the 96 webs from~\citet{Brose2006}, which were originally
  collected to XXXYYYY. This gave us a database of 259 webs. However, as we wished to use the
  logarithmic form of power-law relationships, we then eliminated any food web which did not have
  at least one each of basal resources, intermediate consumers, and top predators (to obtain non-zero
  proportions of each category). Our final sample size was thereby reduced to 163 food webs.



 (See Table S1 for a list of rejected webs and reasons for their
  exclusion.) 



\subsection*{Appendix S2: Supplemental Figures}

\newpage

\begin{figure}[H]
\includegraphics[width=.9\textwidth]{}
\caption{ }
\label{}
\end{figure}


\end{document}

