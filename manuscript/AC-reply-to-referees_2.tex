\documentclass[12pt]{letter}

%\usepackage[britdate]{canterbury-letter}
 \usepackage[britdate,alyssa-signature]{canterbury-letter}
\usepackage{times}
\usepackage{letterbib}
\usepackage{geometry}
\usepackage[numbers,sort&compress]{natbib}
\usepackage{graphicx}
\geometry{a4paper}
\usepackage[T1]{fontenc}
\usepackage[utf8]{inputenc}
\usepackage{authblk}
\usepackage[running]{lineno}
\usepackage{amsmath,amsfonts,amssymb}
% \usepackage[margin=10pt,font=small,labelfont=bf]{caption}

%\usepackage{natbib}
% \bibpunct[; ]{(}{)}{;}{a}{,}{;}

\newenvironment{refquote}{\bigskip \begin{it}}{\end{it}\smallskip}

\newenvironment{figure}{}


\position{PhD Candidate}
\department{School of Biological Sciences}
\location{Private Bag 4800}
\telephone{+64 3 364 2729}
\fax{+64 3 364 2590}
\email{alyssa.cirtwill@pg.canterbury.ac.nz}
\url{http://stoufferlab.org}
\name{Alyssa R. Cirtwill}

% \position{PhD student}
% \department{School of Biological Sciences}
% \location{Private Bag 4800}
% \telephone{+64 3 364 2729}
% \fax{+64 3 364 2590}
% \email{alyssa.cirtwill@pg.canterbury.ac.nz}
% \url{http://stoufferlab.org}
% \name{Ms. Alyssa R. Cirtwill}


\newcommand{\mytitle}{\emph{Latitudinal gradients in biotic niche breadth vary across ecosystem types.}}
\newcommand{\myjournal}{\emph{Proceedings of the Royal Society B}}

\begin{document}

\begin{letter}{\bf Dr. Daniel Costa\\
               Editor\\
               Proceedings of the Royal Society B\\
               6-9 Carlton House Terrace\\
               London, UK\\
               SW1Y 5AG\\
                }

\opening{Dear Dr. Costa:}

Thank you for accepting our manuscript ``\mytitle'', as well as for giving us
the opportunity to address a final round of comments from the reviewers. We
have followed their suggestions and are proud to submit our revision.
Thank you for contributing your time and effort as editor.


\closing{Regards,}


\end{letter}

\newpage

\setcounter{page}{1}


% -----------------------------------------------------------------------------
% -----------------------------------------------------------------------------
{\Large \bf Reply to Associate Editor}
% ---------------------------------------

  We thank the Associate Editor for his kind words and congratulations, as
  well as for the care and attention shown to our manuscript during the review
  process. We have followed the latest suggestions by the reviewers and are
  happy to acknowledge the Associate Editor and Reviewers' 
  contributions in the manuscript.


{\Large \bf Reply to Reviewer 1}

  We thank the Reviewer for their positive comments and have formally
  acknowledged their contributions in the manuscript. We are pleased that our
  jack-knife analysis has satisfied their concerns about variability between
  studies and appreciate the suggestion that we look into Riede et. al.'s 2011
  results. Although their results for lakes and streams are very interesting
  in the context of our study, our manuscript is already rather lengthy for
  \myjournal. As such, we have included only a very brief reference to Riede
  et. al.'s work. We hope that this will be enough to encourage future
  research exploring the links between specialisation, predator-prey biomass
  ratios, and latitude.
 

{\Large \bf Reply to Reviewer 2}

  We are pleased that our revised and extended Discussion has met with the Reviewer's approval and appreciate his further minor comments. We have followed all of the Reviewer's suggestions and acknowledge his contributions in the manuscript. With regard to the slight contradiction in the discussion, we have taken care to soften our statements on page 10 to make it clear that there is no latitudinal gradient in our dataset without implying that there is no latitudinal gradient in the true species richness of these ecosystems. We hope that this will remove the apparent contradiction in our discussion.


  \begin{quotation} 
  In either case, the lack of a strong association
  between observed species richness and latitude in any ecosystem type in our dataset
  means that any effect of latitude on other scaling relationships is not being driven by
  the scaling of specialisation with species richness in our food webs.
  \end{quotation}

\end{document}
