\documentclass[12pt]{letter}

\usepackage[britdate]{canterbury-letter}
% \usepackage[britdate,alyssa-signature]{canterbury-letter}
\usepackage{times}
\usepackage{letterbib}
\usepackage{geometry}
\usepackage[round]{natbib}
\usepackage{graphicx}
\geometry{a4paper}
\usepackage[T1]{fontenc}
\usepackage[utf8]{inputenc}
\usepackage{authblk}
\usepackage[running]{lineno}
\usepackage{amsmath,amsfonts,amssymb}
% \usepackage[margin=10pt,font=small,labelfont=bf]{caption}

%\usepackage{natbib}
% \bibpunct[; ]{(}{)}{;}{a}{,}{;}

\newenvironment{refquote}{\bigskip \begin{it}}{\end{it}\smallskip}

\newenvironment{figure}{}


\position{PhD Candidate}
\department{School of Biological Sciences}
\location{Private Bag 4800}
\telephone{+64 3 364 2729}
\fax{+64 3 364 2590}
\email{alyssa.cirtwill@pg.canterbury.ac.nz}
\url{http://stoufferlab.org}
\name{Alyssa R. Cirtwill}

% \position{PhD student}
% \department{School of Biological Sciences}
% \location{Private Bag 4800}
% \telephone{+64 3 364 2729}
% \fax{+64 3 364 2590}
% \email{alyssa.cirtwill@pg.canterbury.ac.nz}
% \url{http://stoufferlab.org}
% \name{Ms. Alyssa R. Cirtwill}


\newcommand{\mytitle}{\emph{Latitudinal gradients in biotic niche breadth vary across ecosystem types.}}
\newcommand{\myjournal}{\emph{Proceedings of the Royal Society B}}

\begin{document}

\begin{letter}{\bf Dr. Daniel Costa\\
               Editor\\
               Proceedings of the Royal Society B\\
               6-9 Carlton House Terrace\\
               London, UK\\
               SW1Y 5AG\\
                }

\opening{Dear Dr. Costa:}

We have now finished the revisions for the resubmission of our manuscript
``\mytitle'' for consideration in the Population and Community Ecology section
of the \myjournal. 

In your original decision, you encouraged us to resubmit a revised version of
the manuscript noting that, while the Reviewers were quite positive, they and
the Associate Editor had a few substantial concerns. We appreciate the
comments by the Associate Editor and the Referees, and have taken care to
address each one while keeping within the size limits set by \myjournal. We
feel that the revised manuscript is both stronger and more accessible, and
hope that it will please the Editors and Reviewers alike.


Thank you again for your consideration of our revised manuscript.

% \closing{Regards,}


\end{letter}

\newpage

\setcounter{page}{1}


% -----------------------------------------------------------------------------
% -----------------------------------------------------------------------------
{\Large \bf Reply to Associate Editor}
% ---------------------------------------

  The Associate Editor stated that both Referees found our study ``very
  interesting'', but echoed and added to their concerns. In particular, the
  Associate Editor requested a demonstration that our results are not
  ``artefacts of differences in sampling effort between studies''. To address
  this, we have taken the Associate Editor's advice and performed the extra
  analyses suggested by Referee 1 (described in detail below). We have also
  taken the Associate Editor's specific comments to heart and address each one
  in turn. We hope that our replies (preceded by \textbf{R:} 
  will assuage the Associate Editor's concerns and are 
  confident that our manuscript is strengthened as a result.


  1) Title doesn't reflect our trends in freshwater communities.

  \begin{refquote}

    Title: the current title seems to suggest that this scaling with latitude
    did never happen, while it was present in freshwater food webs.

  \end{refquote}


  \textbf{R:} We had originally intended to emphasize that, contrary to our expectations,
  scaling relationships did not vary over latitude in~\emph{most} ecosystems. We have now amended our
  title to ``\mytitle'' in order to better capture the scaling that did occur in streams.


  2) L29, L42: suggest the likely directions of effects.

  \begin{refquote}

    L29. To help the reader, please be explicit about the direction of these ‘direct relationships’.

    \smallskip

    L42. Can you make a directional prediction on how latitude is expected to
    affect the scaling? Otherwise, the study becomes rather exploratory
    presenting some phenomenological patterns not grounded in theory that need
    ad hoc explanations in the discussion. Similarly, can you provide a priori
    motivations why the relationship may differ among ecosystem types?

  \end{refquote}


  \textbf{R:} The goal of our study was to give support to
  one of two contrasting hypotheses (narrower niches in the
  tropics~\cite{Vazquez2004} versus broader niche space in 
  the tropics~\cite{Davies2007}). Given the limited support
  available for either hypothesis, we had no~\emph{a priori}
  expectation that one was more likely than the other-- that
  is, it was plausible that scaling of specialisation with
  species richness might be stronger towards the poles or 
  that scaling would be unaffected by latitude. We have 
  added to the introduction at lines 22-27 to make these two
  potential relationships clearer and at lines 37-42 to more
  explicitly frame our study as a test of these two versions 
  of the latitude-niche breadth hypothesis. We have also 
  followed Reviewer 2's suggestion and now provide a 
  conceptual figure (Fig. 1) that clearly places our study
  within this context. We also now provide an explanation 
  for why we included ecosystem type as a covariate at lines
  42-44.
 

  \begin{quotation}

    If the latitude-niche breadth hypothesis is correct, 
    there should also be direct relationships between 
    latitude and the degree of specialisation (i.e., the 
    breadth of the Eltonian 
    niche;~\cite{Elton1927,Leibold2010}) of species within
    food webs. Specifically, narrower niches in the tropics 
    would equate to greater specialisation (narrower niches) 
    while constant niche sizes but greater productivity 
    would translate to constant specialisation and niche 
    width across latitude (Fig. 1). 

    \smallskip

    If the scaling of specialisation with species richness 
    is weaker in the  tropics (i.e., if species gain fewer
    links, predators, or prey as the size  of the network
    increases), this will indicate narrower niches at the 
    tropics.  If, however, the scaling of specialisation 
    with species richness does not vary over latitude, this 
    will indicate that niches are similarly-sized worldwide 
    but that there is a broader niche space in the tropics. 
    Additionally, as food webs describing different 
    ecosystem types may differ in their 
    topology~\cite{Dunne2004,Shurin2006}, we also explored 
    the differences in scaling relationships across 
    ecosystem types.

  \end{quotation}


  3) L98-99: explain our model-selection procedure in more detail.

  \begin{refquote}

    L98-99. According to some experts (e.g. the landmark 2002 book by Burnham
    and Anderson) when models differ less than 2 units in AIC, they have the
    same support and fit. On what theoretical grounds do you motivate the
    choice to select the model with the lowest AIC as the best-fit model when
    models have $\delta$AIC\textless2? To what extent does this choice affect your
    conclusions?

  \end{refquote}

  \textbf{R:} We are also aware of the general consensus    that models where
  $\delta$AIC\textless2 are roughly equally well-supported. This is why our
  procedure (as stated in lines 83-86) was to first consider the set of models
  where $\delta$AIC\textless2 and select the model with the fewest degrees of
  freedom. We chose the smallest model where $\delta$AIC\textless2 in order to
  avoid over-fitting the models. 


  We only used AIC as a deciding factor where
  there were several models that shared the minimal degrees 
  of freedom and all had $\delta$AIC\textless2. In fact, 
  this scenario did not occur for the models presented in 
  the previous draft (it does occur when the 
  by-trophic-level models use counts of species rather than
  proportions, an analysis which we had explored in an 
  earlier draft). To avoid confusion we have therefore 
  removed the description of this step of our analysis from 
  the main text. 


  \begin{quotation}

    Simplified models were obtained by systematically 
    removing all possible combinations of terms from the 
    full model. The best-fitting model was then determined 
    to be the model with the fewest terms where 
    $\Delta$AIC\textless2, as this model is the least likely 
    to suffer from over-fitting. 

  \end{quotation}


  4) Move by-trophic-level analyses to appendix and/or better-explain our motivations.

  \begin{refquote}

    L100-106. After reading the introduction, it is not clear why you ‘were
    also interested’ in this topic. I suggest following the advice of reviewer
    2 and moving this part and the associated results to the appendix. Please
    also then explain why this is a priori interesting and relevant to
    investigate in the context of your study.

  \end{refquote}

  \textbf{R:} After carefully considering the Associate 
  Editor and Reviewer's comments, we have opted to remove
  the by-trophic-level analyses entirely. Due to length
  restrictions, we were not able to both expand our 
  description of the motivations for and implications of
  these analyses and accommodate the other changes suggested
  by the Associate Editor and Reviewers. We chose to focus
  on ensuring that the discussion of the species-richness
  results is as clear and complete as possible, and believe
  that our manuscript is stronger and more to-the-point
  after these changes. 


  5) L113: is there any covariance in the network properties?

  \begin{refquote}

    L113. To what extent do the property estimates covary with each other?
    This may be presented in an appendix.

  \end{refquote}


  \textbf{R:} As the Associate Editor likely suspected, the 
  property estimates are all strongly positively correlated. 
  As generality and vulnerability describe the two 
  components (number of prey and number of predators,
  respectively) of links per species, this is to be 
  expected. We have added a brief description of these 
  correlations to the main text (lines 122-125) as we 
  believe it will provide some context for the very similar 
  results for link density and vulnerability.

  \begin{quotation}

    Link density, generality, and vulnerability were strongly and positively
    correlated ($R^2$=0.891 for link density and generality,
    $R^2$\textgreater0.999 for link density and vulnerability, and $R^2$=0.890
    for generality and vulnerability).

  \end{quotation}


  6) L116: missing supporting statistics

  \begin{refquote}

    L116. It seems the stats to support this statement are missing.

  \end{refquote}
 

  \textbf{R:} We have ammended the text (lines 135-138) to support our statement.

  \begin{quotation}
    In stream food
    webs, generality increased more rapidly towards the poles
    ($\beta_{Latitude:Stream}$=0.007, $p$=0.001) while link density and
    vulnerability did not vary with latitude (i.e., the interaction term $\beta_{Latitude:Stream}$ was not retained in the best-fit models).
  \end{quotation}


  7) L187-198: provide citations for an absence of prey-switching in communities other than streams.

  \begin{refquote}

    L187-198. For your explanation why the scaling relationship changes with
    latitude in lake and stream food webs and not in the other ecosystem types
    to be convincing it is not enough to cite studies that high-latitude
    aquatic species tend to switch between different seasonally available
    prey, you also need to provide evidence from the literature that this is
    not the case in the other ecosystem types.

  \end{refquote}


  \textbf{R:} Following Reviewer 1's suggestion, we have 
    restructured our discussion
    to focus more on seasonality in different ecosystem types than on prey-switching.
    As we are not aware of any studies explicitly comparing the frequencies of 
    prey-switching in different ecosystem types, and do not feel qualified to conduct
    such a review ourselves, we hope that this revised discussion will be satisfactory.


  8) Ensure layout (especially citations) matches journal specifications.

  \begin{refquote}

    Please strictly follow the author guidelines with regard to layout (for
    example how to cite studies in the text).

  \end{refquote}

  \textbf{R:} We have changed all refences to the Vancouver style and have 
    added sections for author contributions, conflicts of interest, and funding.

  \newpage

% -----------------------------------------------------------------------------
% -----------------------------------------------------------------------------
{\Large \bf Reply to Reviewer 1}
% ---------------------------------------

  The Reviewer was very positive about our manuscript, calling it
  ``an interesting paper that takes advantage of recent compilations
  of food webs to test basic theory with respect to latitudinal gradients in
  niche breadth'', ``exceptionally well written and presented'', ``sophisticated
  and well thought out''. We thank the Reviewer for their kind words,
  and also for their suggestions for further improving our manuscript. We
  have addressed each comment below (in italics), and are confident that our
  responses (preceded by a \textbf{R}) have further improved the manuscript.
  

% ---------------------------------------
  \emph{
  1) Some trends may be due to differences in sampling effort, especially
  due to differences between Thompson \& Townsend stream food webs and other
  stream webs.}


  \begin{refquote}

    My concern here is how much of the pattern can be attributed to true
    ecological effects – as opposed to the consequences of differences in the
    effort used to describe different food webs... It could be explored to some extent by
    analysing for an effect of particular authors (e.g. the Thompson and
    Townsend food webs which dominate the stream dataset) but ultimately there
    is every chance that the results here are artefacts of differences in
    sampling effort between studies. Might it be possible to randomise the
    datasets, excluding particular food webs (or authors) to determine
    robustness to this effect?? 

  \end{refquote}


  \textbf{R:} We share the reviewer's concern that variation in sampling
  effort could be driving the trends we observe, as it is highly likely that
  the authors of all 196 webs in our dataset did not follow the same procedures.
  Following the reviewer's suggestion, we have performed a jackknife analysis by
  systematically removing each web in turn and then repeating our analysis. We
  also performed a similar process removing each set of 
  food webs compiled by a common author (or set of authors
  where all authors worked on identical sets of food webs).
  We are happy to report that removing
  any single web had no significant effect on any model, 
  nor did removing any set of webs compiled by a common 
  author. 
  We have added a description of these methods and summary
  of their results to the supporting information 
  (~\emph{Appendix S3}), and we thank the reviewer for 
  suggesting this validation of our overall results. We 
  have also added a brief note in the main text to direct 
  other readers with similar concerns to the supporting 
  information (lines 111-120). 


  We removed webs compiled by Thompson and by 
  Townsend separately as each author compiled one food
  web that the other did not. However, as 26/28 of the
  food webs compiled by Thompson~\emph{or} Townsend were
  compiled by both together, the results of both jackknife
  iterations were very similar (as seen in authors 7 and 6, 
  respectively, in~\emph{Figs. S4-S6, Appendix S3}). 
  We are particularly reassured by~\emph{Fig. S5}, where
  the removal of large subsets of the stream food webs
  had very little impact on the coefficient representing
  the effect of latitude on the scaling of generality with
  species richness in streams.


  \begin{quotation}

    As a supplemental check to ensure that variation in sampling effor across
    food webs was not responsible for the trends we observed, we then repeated
    our analyses using jackknifed data sets in which we 1) sequentially removed
    each food web in the dataset and 2) sequentially removed sets of food webs 
    associated with each author of the food webs in the dataset. Parameter 
    estimates varied very little across either series of jackknifes (\emph{Appendix S3}).

  \end{quotation}


   2) Emphasize our results for streams, include the possibility that they
   may be due to greater seasonality in stream food webs. 

  \begin{refquote}

    It is particularly interesting in that stream systems (as opposed to
    estuarine, marine or terrestrial food webs) are more likely to be affected
    by seasonal variability – and this is the mechanism evoked as underpinning
    the latitudinal gradient. I am surprised that the authors didn’t make more
    of this point – to some degree the difference in habitats supports their
    proposed mechanism, and this could be discussed.

  \end{refquote}


  \textbf{R:} We thank the Reviewer for this suggestion, 
  and have substantially revised our discussion (lines 
  182-217) to follow it. The Reviewer will note that our 
  arguments include lakes as well as streams since both 
  systems showed effects of latitude on scaling, and both 
  experience severe seasonal variability as the Reviewer
  points out for streams. We have also added a brief note 
  to address the potential that the New Zealand stream food 
  webs mentioned by the Reviewer above might be affecting 
  our results. As described below, they are grouped in a 
  narrow band of latitude; this likely explains their 
  minimal impact on the trends we observe. We think that 
  our discussion is much stronger after these changes.


  \begin{quotation}

    That freshwater food webs supported the hypothesis of 
    narrower niches in the tropics --while other ecosystem 
    types did not-- is noteworthy given that these
    ecosystems (especially streams) are known for being 
    highly variable and that seasonal variability is one of 
    the proposed drivers of the latitude-niche breadth 
    hypothesis~\cite{Vazquez2004}. Both streams and lakes
    can experience severe changes in water temperature and 
    volume (e.g., floods, drying, freezing) that remove 
    food or other resources (notably oxygen during
    freezing events)~\cite{Winterbourn1997,Meding2001}. 
    These events are often linked to seasonal events such 
    as snowmelts or summer drought~\cite{Winterbourn1997}). 
    Further, both temperate streams and lakes tend to 
    experience seasonal strong pulses of allochthonous 
    inputs (e.g., fallen leaves, terrestrial
    invertebrates~\cite{Nakano2001,Lennon2004,Zeng2008}. 
    These trends combined mean that, relative to estuarine 
    and marine communities, freshwater food webs may 
    experience high turnover in both community composition 
    and productivity~\cite{Tilzer1988,Magalhaes1993,Baird1989}. Notable exceptions from the above
    trends are New Zealand stream communities (representing 
    31 of the 71 stream food webs in our dataset), which 
    experience unpredictable flooding and drying throughout 
    the year and do not receive seasonally pulsed
    subsidies~\cite{Winterbourn1997,Winterbourn1981}. 
    However, as this subset of webs is very tightly grouped 
    in latitude ($44.64-46.41^{\circ}$S, within an
    overall range of $23-69.02^{\circ}$ for stream 
    communities), it is unlikely that they have greatly
    influenced our results (see also \emph{Appendix S3}). 
    Moreover, just as in highly-variable communities where 
    said variation is more seasonal, New Zealand 
    communities are dominated by ecological
    generalists~\cite{Winterbourn1997,Winterbourn1981} 
    implying that they appear to fit the general pattern of 
    streams worldwide.


    Importantly, while terrestrial communities are also 
    strongly seasonal at high latitudes and can receive 
    significant allochthonous inputs~\cite{Nakano2001}, 
    terrestrial consumers tend to be morphologically 
    specialised for feeding on particular 
    prey~\cite{Liem1990}. This means that primarily 
    gape-limited aquatic consumers tend to be more
    generalist across all types of aquatic habitats than 
    terrestrial consumers~\cite{Liem1990,Shurin2006}. The
    key to this explanation of the differences between 
    freshwater and marine and estuarine  ecosystems is 
    whether the former experience more severe seasonal 
    variation. Although we are not aware of any study 
    explicitly comparing seasonal variation in freshwater 
    and saltwater or brackish food webs in a similar 
    location, we believe that freshwater ecosystems are 
    indeed likely to  experience more severe changes 
    because of their small size. While oceans and estuaries 
    certainly vary in terms of water temperature and
    nutrients over the course of a year~\cite{Baird1989}, 
    these changes are likely to be slower and milder than 
    in freshwaters because marine and estuarine communities 
    are buffered by being open to the ocean rather than 
    isolated in the midst of a terrestrial matrix. Net 
    primary productivity in particular is much more 
    variable over the course of a year in non-marine 
    communities~\cite{Field1998}, suggesting that niche 
    breadths may also be more variable over the course of 
    the year.

  \end{quotation}


  \newpage

% -----------------------------------------------------------------------------
% -----------------------------------------------------------------------------
{\Large \bf Reply to Reviewer 2}
% ---------------------------------------

  The Reviewer stated that our study is interesting and covers ``a timely and
  important topic. The core analysis of the paper (scaling relationships of
  food web metrics with species richness) are convincingly presented and
  comprehensively discussed.'' Nevertheless, the Reviewer has suggested several
  ways in which our manuscript could be improved. We have responded to each suggestion
  in turn, and believe that our replies (preceded by \textbf{R}) have further improved
  the manuscript.


% ---------------------------------------

  1) Network terminology and methods might be inaccessible to researchers
  without network background.

  \begin{refquote}

    the accessibility to readers outside the food-web community could be
    improved by a more detailed explanation of the methods, especially the
    applied network metrics and underlying scaling relationships; a conceptual
    figure visualizing the expected relationships for different communities
    according to the latitude-niche breadth hypothesis could be very helpful
    here

    \smallskip

     For readers outside the field, please explain in more detail the
     difference between `original species' and `trophic species' in food webs.

  \end{refquote}


  \textbf{R:} We thank the reviewer for their concern that our manuscript
  should be accessible to as wide a variety of researchers as possible, and
  have followed the reviewer's suggestions in several places in the manuscript.
  We now include a conceptual figure to clearly and succinctly link the changes
  to scaling relationships we expected to these two versions of the latitude-niche
  breadth hypothesis (Fig. 1).


  We have explained ``trophic species webs" more thoroughly
  in lines 58-60 and added parenthetical definitions of
  trophic-species and orignal webs in lines 62-64. We have also added parenthetical definitions
  of link density, generality, and vulnerability at the begining of the results section
  to remind readers of the biological meaning of these measures.

  \begin{quotation}

    Many analyses of food-web structure attempt to reduce 
    this variation by using food webs comprised of ``trophic
    species'' --aggregations of species with identical sets 
    of predators and prey-- rather than species 
    \emph{per se}~\cite{Martinez1991,Dunne2004,Vermaat2009,Dunne2013}.

    \smallskip

    We therefore analysed both original (i.e., without 
    aggregating any species) and trophic-species (i.e., 
    after aggregating species with identical predators and 
    prey) versions of the dataset ...

    \smallskip

    Link density (mean number of feeding links per 
    species), generality (mean number of prey per species), 
    and vulnerability (mean number of predators per 
    species) were strongly and positively correlated ...

  \end{quotation}


  2) Reconsider by-trophic-level analysis.

  \begin{refquote}

    the additional analyses of scaling relationships by trophic levels and
    latitude-species richness relationships are not well introduced. If the
    analysis on scaling relationships by trophic levels was maintained, then
    it should already be mentioned in the Introduction (including specific
    expectations for this analysis) and would require a more comprehensive
    discussion.

    \smallskip

    The abstract is incomplete. Results of the analysis on the scaling
    relationships by trophic levels are missing.

    \smallskip

    Please justify why you use the proportion of species within a trophic
    level and not the absolute number of species per trophic level; at a first
    glance, this would be more intuitive for me given the focus of the study
    on scaling relationships with species richness.

  \end{refquote}


  \textbf{R:} We appreciate the Reviewer's concerns about 
  the integration of these results into the main text. Due 
  to length restrictions, we were unable to explain our 
  motivations and discuss our results more fully in the main
  text. Moreover,
  referring briefly to by-trophic-level results in the supplemental
  information repeated the problems with lack of justification and
  poor integration of the by-trophic-level results with the 
  species-richness results. To that end we have opted to remove
  these results entirely. Instead, we have prioritised enriching
  our discussion of the species-richness results, clarifying our
  methods, and including the latitudinal gradient in species-richness
  as requested below. We hope that, on balance, this restructuring has
  resulted in a clearer and more well-integrated manuscript.


  3) Move description of the species-richness gradient to methods and results.


  \begin{refquote}

     Analyses of species richness along latitude should already be mentioned
     and described in the Methods and should be referred to in the Results
     section instead of the Discussion.

     \smallskip

      The start of the Discussion confused me since it presents another
      additional analysis. This analysis on the latitudinal gradient in
      species richness is interesting and important for the interpretation of
      the patterns, but should already be presented in Methods and Results
      section.

  \end{refquote}


  \textbf{R:} Given the lack of significant relationships,
  we had originally viewed this analysis as a very minor 
  part of our paper, but are happy to follow the Reviewer's 
  suggestion and integrate it in the methods and
  results along with our main analyses. We now describe 
  the methods for this analysis in lines 74-86 and 
  summarize our results in lines 125-128.


  \begin{quotation}

    \textbf{Gradients over Latitude}

    To put our dataset in the context of other research on latitudinal gradients in species richness,
    we first examined simple linear relationships between latitude and each of 
    species richness, links per species, generality, vulnerability, and proportions
    of basal resources, intermediate consumers, and top predators. We fit models of the form

    \begin{equation}
    \label{Latfull}
    S_{i} = \alpha_{0} + \alpha_{1} L_{i} + \alpha_{2} E_{i} + \alpha_{3} L_{i} E_{i} + \epsilon_{i} ,
    \end{equation}

    \noindent where $S_{i}$ is the species richness of web $i$, $L_{i}$ its absolute
    latitude (degrees north or south  regardless of direction), $E_{i}$ is a categorical
    variable indicating the ecosystem type of network $i$ (comprising terms for stream, 
    marine, lake, and terrestrial networks with estuarine
    networks corresponding to $E_{i}=0$) and $\epsilon_{i}$ is a residual error term. 
    We next calculated the AIC
    of the maximal model as well as the AIC's of a suite of candidate simplified models identified
    using the R~\cite{R} function dredge from package MuMIn~\cite{MuMIn}. 
    Simplified models were obtained by
    systematically removing all possible combinations of terms from the full model.
    The best-fitting model was then determined to be the model with the fewest terms 
    where $\Delta$AIC\textless2, as this model is the least likely to suffer from over-fitting. 


    \smallskip

    Contrary to the expected latitudinal gradient, the 
    best-fit version of equation (1) did not include a 
    significant effect of latitude for any ecosystem type. 
    Further, there were no significant relationships 
    between link density, generality, or vulnerability with
    latitude for any ecosystem type.

  \end{quotation}


  4) Explain habitat categories.

  \begin{refquote}

    Please describe in more detail why you used the chosen habitat categories
    and provide the number of webs per habitat type also in the main
    manuscript.

  \end{refquote}


  \textbf{R:} We followed the habitat categories given by the publications which
  aggregated the food webs. Although more detailed descriptions were also provided, 
  we opted to use the set of five more general habitat types in order to have a 
  greater number of food webs in each category.


  We have added a brief explanation, and the number of webs in each ecosystem type, at lines 52-55:

  \begin{quotation}
    We grouped food web by ecosystem type (stream, N=71; lake, N=47; marine, N=28; estuarine, N=18;
    and terrestrial, N=31) according to their designation in previous aggregations of food webs 
    (i.e.,~\citet{GlobalWeb,Riede2011,Dunne2013}).
    
  \end{quotation}


  5) The y-xais of Fig. 1 is unclear, top predators are unlabelled.


  \begin{refquote}

    Please re-label the y-axis on Fig. 1 to improve clarity. The top predators
    are missing in this figure although they are mentioned in the legend.

  \end{refquote}


  \textbf{R:} We thank the reviewer for pointing this out 
  and, while the inclusion of top predators specifically
  is moot since these results have been moved to the 
  supplemental information, have taken care that all labels
  are included in all figures. We have also re-labelled the
  y-axes on Fig. 2 (the old Fig. 1) to improve clarity.


  6) Add generality, vulnerability to Fig. 2 in main text.


  \begin{refquote}

    I would suggest adding generality and vulnerability to Fig. 2 in the main
    manuscript (instead of showing them in Fig. S2 only).

  \end{refquote}


  \textbf{R:} We have followed the reviewer's suggestion.


  \newpage


\bibliographystyle{geb}%Compile with jae.bst style file
\bibliography{noISBN.bib}

\end{document}
