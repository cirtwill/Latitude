\documentclass[12pt]{letter}

\usepackage[britdate]{canterbury-letter}
% \usepackage[britdate,alyssa-signature]{canterbury-letter}
\usepackage{times}
\usepackage{letterbib}
\usepackage{geometry}
\usepackage[round]{natbib}
\usepackage{graphicx}
\geometry{a4paper}
\usepackage[T1]{fontenc}
\usepackage[utf8]{inputenc}
\usepackage{authblk}
\usepackage[running]{lineno}
\usepackage{amsmath,amsfonts,amssymb}
% \usepackage[margin=10pt,font=small,labelfont=bf]{caption}

%\usepackage{natbib}
% \bibpunct[; ]{(}{)}{;}{a}{,}{;}

\newenvironment{refquote}{\bigskip \begin{it}}{\end{it}\smallskip}

\newenvironment{figure}{}


\position{PhD Candidate}
\department{School of Biological Sciences}
\location{Private Bag 4800}
\telephone{+64 3 364 2729}
\fax{+64 3 364 2590}
\email{alyssa.cirtwill@pg.canterbury.ac.nz}
\url{http://stoufferlab.org}
\name{Alyssa R. Cirtwill}

% \position{PhD student}
% \department{School of Biological Sciences}
% \location{Private Bag 4800}
% \telephone{+64 3 364 2729}
% \fax{+64 3 364 2590}
% \email{alyssa.cirtwill@pg.canterbury.ac.nz}
% \url{http://stoufferlab.org}
% \name{Ms. Alyssa R. Cirtwill}


\newcommand{\mytitle}{\emph{Specialisation in food webs scales with species richness but not with latitude}}
\newcommand{\myjournal}{\emph{Proceedings of the Royal Society B}}

\begin{document}

\begin{letter}{\bf Dr. Daniel Costa\\
               Editor\\
               Proceedings of the Royal Society B\\
               6-9 Carlton House Terrace\\
               London, UK\\
               SW1Y 5AG\\
                }

\opening{Dear Dr. Costa:}

We have now finished the revisions for the resubmission of our manuscript
``\mytitle'' for consideration in the Population and Community Ecology section
of the \myjournal. 

In your original decision, you encouraged us to resubmit a revised version of
the manuscript noting that, while the Reviewers were quite positive, they and
the Associate Editor had a few substantial concerns. We appreciate the
comments by the Associate Editor and the Referees, and have taken care to
address each one while keeping within the size limits set by \myjournal. We
feel that the revised manuscript is both stronger and more accessible, and
hope that it will please the Editors.


Thank you again for your consideration of our revised manuscript.

% \closing{Regards,}


\end{letter}

\newpage

\setcounter{page}{1}


% -----------------------------------------------------------------------------
% -----------------------------------------------------------------------------
{\Large \bf Reply to Associate Editor}
% ---------------------------------------

  The Associate Editor stated that both Referees found our study ``very
  interesting'', but echoed and added to their concerns. In particular, the
  Associate Editor requested a demonstration that our results are not
  ``artefacts of differences in sampling effort between studies''. To address
  this, we have taken the Associate Editor's advice and performed the extra
  analyses suggested by Referee 1 (described in detail below). We have also
  taken the Associate Editor's specific comments to heart and address each one
  in turn. We hope that our replies (preceded by \textbf{R:} [[do something
  good]].


  1) Title doesn't reflect our trends in freshwater communities.

  \begin{refquote}

    Title: the current title seems to suggest that this scaling with latitude
    did never happen, while it was present in freshwater food webs.

  \end{refquote}


  \textbf{R:} We had originally intended to emphasize that, contrary to our expectations,
  scaling relationships did not vary over latitude in most ecosystems. We have amended our
  title to ``[[newtitle]]'' in order to better capture the scaling that did occur in streams.


  2) L29, L42: suggest the likely directions of effects.

  \begin{refquote}

    L29. To help the reader, please be explicit about the direction of these ‘direct relationships’.

    \smallskip

    L42. Can you make a directional prediction on how latitude is expected to
    affect the scaling? Otherwise, the study becomes rather exploratory
    presenting some phenomenological patterns not grounded in theory that need
    ad hoc explanations in the discussion. Similarly, can you provide a priori
    motivations why the relationship may differ among ecosystem types?

  \end{refquote}


  \textbf{R:} Providing \emph{a priori} expectations for the directions of relationships between latitude and niche width/specialisation is difficult due to the 
  contrasting hypotheses of narrower niches in the tropics~\citep{Vazquez2004} versus broader niche space in the tropics~\citep{Davies2007}. Given the limited support available for either hypothesis, it was equally likely that scaling of specialization with species richness might be stronger or weaker towards the poles. 
  We do not share the Associate Editor's concern that this makes our study too exploratory but rather had intended to use the results of our analyses to support one or the other of these hypotheses. We have added to the manuscript at lines XX and YY to more explicitly frame our study as a test of these two versions of the latitude-niche breadth hypothesis.

 
  Line 29:

  \begin{quotation}

  If the latitude-niche breadth hypothesis is correct, there should also be
  direct relationships between latitude and the degree of specialisation
  (i.e., the breadth of the Eltonian niche;~\citealp{Elton1927}) of species within
  food webs. Specifically, narrower niches in the tropics would equate to
  greater specialisation (narrower niches) while constant niche sizes but
  greater productivity would translate to constant specialisation and niche
  width across latitude.

  \end{quotation}

  \smallskip

  Line 43:

  \begin{quotation}

  By testing whether latitude affects the scaling of each property with
  species richness, we test for the effects of latitude on specialisation
  implied by the ``latitude-niche breadth hypothesis''. If the scaling of
  specialisation with species richness is weaker at the tropics (i.e., if
  species gain fewer links, predators, or prey as the size of the network
  increases), this will indicate narrower niches at the tropics. If, however,
  the scaling of specialisation with species richness does not vary over
  latitude, this will support the hypothesis that niches are similarly-sized
  worldwide but that there is a broader niche space in the tropics.

  \end{quotation}


  3) L98-99: explain our model-selection procedure in more detail.

  \begin{refquote}

    L98-99. According to some experts (e.g. the landmark 2002 book by Burnham
    and Anderson) when models differ less than 2 units in AIC, they have the
    same support and fit. On what theoretical grounds do you motivate the
    choice to select the model with the lowest AIC as the best-fit model when
    models have $\delta$AIC\textless2? To what extent does this choice affect your
    conclusions?

  \end{refquote}

  \textbf{R:} We are also aware of the general consensus    that models where
  $\delta$AIC\textless2 are roughly equally well-supported. This is why our
  procedure (as stated in lines XX-XX) was to first consider the set of models
  where $\delta$AIC\textless2 and select the model with the fewest degrees of
  freedom. We chose the smallest model where $\delta$AIC\textless2 in order to
  avoid over-fitting the models. We only used AIC as a deciding factor where
  there were several models that shared the minimal degrees of freedom and all
  had $\delta$AIC\textless2.  In fact, this scenario did not occur for the
  models presented in the previous draft   (it does occur when the by-trophic-
  level models use counts of species rather than proportions), so we have simply
  removed the description of this step of our analysis from the main text. [[If we go with species counts, will need to describe what the differences are between the similar models]]
  

  4) Move by-trophic-level analyses to appendix and/or better-explain our motivations.


  \begin{refquote}

    L100-106. After reading the introduction, it is not clear why you ‘were
    also interested’ in this topic. I suggest following the advice of reviewer
    2 and moving this part and the associated results to the appendix. Please
    also then explain why this is a priori interesting and relevant to
    investigate in the context of your study.

  \end{refquote}

  \textbf{R:} We have followed Reviewer 2's suggestion and 
  moved these additional analyses to the supplemental 
  iformation. This analysis was purely exploratory. The 
  scaling of specialisation with proportions (or numbers) 
  of species at different trophic levels has not been 
  previously investigated (to our knowledge), and it 
  seemed to us to be a logical extension of the main 
  analysis. If the Associate Editor disagrees in principle 
  with exploratory science, we are willing to remove these 
  analyses altogether.


  5) L113: is there any covariance in the network properties?

  \begin{refquote}

    L113. To what extent do the property estimates covary with each other?
    This may be presented in an appendix.

  \end{refquote}

  \textbf{R:} As the Associate Editor likely suspected, the property estimates
  are all strongly    positively correlated. As generality and vulnerability
  describe the two components (number of prey and number    of predators,
  respectively) of links per species, this is   to be expected. We have added a
  brief description of this correlation to the main text as we believe it   will
  provide some context for the very similar results for link density and
  vulnerability.

  \begin{quotation}

    Link density, generality, and vulnerability were strongly and positively
    correlated ($R^2$=0.891 for link density and generality,
    $R^2$\textgreater0.999 for link density and vulnerability, and $R^2$=0.890
    for generality and vulnerability).

  \end{quotation}


  6) L116: missing supporting statistics

  \begin{refquote}

    L116. It seems the stats to support this statement are missing.

  \end{refquote}
 

  \textbf{R:} We have ammended the text (lines XX-XX) to support our statement.

  \begin{quotation}
    In stream food
    webs, generality increased more rapidly towards the poles
    ($\beta_{Latitude:Stream}$=0.007, $p$=0.001) while link density and
    vulnerability did not vary with latitude (i.e., the interaction term $\beta_{Latitude:Stream}$ was not retained in the best-fit models).
  \end{quotation}


  7) L187-198: provide citations for an absence of prey-switching in communities other than streams.

  \begin{refquote}

    L187-198. For your explanation why the scaling relationship changes with
    latitude in lake and stream food webs and not in the other ecosystem types
    to be convincing it is not enough to cite studies that high-latitude
    aquatic species tend to switch between different seasonally available
    prey, you also need to provide evidence from the literature that this is
    not the case in the other ecosystem types.

  \end{refquote}


  \textbf{R:} [[Try to find evidence that this doesn't happen, otherwise I guess we're just going to have to hedge like summbitches??]]


  8) Ensure layout (especially citations) matches journal specifications.

  \begin{refquote}

    Please strictly follow the author guidelines with regard to layout (for
    example how to cite studies in the text).

  \end{refquote}

  \textbf{R:} We have changed all refences to the Vancouver style and have added sections for author contributions, conflicts of interest, and funding.
  [[Check the in-text citations match published papers]]

  \newpage

% -----------------------------------------------------------------------------
% -----------------------------------------------------------------------------
{\Large \bf Reply to Reviewer 1}
% ---------------------------------------

  Polite pre-amble.

  ``This is an interesting paper that takes advantage of recent compilations
  of food webs to test basic theory with respect to latitudinal gradients in
  niche breadth. The paper is exceptionally well written and presented, and this
  is a credit to the    authors. The analysis is of a hitherto unattempted size,
  and is sophisticated and well thought out.''

  ``The paper is remarkable for a lack of grammatical and other errors."

% ---------------------------------------
  \emph{
  1) Some trends may be due to differences in sampling effort, especially
  due to differences between Thompson \& Townsend stream food webs and other
  stream webs.}


  \begin{refquote}

    My concern here is how much of the pattern can be attributed to true
    ecological effects – as opposed to the consequences of differences in the
    effort used to describe different food webs... It could be explored to some extent by
    analysing for an effect of particular authors (e.g. the Thompson and
    Townsend food webs which dominate the stream dataset) but ultimately there
    is every chance that the results here are artefacts of differences in
    sampling effort between studies. Might it be possible to randomise the
    datasets, excluding particular food webs (or authors) to determine
    robustness to this effect?? 

  \end{refquote}


  \textbf{R:} We share the reviewer's concern that variation in sampling
  effort could be driving the trends we observe. To ensure that this is not
  the case, we have performed several tests. We have included a brief
  allusion to these tests in the main text and a fuller explanation in the
  supplemental information.

  [[Details of the tests]]


  \begin{quotation}


  \end{quotation}


   2) Emphasize our results for streams, include the possibility that they
   may be due to greater seasonality in stream food webs. 

  \begin{refquote}

    It is particularly interesting in that stream systems (as opposed to
    estuarine, marine or terrestrial food webs) are more likely to be affected
    by seasonal variability – and this is the mechanism evoked as underpinning
    the latitudinal gradient. I am surprised that the authors didn’t make more
    of this point – to some degree the difference in habitats supports their
    proposed mechanism, and this could be discussed.

  \end{refquote}


  \textbf{R:} [[We'll need to look for citations re: seasonal effects in streams.]]


  \begin{quotation}


  \end{quotation}


  \newpage

% -----------------------------------------------------------------------------
% -----------------------------------------------------------------------------
{\Large \bf Reply to Reviewer 2}
% ---------------------------------------

  Polite pre-amble.
  ``This is an interesting study on a timely and important topic. The core analysis of the paper (scaling relationships of food web metrics with species richness) are convincingly presented and comprehensively discussed.''

% ---------------------------------------

  1) Network terminology and methods might be inaccessible to researchers
  withoout network background.

  \begin{refquote}

    the accessibility to readers outside the food-web community could be
    improved by a more detailed explanation of the methods, especially the
    applied network metrics and underlying scaling relationships; a conceptual
    figure visualizing the expected relationships for different communities
    according to the latitude-niche breadth hypothesis could be very helpful
    here

    \smallskip

     For readers outside the field, please explain in more detail the
     difference between `original species' and `trophic species' in food webs.

  \end{refquote}


  \textbf{R:} We thank the reviewer for their concern that our manuscript
  should be accessible to as wide a variety of    researchers as possible, and
  have followed the reviewer's suggestions in several places in the manuscript.
  [[sent DBS a draft of the conceptual figure]]
  

  At lines XX-XX we have added parenthetical explanations of the original and 
  trophic-species versions of food webs.

  \begin{quotation}

    We therefore analysed both original (i.e., without aggregating any species) and
    trophic-species (i.e., after aggregating species with identical predators and
    prey) versions of the dataset;

  \end{quotation}


  2) Move by-trophic-level analysis to supplemental information, consider
  numbers of species rather than proportions.

  \begin{refquote}

    the additional analyses of scaling relationships by trophic levels and
    latitude-species richness relationships are not well introduced. If the
    analysis on scaling relationships by trophic levels was maintained, then
    it should already be mentioned in the Introduction (including specific
    expectations for this analysis) and would require a more comprehensive
    discussion.

    \smallskip

    The abstract is incomplete. Results of the analysis on the scaling
    relationships by trophic levels are missing.

    \smallskip

    Please justify why you use the proportion of species within a trophic
    level and not the absolute number of species per trophic level; at a first
    glance, this would be more intuitive for me given the focus of the study
    on scaling relationships with species richness.

  \end{refquote}

  \textbf{R:} Following the Referee's suggestion, we have moved our analysis
  of scaling relationships by trophic levels to the supplemental information.
  We hope that this will address the Referee's first two concerns above. We thank the
  Referee for this suggestion as it has allowed us more space to describe our network
  measures in more accessible terms.


  In fact we did analyse absolute numbers of species within a trophic level as well 
  as proportions. Both ways of defining the distribution of species across trophic levels
  have benefits (absolute numbers are more intuitive, as the Referee points out, while
  proportions feature in more of the food-web literature) and so we ultimately decided
  on proportions because the trends were clearer than in absolute numbers. [[Should we just
  give both now that this is in the supplemental?]]



  3) Move description of the species-richness gradient to methods and results.


  \begin{refquote}

     Analyses of species richness along latitude should already be mentioned
     and described in the Methods and should be referred to in the Results
     section instead of the Discussion.

     \smallskip

      The start of the Discussion confused me since it presents another
      additional analysis. This analysis on the latitudinal gradient in
      species richness is interesting and important for the interpretation of
      the patterns, but should already be presented in Methods and Results
      section.

  \end{refquote}

  \textbf{R:} [[Going to be hard to fit this in gracefully.]]


  4) Explain habitat categories.

  \begin{refquote}

    Please describe in more detail why you used the chosen habitat categories
    and provide the number of webs per habitat type also in the main
    manuscript.

  \end{refquote}


  \textbf{R:} We followed the habitat categories listed in [[foodwebdb.org]].
  Although more detailed descriptions were also    provided, we opted to use the
  set of five more general habitat types in order to have a greater number of
  food webs in each category.


  5) The y-xais of Fig. 1 is unclear, top predators are unlabelled.


  \begin{refquote}

    Please re-label the y-axis on Fig. 1 to improve clarity. The top predators
    are missing in this figure although they are mentioned in the legend.

  \end{refquote}

  \textbf{R:} [[Sure fine whatever]]


  6) Add generality, vulnerability to Fig. 2 in main text.

  \begin{refquote}

    I would suggest adding generality and vulnerability to Fig. 2 in the main
    manuscript (instead of showing them in Fig. S2 only).

  \end{refquote}

  \textbf{R:} [[That's going to be ugly]]





  \newpage


\bibliographystyle{geb}%Compile with jae.bst style file
\bibliography{noISBN.bib}

\end{document}
